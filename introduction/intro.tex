% https://explorationsofstyle.com/2013/02/20/structuring-a-thesis-introduction/

% provide a quick trip through the whole project in the first few paragraphs, before beginning to contextualize in earnest

% The first step will be a short version of the three moves, often in as little as three paragraphs, ending with some sort of transition to the next section where the full context will be provided.

%\section{Summary of the thesis}

The fact that human languages appear to have an inherent propensity to undergo change is well-known to linguists and observant laypeople alike.
But while much quantative data on the unfolding and spread of individual changes has been collected over the past century, the underlying mechanisms by which new conventions arise and spread within individual speech communities, as well as how the micro-level dynamics give rise to recurring macro-level patterns across languages, are still not fully understood.
Even though changes of a similar kind do appear to re-occur across both related and unrelated languages and language families, and general tendencies in how languages are organised can be found all across the globe, the sporadic and haphazard occurrence of changes leads to the diversification of languages rather than convergence.

One fundamental question regarding language change is why it occurs at all: linguistic conventions draw their power from the fact that they are shared -- both used and understood -- by a group of speakers. While the initial emergence of a communication system and the introduction of new meaningful distinctions into an existing one can be of advantage to its users, most instances of language change do not entail such straightforward improvements, 
but rather exhibit circular patterns that leave the languages as expressive as before the change. Negotiating the replacement of one working convention by another entails some effort for a speech community for no clear functional-communicative gain.
The fact that languages \emph{do} undergo change, in combination with what is known about the very \emph{directed} nature with which linguistic innovations spread through speech communities have led to a number of theories, `explanations' and `accounts' of language change.
% both generally and in particular.
%In order to understand the nature of the phenomenon at hand

The first central task of this thesis is to give an overview of the many different accounts of and approaches to language change that have been proposed in the literature. Although these different accounts make reference to many different `factors', `pressures' and `biases' that are thought to be driving or influencing the spread of novel linguistic variants at the expense of existing conventions, I will argue that many seemingly different approaches are conceptually very similar in that they all rely on a presumed \emph{asymmetry} between the variants in competition.\index{asymmetry}
% what isn't yet well understood
Much work on language change has focussed on identifying asymmetries, primarily by gathering empirical evidence showing that not all linguistic conventions are equally preferred and that not all language changes are equally likely, both based on the analysis of historical changes in communities as well as the experimental testing of preferences in individuals. 
But the relative influence of the many different pressures identified and particularly the inconsistencies with which those pressures apply or don't apply at specific points in time is often left unaccounted for.

The fundamental problem is that, although language changes appear to go down the same paths over and over again, changes are not predictable in any strong sense.
While many of the identified asymmetries embody strong \emph{universal} constraints on language change, for example by identifying strong unidirectional patterns of change, conclusions are mostly limited to specifying on the \emph{macro}-level whether a general type of change is more or less likely to happen to one language than another, relative to other changes.
% macro-level: if this change happens rather than that change you can only make claims about that change/incoming variant relative to other changes/incoming variants (i.e. asymmetry in innovation)
The question of why a \emph{particular} change occurs at all, why it does so exactly when it does (as opposed to earlier, later, or not at all), as well as how the underlying pressures give rise to the \emph{micro}-level patterns of its diffusion cannot easily be explained through universal asymmetries.
% micro-level: if this change happens rather than nothing you can make (some) claims about the selection of the change/incoming variants (at that point only?)
For decades, linguists have been struggling to bridge the disconnect between the idiosyncracies of individual language changes that result in the vast linguistic variation we see in the world today, and the fact that language changes tend to follow similar trajectories.
While it is possible to identify a pool of possible changes that a language is likely to undergo, languages appear to have some arbitrary choices over which of those path to go down, and at what point.

%Causal explanations of particular changes lack generalisability to other instances of language change, while accounts based on strong general pressures fail to account for the idiosyncracy of languages.

In order to address this problem, this thesis takes an explicit evolutionary approach to language change, as change by replication of concrete linguistic conventions.
%Adopting the particular framework proposed by \citet{Croft2000} I argue that the separate consideration of mechanisms responsible for the \emph{innovation} or introduction of new variants, as opposed to their subsequent \emph{selection} and spread within a community, can help account for both the common patterns as well as the idiosyncratic nature of language changes.
% what I'm actually gonna do
In line with \citet{Croft2000} I will argue that both the micro- and macro-level dynamics of language change can be explained by separating out pressures that are responsible for the \emph{innovation} of new linguistic variants, which are in some part universal, from the pressures that drive the \emph{selection} of specific variants at certain points in time.
%separate consideration of mechanisms responsible for the \emph{innovation} or introduction of new variants, as opposed to their subsequent \emph{selection} and spread within a community, can help account for both the common patterns as well as the idiosyncratic nature of language changes.
There is ample empirical evidence for the functional pressures behind the innovation of new variants that is highly asymmetric. However, the exact nature of the mechanism underlying the selection of variants out of the pool of innovations remains unclear.

The main proposal of this thesis is that the detection and amplification of \emph{trends} in language use by individuals constitutes a concrete mechanism which can account for the second step, the occasional and seemingly arbitrary selection of new linguistic conventions.
The thesis will make use of different tools, in particular computational modelling and sociolinguistic fieldwork on the perception of language changes in the individual, to argue that trend amplification is not only a viable candidate for explaining crucial aspects of the dynamics of language change, but that it naturally complements the many functional and communicative pressures which are known to influence language change. %In particular, I will argue that the asymmetries in the innovation of linguistic variants 

Rather than simply contribute another model of language change to an existing pool of explanations, this thesis is equally concerned with the bigger question of how the task of `explaining' or accounting for language change(s) is thought about in the scientific literature.
In particular, I will argue that if it is not possible to predict the occurrence of specific changes, a complete theory of language change should not just limit itself to capturing the more or less predictable aspects of language change, but also provide an account of just \emph{why language change is unpredictable}.
Wanting an account of language change to predict the unpredictability of changes might seem strange at first, but it does not imply that `anything goes': a final model that combines the asymmetric innovation of variants with the symmetric selection mechanism based on the amplification of linguistic trends shows how a theory that leaves the temporal prediction of changes at the idiosyncratic micro-level underspecified can nevertheless allow for concrete, testable predictions at the macro-level. %, thus unifying the results two approaches to language that appear superficially at odds.
%(or ideally even specify a discrete mechanism that explains)
%predict its own unpredictability.

%https://explorationsofstyle.com/2013/01/22/introductions/
% familiar stuff: conventions change

\subsubsection{Outline of the thesis}

The remainder of this thesis is structured as follows: Chapter~\ref{ch:review} provides an overview of the vast literature on language change. It covers both the nature of language change as well as the history of how language change has been studied up until the present day. % particularly quantitative framework and s-shaped curves.

Chapter~\ref{ch:modelling} is dedicated to the topic of mathematical and computational models of language change. It offers a critical perspective on the subject of modelling, before presenting two in-depth replications of existing models of language change: the Utterance Selection Model on one hand, and a Markov chain model of Bayesian Iterated Learning on the other.

Chapter~\ref{ch:momentummodel} presents a novel model of trend amplification as an as of yet understudied factor in language change. Here I augment the Utterance Selection Model with a mechanism for \emph{momentum-based selection}, a model of trend detection and amplification in the individual that was originally proposed to account for cycles in cultural change more generally. Based on multi-agent modelling I will argue that a mechanism like momentum-based selection can account for the spontaneous and sporadic nature of the actuation of language change.

Having explored the theoretical dynamics of momentum-based selection, Chapter~\ref{ch:questionnaire} sets out to contribute to ongoing quantitative research into individuals' awareness of ongoing changes in their community. Beyond anecdotal data, evidence to this end is currently limited to a few experimental studies on the use of implicit knowledge of sound changes during speech perception. This chapter provides evidence by testing individuals' explicit knowledge about the direction and progress of three related changes to low frequency syntactic variables in the Shetland dialect of Scots.
%assumptions of the model empirically. % Complementary evidence.

Chapter~\ref{ch:bigpicture} takes a step back to look at the bigger picture of the many pressures and biases that have been attested (or at least posited) to influence language change. Using a basic mathematical model from population genetics and augmenting it with a simple trend-amplification mechanism, I will show how the interaction between asymmetric, functionally-driven innovation pressures and a symmetric selection bias like momentum-based selection can account for both the micro- and macro-level dynamics of language change that we observe empirically.

Finally, Chapter~\ref{ch:conclusion} provides an summary, recapitulating the main arguments as well as pointing to potential future work, in particular relating to new research questions raised by the thesis.
