% https://explorationsofstyle.com/2013/02/20/structuring-a-thesis-introduction/

% provide a quick trip through the whole  project in the first few paragraphs, before beginning to contextualize in earnest

% The first step will be a short version of the three moves, often in as little as three paragraphs, ending with some sort of transition to the next section where the full context will be provided.

\section{Language change}

%https://explorationsofstyle.com/2013/01/22/introductions/
% familiar stuff: conventions change

\section{Problem}

% what isn't yet well understood
% language change is pointless

% this statement will echo what was said in the opening, but will have much more resonance for the reader who now has a deeper understanding of the research context

% ESTABLISH SIGNIFICANCE

\section{Response}

% what I'm actually gonna do

\section{Roadmap}

The remainder of this thesis is structured as follows: Chapter~\ref{ch:review} provides an overview of the vast literature on language change. It covers both the nature of language change as well as the history of how language change has been studied up until the present day. % particularly quantitative framework and s-shaped curves.

Chapter~\ref{ch:modelling} is dedicated to mathematical and computational models of language change. It offers a critical perspective on the subject of modelling, before presenting two in-depth replications of existing models of language change: the Utterance Selection Model on one hand, and a Markov chain model of Bayesian Iterated Learning on the other.

Chapter~\ref{ch:momentummodel} presents a novel model of an as yet understudied pressure: \emph{momentum-based selection}

Having explored the theoretical dynamics of momentum-based selection, Chapter~\ref{ch:questionnaire} sets out to test the assumptions of the model empirically. % Complementary evidence.

Chapter~\ref{ch:bigpicture} takes a step back to look at the bigger picture of the many pressures and biases that have been attested (or at least posited) to influence language change.
