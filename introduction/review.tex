\section{What is language change?}

Examples, a working definition

\section{Explaining language change}

\subsection{Early accounts}

\subsection{Language-internal accounts}

%It is not our intention to discard functional pressures completely of course, the issue we wanted to raise in the introduction section is whether functional factors should be construed as replication pressures which drive the selection of linguistic variants. Wedel's functional load result is a case in point: let's assume (I think this would be in the spirit of Wedel) that there is a general laziness/economy/blending bias that would lead all phonemes of a language to merge. Since the 26 minimal pair threshold is a probabilistic rather than a deterministic predictor, we should probably think of the influence of each individual minimal pair as contributing towards an overall functional selection bias against merging the phonemes, a 'distinction-maintaining' bias that is in direct competition with the blending one. Now between those two pressures there is a sweet spot somewhere around 26 minimal pairs where the two biases balance each other out, so that at this point the decision of whether to merge or not to merge could go either way. On the 'macro' level, where we just ask the binary yes/no question of whether the population adopts a merger or not, this might be a satisfying result, but in our opinion it fails to speak to the 'micro' level of individual behaviour: what happens at or around this critical point, is there some non-linear response across individuals where half of them exhibit a tendency to merge the phonemes and half don't? Or, assuming a more gradual effect of the bias, wouldn't we enter the regime of neutral evolution of variants within every individual where the entire population should (in a synchronised fashion) drift around between sometimes merging the phonemes, and sometimes not?

%This question of the 'sweet spot' at which universal functional pressures that have presumably always been there on the individual level without affecting the community language suddenly kick in and start an ordered directed change is what the 'actuation problem' is all about. It should be noted that the quasi-deterministic requirement for a model of language change as quoted by the reviewer that "would predict, from a description of a language state at some moment in time, the course of development which that language would undergo within a specified interval" is not what Weinreich et al. had in mind either, in the immediately following paragraph they reveal this bold goal to be nothing but a straw man, stating that "Our own view is that neither the strong nor the modest version of such theories of language change, as they proceed from current generative grammar, will have much relevance to the study of language history".

%However, in our view a full account of the pressures that drive language change *has* to make specific predictions at the level of individual behaviour, since this is what gives rise to the population-level behaviour and not the other way around. In other words, accounting for a macro-level pattern while completely underspecifying how exactly the probabilistic pressures (as observed from cross-linguistic samples like Wedel's) actually get expressed on an individual level during one particular change won't do. While macro-level accounts (like the results regarding functional load) are of course worthwhile and interesting it can be dangerous to equate them with explanations for individual events. What we are asking for here is not a deterministic account of whether a change occurs or not, we are still dealing with probabilistic models here, but models of individual behaviour that, when put into interaction with other individuals, yield predictions about population-level shifts alongside equally specific predictions about the underlying individual behaviours that give rise to the different population-level scenarios.

%In summary we never intended to be dismissive of functional factors, and we have now expanded the introduction section to discuss how functional factors play an important role within social accounts.

The levels of language change: Fedzechina graph

\subsection{Social accounts}

\subsubsection{Mechanical accounts}

% mislearning:
%1. people 'fail' to meet their speech goals
%2. kids hear these errors and decide they're the real goal
%failing is more optimal:
%Todorov and Jordan 2002: "variability which is larger in task-irrelevant directions" in optimal feed-forward synergies
%+corollary 'minimal intervention principle' (2003, p.28)

\subsubsection{Prestige accounts}

\subsection{Random drift}

% more metaphorical: Sapir
% more mechanical: Bauer etc.

\section{The actuation problem}

% define fundamental problem of accounts through quotes

\section{Language change: a quantitative framework}

\subsection{The sociolinguistic variable}

\subsection{Evolutionary approaches to language change}

Including references to general cultural evolution models

\input{introduction/scurves}

\section{Summary}

From the many different \emph{biases}, \emph{pressures} and \emph{explanations} for language change that have been put forward it becomes obvious that it is often unclear how exactly these \emph{biases} are supposed to be understood quantitatively.

Clarifying the respective roles of the different pressures is one of the goals of this thesis, and we will return to this point in Chapter~\ref{ch:bigpicture}.
