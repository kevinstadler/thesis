\section{Defining language change}

Most every talking person, whether they speak or sign, is aware of the variation found in language. While linguistic communication is a central (if not defining) part of what makes us human, if we look at specific conventions such as sounds, words or grammatical patterns, we find vast variation in the different accents, dialects and languages within a community, across a country, or across the entire globe~\citep{Evans2009}.

The same variation has not always existed, nor did it come about through abrupt divine intervention, as the tale of the tower of Babel would suggest. Rather, linguistic diversity is the result of small incremental changes to a community's linguistic system which slowly accumulate over time, first within speakers, then across generations, until eventually descendants of the same linguistic community speak a variety that would be unintelligible to speakers from a few generations earlier.

Language itself is a multi-layered phenomenon which can be approached and characterised in many different ways. While there is no single definition of what exactly language is that even a fraction of linguistic researchers would agree on, for the purposes of this thesis it will suffice to define language as the collection of learned, conventional behaviours % or system/or signs
that members of a given speech community use to communicate with each other. In its most general sense, language change then is any change to the makeup or distribution of this collection of conventions, as well as significant changes in their usage frequencies.

Linguistic conventions can be characterised at several levels of analysis, and languages undergo change at virtually all levels of description: the inventory of physical articulation gestures that form the basis of linguistic communication can change, whether by slightly altering the production of existing gestures in a gradual fashion, replacing or dropping them from the articulatory inventory, or through the introduction of new gestures and distinctions altogether. On the level of meaningful signs, i.e.~associations between articulation patterns and specific communicative functions, both sides of the association can undergo change, with a form becoming associated to a different meaning, or otherwise the same meaning coming to be expressed using a different form. Going beyond simple meaningful associations, the morphosyntactic patterns in which a language augments or combines lexical items, such as word order patterns or complex inflectional paradigms, can change as well.

While most humans are to some degree aware of the changes going on in their communities on many of these levels without the need for scientific measuring devices of any kind~\citep{Labov2001,Tagliamonte2012}, the systematic study of language changes has yielded not just many different approaches to capturing and describing changes, but also in many theories as to how the simple fact that languages undergo change at all should be~\emph{explained}.

\section{Explaining language change}

Like any empirical science, linguistics~(\emph{historical} linguistics in particular) is not just concerned with \emph{describing} natural social phenomena. The goal of scientific enquiry is to gather insights into how the phenomena at hand can be explained or accounted for, typically by making plausible claims as to their underlying mechanisms and testing falsifiable predictions made by those accounts against empirical data. Given the breadth of ways in which language changes there is a substantial amount of literature on the topic, much of which is at the same time concerned with both the documentation or \emph{description} of language changes, as well as with providing \emph{explanations} for the changes at hand.
In this section, I will attempt to provide a structured, historical overview over the main strands of thinking about how language change is explained. While much has been written about the epistemological status of explanations~(especially~\citealt{Lass1980} and \citealt[ch.3]{Newmeyer1998}), I want to draw particular attention not just to differences in the competing explanations themselves, but also to how different approaches to language change and different types of evidence lead to divergent priorities and no consensus on how~(and at what level) language change can or should be explained. 

%factors and causes~\citep{McMahon1994?}

\subsection{Early accounts}

While descriptions of languages~(in the sense of \emph{grammars}) are already attested in early antiquity, documentation relating to language \emph{changes} are far more recent~\citep{Jespersen1922}. In the absence of recordings from earlier stages of a language or a written record of any reasonable time depth, even a cunning linguist had little to work with that would have allowed for a systematic study of language change. % beyond the fact that it happens
What was required to start investigations in any meaningful way was to make the inference that \emph{dialectal} variation that can be observed for many languages was really a matter of variation that first occurred \emph{temporally} through local change, and which had then diffused across geographical space at different rates and in different directions~\citep{Chambers1998}.

While several earlier sources show that many disconnected scholars inferred this based on their study of ancient languages like Greek, Roman and Sanskrit~\citep[ch.2]{Jespersen1922},
the insight is today often credited to the orientalist Sir William~\citet{Jones1799}, mainly due to the impact his particular publication had on the quickly expanding field of Indo-European studies. The diverse varieties of the then still to be established Indo-European language family, which covered (and still cover) almost the entirety of Europe's linguistic landscape, were already densely documented at the time, forming a perfect testbed for the emergent methodology of linguistic reconstruction.

As soon as historical linguistics had moved from describing superficial similarities between lexical items or morphosyntactic constructions to positing concrete sound correspondences based on the idea of the regularity of sound changes~\citep{Paul1880}, the first theories attempting to explain the correspondences, or rather the changes presumed to have led to the later correspondences, arose. Before the idea that all languages served some universal functions and therefore had to obey particular constraints had taken hold within linguistics, the first such accounts were primarily characterised by creative psychological or even intentional reasoning reasoning behind changes such as the Germanic consonant shift:

%First faced with the discovery of regular sound changes attested in the divergence of the Indoeuropean language family, the early Indo-europeanists shared the intuitive position presented above, expressing a certain puzzlement over why a perfectly working set of conventions (in this case phonetic realisations) should be replaced for no apparent reason, often forcing them to fall back on more 

\begin{quote}
By no means do I want to claim that this change occurred without any disadvantage, in some sense sound shifting appears to be barbarism which other, quieter people abstained from, but this has to do with the immense advancements and desire of freedom of the Germans \ldots they ventured into the innermost sounds of their language~\citep[p.417, own translation]{Grimm1848}
\end{quote}

%Realising that such semi-intentional accounts, for example where phonetic changes are said to reflect underlying political ones, allow for too much leeway in
From such fanciful ad-hoc accounts which were merely limited by the creative abilities of the linguistic researcher,
explanations moved to the somewhat more accountable domain of physiological factors. Trying to account for the High German consonant shift, Grimm finds that ``experience teaches us that mountain air makes sounds sharp and rough, where the flat land makes them soft and dull''~(ibid, p.828), leading to a theory that traced the sound shift back to the increased expiration that befell the High German tribes as they settled in mountaineous regions. As more and more descriptions and evidence for similar changes across different languages began to accumulate, linguists started appealing to even more general, mechanistic explanations.
With the field still very much focussed on sound changes but before the advent of reliable tools for acoustic analysis and therefore reliant on what could be gathered from written records, \citet{Jespersen1922} contemplated that

\begin{quote}
at one moment, for some reason or other, in a particular mood, in order to lend authority or distinction to our words, we may happen to lower the jaw a little more, or to thrust the tongue a little more forward than usual, or inversely, under the influence of fatigue or laziness, or to sneer at someone else, or because we have a cigar or potato in our mouth, the movements of the jaw or of the tongue may fall short of what they usually are. We have all the while a sort of conception of an average pronunciation, of a normal degree of opening or of protrusion, which we aim at, but it is nothing very fixed, and the only measure at our disposal is that we are or are not understood.~(p.166)
%Everyone thinks that he talks today just as he did yesterday, and, of course, he does so in nearly every point. But no one knows if he pronounces his mother-tongue in every respect in the same manner as he did twenty years ago. May we not suppose that what happens with faces happens here also? One lives with a friend day in and day out, and he appears to be just what he was years ago, but someone who returns home after a long absence is at once struck by the changes which have gradually accumulated in the interval.''
\end{quote}

While the particular depiction of what might today be thought of as underspecification or coarticulation effects is rather baroque, the quote exemplifies a general trend in the field to abandon particular explanations for particular changes in favour of more general \emph{linguistic} pressures that apply across languages~\citep[see also][ch.21]{Bloomfield1933}. This period gave rise to what is probably the biggest framework for explaining changes whose legacy extends well into the present day, in what can be referred to as \emph{language-internal} accounts of change.

\subsection{Language-internal accounts}

The defining feature of language-internal accounts of change is that they explain the occurrence of changes based on an \emph{asymmetry} between the language states before and after the change that is \emph{internal} to the linguistic system and its acquisition. % could be general or language-specific
While there is no strict definition of exactly what makes an account language-internal, one finds several attempts to delineate the approach in the literature, e.g.~that ``specific changes are thought to be internally caused when there is no evidence for external causation, i.e.~for language contact''~\citep[p.366]{Luraghi2010}.
While the matter of distinguishing `internal' from `external' causation is far from simple when investigated in detail, we find similar definitions across the literature, e.g.~\citeauthor{Hickey2012}'s handbook definition of internal change as ``Any change which can be traced to structural considerations in a language and which is independent of sociolinguistic factors can be classified as internally-motivated.''~\citeyearpar[p.388]{Hickey2012}.
%While there are authors who argue that ``no empirical study so far carried out has actually demonstrated that sound change can arise spontaneously within a variety''~\citep[p.24]{Milroy1999}, i.e.~that there is no such thing as true internal change.
These accounts generally differ from earlier accounts by not attempting to ground language changes in non-linguistic features of their specific speaker groups, nor in other external inter-individual or political dynamics that form the basis of \emph{social accounts} to be discussed below.

In terms of the scientific treatment of language change, language-internal accounts signified a transition from the study of \emph{particular} changes to that of general, \emph{universal} principles of change. This move was primarily supported by the methodology of and results obtained from the systematic study of sound changes. On one hand, the success of the new dogma of the \emph{exceptionlessness} of sound changes indicated that language change was not completely erratic, but followed rule-like patterns. In terms of scientific thinking about language change, this suggested that it might be possible to discover the underlying mechanical principles of change, which would turn linguistics from a historical social science to one of more or less deterministic explanation like in the physical sciences~\citep{Sapir1929}.

The impression that it should be possible to move beyond the mere post-hoc description of changes to their prediction was not just due to the theoretical framework behind linguistic reconstruction, but also due to the empirical results obtained from the method. The systematic documentation of sound changes across many languages and language families revealed \emph{unidirectional} developments, the catalogisation of which forms part and parcel of any historical linguistics textbook until today~\citep[see e.g.][]{Campbell2013}. A prime example are changes in the realisation of phonemes which were copiously attested in one direction, but hardly the other, such as the frequent frication of voiceless stops, as opposed to the more unusual stopping of fricatives. While the identification of these universal \emph{asymmetries} was perceived as an improvement over the earlier ad-hoc accounts of individual changes, the lack of a well-defined framework for describing languages meant that researchers were left with a lot of leeway in how to classify individual changes, so as to `explain' them as an instance of some generally observed tendency.

In a recursive manner, the observation of general trends or constraints on the \emph{direction} of changes gave rise to the postulation of general principles which were then invoked to explain the very changes they were derived from~\citep{Lass1980,Haspelmath2006}. While this approach certainly helped to identify paths that sound changes were unlikely %~\citep[although only arguably impossible][]{?}
to go down, it allowed the occurrence of any individual change to be `explained' by picking from a broad inventory of possible or `likely' changes that even included ones going in opposite directions, for example the processes of phonetic assimiliation and dissimilation, both of which would be considered `natural' developments.
%While some light was shed on the types of changes, the question of why exactly why what changes occurred was forgotten.

While the comparatively `simple' matter of sound change was often maintained to be due to the gradual \emph{accumulation of errors} in pronounciation already hinted at above~\citep[see e.g.][for a textbook account of the general principle]{Hockett1958}, the rising interest in syntax during the advent of generativism also resulted in more precise theories of how syntactic changes unfold over time. While syntax constituted an arguably more complex domain of inquiry than phonetics or phonology, the basic idea of change as a matter of \emph{mislearning}, i.e.~failing to acquire the `correct' target convention from ones linguistic environment, was transferred from sound to morphosyntactic changes~\citep{Salmons2013}.
The main difference between the fields was that, with its underlying assumption of an idealised language capacity and particular focus on the \emph{language acquisition device}, generativist explanations sought mislearning mainly in children's acquisition of a language's syntactic rules. Given the purpose-specific nature of said acquisition device, child-based language change was largely explained to be due to changes in the `primary linguistic data' from which the device is meant to derive a language's underlying grammar.
%child-based language change~\citep[p.109]{Croft2006}

The relative ease with which categorical morphosyntactic changes could be described also gave rise to a large body of quantitative work related to grammatical parameter setting~\citep{Lightfoot1991} with concrete claims about its reliance on specific learning cues in linguistic data~\citep{Gibson1994}.
In contrast to earlier work on sound changes which still struggled with the lack of precise quantitative description of its subject matter, the exact description of specific cases of syntactic change as well as concrete postulated thresholds for the amount of external data required to correctly acquire their underlying grammars led to specific models that were meant to recapitulate specific historical changes, e.g.~\citep{Yang2002}.

It is interesting to note how the `success' or correctness of an \emph{explanation} of language change is evaluated in language-internal accounts of particular changes, as opposed to those oriented towards the identification of general pressures mentioned above.
Given that the causal triggering of the change is framed to be internal to the language, rather than due to some arbitrary external, historical factor, the particular language changes under investigation are more or less explicitly framed as inevitable, rather than accidental.
Whether the asymmetry is based on universal articulation biases or on changes to the learning data~(in particular the frequency of use of morphosyntactic constructions or lexical items, which are for some reason also often regarded as inevitable language-internal facts), an account~(and especially a quantitative model) of a specific change would be considered `wrong' if it failed to predict the occurrence of the change.
%~(more or less deterministically).

%As in interesting corollary, models of those 
The fact that there are many models that `successfully' predict the occurence of particular changes anticipates a criticism of language-internal accounts that will be elaborated on later namely that it is always possible to find some asymmetry between the competing language states that biases acquisition towards the later stage of the language.
Although language-internal accounts appear to be in principle falsifiable, the fact that the methodology is geared towards the identification of successful particular explanations is not merely a theoretical concern~\citep{Lass1980}: the vast majority of published accounts aimed at identifying specific language-internal causes or triggers for a change do so `successfully' in the sense of correctly predicting particular historical changes~\citep[e.g.][among many others]{Troutman2008,Ritt2012,VanTrijp2013}. While there are also publications which consider unsuccessful, i.e.~historically unattested (competitor) models~\citep[e.g.][]{Sonderegger2010} or those which also consider a model successful even if it underdetermines the direction of changes to some degree~\citep[e.g.][]{Lau2016} these are the exception rather than the rule.
%Similar to concerns regarding confirmation bias in experimental work, one must assume that for every case where a certain factor is shown to correctly predict the occurence of a change, there is a much higher number of unreported cases where efficiency does not seem to affect change at all.
%with an important twist: it's historical, it can't be repeated.
The construction of quantitative language-internal models of specific changes provides a precise account of the supposed underlying mechanisms responsible for those particular changes, but it offers no generalisability beyond that one historical event. %While the study of sound change had already generalised its findings from particular change events to the general patterns that changes across the world appeared to follow, the study of lexical and morphosyntactic change followed suit.

\index{unidirectionality}\index{grammaticalisation}
Alongside the generative work focussed on explaining grammatical change as internal to syntax, cross-linguistic typological approaches to language change gave rise to an explanatory framework similar to that used in sound change. In parallel to the directional tendencies that had been revealed to guide sound changes, the study of \emph{grammaticalisation} suggested a similar pattern where closed-group, bound grammatical morphemes can be traced back to formerly independent lexical items with specific referential content~\citep{Hopper1993}.
%Jespersen's \emph{negation cycle} 
In contrast, barely any examples of the development of (bound)~grammatical morphemes becoming standalone lexical items can be found across the world's languages~(English~`ish' being one of the rare exceptions).
This apparent \emph{irreversibility} of grammaticalisation~\citep{Haspelmath2004directionality} lent support to a new set of universal asymmetries due to the unidirectionality of both semantic and phonetic erosion that is thought to guide morphosyntactic changes. The eternal cycle of wear of morphosyntactic material and consequent need for replenishment suggested by grammaticalisation theory also fit neatly with the universal circular nature of language changes suggested by more descriptive level work by typologists~\citep{Cowgill1963,Hodge1970}.

While the various approaches to language change listed here come from very different approaches and cover very different aspects of language change, 
the emerging unifying principle of language-internal explanations is the conceptually similar approach of accounting for changes in a \emph{mechanistic} way due to them being based on some \emph{inherent asymmetry}, either between competing linguistic variants, or between the language state before and after the change. In all cases the acquisition or use of the original language state is assumed to be prone to actually acquire or perform the target state. Exactly how this asymmetry is framed depends mostly on the particular theoretical approach taken, ranging from talk about \emph{pressures}~\citep[e.g.][]{Thomsen2006}, a nod towards deterministic explanation in the natural sciences, over \emph{motivations}~\citep{MacWhinney2014} and \emph{preferences}~\citep{Fedzechkina2016} in psychologically-grounded work, all the way to \emph{biases}, the latter term used both generally as well as in reference to learning and inference frameworks in particular.
Putting forward such a mechanistic language-internal account for a particular historical change often implies that that change was in some sense inevitable, an approach to explaining -- in the sense of deterministically predicting -- the occurrence of specific changes that is still most widespread today in the domain of morphosyntactic changes~\citep{Lightfoot2010}.

In the study of sound change, equally, mechanistic error-based accounts continue to be popular~\citep[see e.g.][who make specific reference to the importance of \emph{asymmetries} between phonetic variants to explain change]{Garrett2013}, %~\citep[see e.g.]{Beddor2009}
now encompassing not just phonetic changes but also exhibiting an increasing interest in how \emph{phonologisation} of formerly non-segmental features gives rise to novel phonemic contrasts~\citep[e.g.][]{Kirby2013}.

While providing insight into the ways and trajectories in which languages change, language-internal accounts have long been criticised for underspecifying \emph{when} changes occur, a conceptual and scientific issue that has come to be known to historical linguists as the \emph{actuation problem}. But before turning to the actuation problem, I will turn to language-internal accounts' bigger and bolder modern sibling, which is based on the idea of characterising language as a \emph{complex adaptive system}.

%The increasing availability of individual changes and their particular explanations and patterns across them led

%``How can speakers be masters of pronunciation change in one area but helpless victims in other cases?''~\citep[p.192]{Ohala1989}

\subsection{Competing motivations and language as a complex adaptive system}
\label{sec:lcas}\index{competing motivations}\index{pressures!competing pressures}\index{complex adaptive system}

The turn towards novel research methods and quest for generalisations across languages has seen the rise of a new research moniker that frames \emph{language as a complex adaptive system}~(\emph{LCAS} for short). This general term provides a catchy name for a wide range of branches of linguistic research which do not share a common methodology or evaluation criteria. Rather, it covers approaches within specific fields that acknowledge the important role of competing factors from a variety of sources~(cognitive, physical or social) while stressing the fact that languages are continuously adapting based on the needs and requirements of everyday language use in interaction~\citep{LCAS2009}. Just as the transition from early ad-hoc accounts to more theoretically motivated language-internal explanations was gradual, there is also no clear divide between the LCAS approach and language-internal accounts, many of which also developed an interest in the intricate interactions of competing motivations in language change~\citep{Berg1998,Thomsen2006,MacWhinney2014}.
%In the exposition of the different accounts of and explanations for language change so far, the pressures were presented as directly set up against each other. But, partly in order to tackle the problem of actuation, scholars have already considered the interaction between different, \emph{competing} pressures.
Crucially, when it comes to the study of language \emph{change} as a natural outcome of language's continuous adaptation, the approach shares with language-internal accounts that its focus is explicitly on the \emph{universal} patterns and direction of change~\citep[p.4-5]{LCAS2009}.

Rather than representing a tight methodological framework, LCAS also stands for an openness towards novel and experimental tools to further our understanding of language use and change, in particular approaches that go beyond explanations that could be based on mere `arm chair reasoning'. In order to get to grip with the kinds of dynamics implied by the word `complex', mathematical and particularly computational modelling became a popular tool to investigate emerging and evolving communication systems. Since the entirety of Chapter~\ref{ch:modelling} is dedicated to both the topic of modelling in general as well as replications of some specific models, I will for now focus on the results obtained from empirical research that is thought to speak to the adaptive nature of linguistic changes.

\subsubsection{Inferring universal pressures from cross-linguistic data}

While diachronically-minded linguists have long been grouping and categorising similar changes to uncover general trends in the paths taken by language change, typologists have arrived at a similar point but from the opposite direction. Based on comparative work on the synchronic distribution of linguistic features such as~\citet{Greenberg1963}, the preponderance of individual features~(or co-occurrence of features) was taken as evidence for some sort of adaptive advantage to those features or feature constellations~\citep{Haspelmath2008}.

In line with LCAS principles, the adaptive pressures are assumed to be associated with language \emph{use}, which would over time influence the development of languages and consequent distribution of grammatical properties across the world.
One contrast between such typological approaches and traditional language-internal accounts is the assumed \emph{specificity} of the pressures that are thought to steer language change. While language-internal accounts often refer to linguistic concepts and mechanisms that are theory- or language-specific, typological work typically works with much more \emph{descriptive} measures that lend themselves to cross-linguistic comparison~\citep{Haspelmath2010}.

Rather than embodying a simple, qualitative asymmetry between specific competing variants, typological pressures are typically described in \emph{quantitative} terms. 
For example, \citet{Zipf1935,Zipf1949} popularised~(and sought to explain) the finding that, across many languages, the frequency of a word is inversely proportional to its rank in the frequency table of all words of that language.
Having such a distribution is indicative of some sort of optimisation since, if more frequently used words are shorter, the average length of signalling decreases overall. While it has since been shown experimentally that individuals will preferentially produce abbreviated forms when it is more effective in information-theoretic terms~\citep{Mahowald2013,Kanwal2017}, a bias for the selective propagation of changes towards such advantageous forms through larger speech communities has not been demonstrated yet.

An early example of a general, quantative pressure whose effect was postulated to affect change directly is that of a phoneme's \emph{functional load}~\citep{Martinet1955}. The phoneme, typically defined to be the largest non-meaning-bearing unit of linguistic analysis, gains its function as a linguistic category by providing meaning distinctions at the higher level of \emph{combinatorially created} meaningful signals, or \emph{morphemes}. In this way, a phoneme owes its existence to linguistic items at a higher level of organisation which allow it to stand in contrast to other phonemes, typically in the form of lexical minimal pairs. In the absence of any such minimal pairs, upholding a phonemic contrast is inefficient in the sense that it would move the lexicon away from being a productive system built on re-using a limited set of signals towards an unstructured inventory of individual holistic lexical items, with as many distinct articulation gestures in the language as there are morphemes~\citep{Spike2017}.
In practice, however, it is possible to find several minimal pairs for most pairs of phonemes in any given human language. Combining this observation with studies of the reorganisation of sound systems, especially the \emph{merger} or \emph{contrast loss} of formerly distinct phonemes, led \citet{Martinet1955} to propose that a phoneme's \emph{functional load}, i.e.~the amount of contrastive function it performs at the level of the lexicon, would have an influence on whether or not it would be maintained as a contrastive element relative to other phonemes. The concept fits well under the modern LCAS umbrella, seeing as it is describes a delicate trade-off between competing pressures of reducing the number of distinct signals maintained by a language against the pressure of keeping distinct phonemes to allow combinatorial signalling to occur at all. While the direct link between number of minimal pairs and contrast maintenance has since seen confirmation based on population-level diachronic data~\citep{Wedel2013short}, the question of whether the observed pattern satisfies some externally-optimal equilibrium or whether it is simply based on the peculiarities of individual human phonemic acquisition still needs to be addressed~\citep{Spike2017}.

More recently, theories speaking to the efficient organisation of language have also taken a firm hold in thinking about linguistic structures and inventories at other levels.
Based on the functional typology work referred to above, the availability of large-scale syntactic corpora has allowed quantitative investigations into the syntactic organisation of languages based on actual usage data, rather than abstract principles of syntactic description. In particular, studies of the average syntactic dependency length of data for various languages, which is taken as a measure of an utterance's processing complexity, have revealed that many of the correlations between constituency orders for different syntactic constructions can be explained based on simple preferences or biases towards efficient processing~\citep{Futrell2015} or language `utility'~\citep{Jaeger2010}.

While the pressures investigated in all these approaches are typically motivated by general cognitive principles at the level of the individual, most of the evidence for their existence is gathered at the very macro-level, often from cross-linguistic data, where those biases are thought to have expressed themselves over time. The question of whether (and how) individual biases become manifest in languages on the inter-individual population level is itself a non-trivial question~\citep{Kirby1999}. The necessity to go beyond macro-level descriptive data to address this question has long been recognised, and given rise to an experimental framework in the LCAS spirit, known as \emph{iterated learning}.

%All these accounts assume an internally stable, consistently acquired language whose stability is broken by external perturbation. But in the absence of any external triggers for the reorganisation of a linguistic system, what are the odds of `eternal optimisation' of a language~\citep{Boersma2003}?

%concrete empirical work is particularly focussed on \emph{universal} pressures, or at least on pressures that can to some extent be generalised cross-linguistically.
%In the same way that sound changes, given that it is motivated by macro-level typological work, such as the statistical universals documented by~\citet{Greenberg1963}.

\subsubsection{Universal pressures and the Iterated Learning Model}
\index{iterated learning}
\label{sec:ilm}

As the name suggests, the Iterated Learning Model~(ILM for short) actually started off as a computational model, primarily to investigate the question of innate biases related to the compositionality of human languages~\citep{Kirby2000,Brighton2002,Kirby2002}. Counter the idea of a strongly innate language capacity that had been dominating most thinking about morphosyntax since Chomsky's `poverty of the stimulus' argument, these first computational efforts at modelling the emergence of languages %beyond the simple lexica of the naming game
suggested that weak, domain-general biases in the individual were enough to yield strong linguistic universals at the population level when learning and production were \emph{iterated}, i.e.~repeated over generations of learners~\citep{Kirby2004}.

While the iterated learning paradigm has undergone gradual transformation in its scope and goals since these early models~(to the point of encompassing any and all experiments or models of repeated communication according to some definitions, see \citealt{Scott-Phillips2010}, Box~3), one aspect that still features strongly in most of the model's incarnations is the \emph{bottleneck}.
The original conception of the role of the bottleneck was that limiting the amount of data learners received would force them to generalise from the limited data they received, causing linguistic signalling systems to become more \emph{structured} through repeated production and learning. The effectiveness of the bottleneck was first shown in computational simulations to lead to the emergence of compositionality~\citep{Kirby2000} as well as recursion~\citep{Kirby2002}, and its role in triggering the emergence of compositionality was later confirmed experimentally with human participants in controlled laboratory experiments~\citep[e.g.][]{Kirby2008,Cornish2009,Smith2010}.
The same effect was also taken to show that, in contrast to the assumptions of strongly nativist theories of language, subtle biases in the individual can give rise to strong universal patterns when the learning and production of linguistic systems is iterated~\citep{Kirby1999}.

This latter idea has undergone implicit generalisation to other biases not necessarily directly related to language \emph{structure}, at the latest with~\citet{Kirby2004} who reframed the acquisition by individual learners as a Bayesian inference process. In the Bayesian framework, the task of a learner is to infer which of the possible grammars~$h$ out of a given grammar hypothesis space underlie the linguistic data~$d$ that they observe in learning. To do so rationally, the learner computes the \emph{posterior probability} of all possible languages~(hypotheses) given the learning data,\index{Bayesian inference}

\begin{equation}\label{eq:bayes}
p(h|d) = \frac{p(d|h) \cdot p(h)}{p(d)}\;.
\end{equation}

%In the symbolic, cognitivist framework embodied by the Bayesian approach that dominates much work in cognitive science today, 
The Bayesian learning framework, which has come to dominate much thinking about symbolic cognitive science in the past few decades, offered a natural grounding for iterated learning. It suggested a direct correspondence between the aforementioned `biases', previously defined as ``everything that the learner brings to the task independent of the data''~\citep[p.590]{Kirby2004}, to the \emph{prior} probabilities~$p(h)$ of the competing hypotheses in Bayesian inference. In this sense, biases could be made explicit simply by skewing the prior probability distribution over all hypotheses away from a uniform distribution corresponding to no a priori preference for any particular language. (It should be noted that, just in terms of \emph{asymmetries} between competing languages, a bias towards specific hypotheses~$h$ can also be introduced by adjusting the different hypotheses' data production structures~$p(d|h)$.)

Combining several such Bayesian learners in an iterated learning \emph{chain} would mean providing the first learner with some input data~$d$ from which they first try to infer the underlying hypothesis by computing their posterior distribution~$p(h|d)$. The learner would then become a `teacher' by producing some of their own data~$d$, either based directly on their posterior, or from a specific hypothesis chosen based on the posterior, passing this data on to the next learner and so forth.

Based on interpreting such chains of Bayesian learners as a stochastic Markov model~(a modelling approach that will be discussed in-depth in Section~\ref{sec:markovmodel}), it is possible to analytically derive the expected probability of observing any of the different hypotheses at a given point in time.
\citet{Griffiths2007} provided a mathematical proof that a chain's probability of exhibiting a hypothesis in this Bayesian inference-based version of the Iterated Learning Model~(BILM) should be identical to the prior distribution~$p(h)$, a finding which seems to run counter the empirical evidence which showed an \emph{amplification} of prior biases. \citet{Kirby2007} reconciled the two results by showing that, when learners adopt more deterministic strategies of selecting a hypothesis from the posterior such as choosing the hypothesis with the highest posterior probability, this will lead prior biases to become exaggerated as learning is iterated, especially when there are only small amounts of learning data available. While the question of which hypothesis selection strategy should be used can in theory be derived from experimental results for specific learning tasks, there is still no conclusive evidence on the matter~\citep{Reali2009,Ferdinand2015}.

% If learners adopt a more deterministic strategy — moving towards simply selecting the hypothesis with highest posterior probability — then iterated learning converges to a distribution that exaggerates the prior: hypotheses with high prior probabilities appear even more often, while those with low prior probabilities become even less likely. The exact distribution depends on how much data is seen by each learner, with the prior having a stronger effect when only small amounts of data are available. This analysis thus helps to explain the circumstances under which cultural transmission can magnify learning biases (allowing weak biases to be a potential explanation for strong linguistic patterns)

%Nevertheless, the iterated learning paradigm has been used extensively to identify and argue for universal pressures that go beyond the mere emergence of structure in linguistic systems.
%the same interaction between ``a tendency towards efficiency and a competing tendency toward redundancy'' was already proposed by~\citet[p.155]{Greenberg1954}~(printed in~\citealt{Osgood1954}), in a period that saw frequent export of concepts from the new field of information theory to other, more traditional areas of research~\citep{Shannon1956}.

% production-perception loop \citep{Winter2016}

The iterated learning paradigm offers an individual-level complement to the macro-level analyses based on cross-linguistic data discussed above, and suggests a way in which the two levels of individual biases and typological distributions can be linked directly~\citep{Kirby1999,Kirby2008}. However, especially due to their origin in studies of the emergence of linguistic systems from scratch, iterated learning models are based on small population sizes, typically pairs of individuals arranged in chains or dyads, who have to learn from very limited and highly variable data in which learning biases are thought to express themselves most strongly~\citep{Fedzechkina2014}. As such, there are still two missing intermediate links between the two levels, namely the synchronic diffusion of traits through larger populations that already possess an established communication system, as well as their consequent spread through diachrony~(ibid, p.26). While the role of individual biases on the latter has received support through studies such as~\citet{Wedel2013short}, what is known about the spread of new linguistic variants at the micro-level of the population appears to be at odds with the idea of a straightforward expression of individual biases across all levels, as I will argue below.

Before I turn to a critique of adaptationist accounts based on universal pressures, however, it is also necessary to acknowledge another branch of research situated within the LCAS paradigm dedicated to uncovering the effects of general but not quite universal pressures, and their impact on language and language change.

\subsubsection{Adaptation to non-linguistic pressures}

Due to its general nature, the LCAS moniker does not just cover the study of \emph{universal} pressures associated with the iterated learning model, but also the investigation of non-universal (yet still generalisable) pressures and how they affect otherwise unrelated languages and language families in similar ways.
With the advent of large-scale databases on languages such as the World Atlas of Language Structures~\citep[WALS][]{WALS2013}, statistical studies have identified correlational relationships between linguistic structures and non-universal external, non-linguistic factors -- whether biological, geographical or social~\citep{Ladd2015}.

Among the first and still thoroughly established results of this kind is \citeauthor{Dediu2007}'s finding that the distribution of tonal languages in the world shows an unexpectedly large degree of overlap with the geographical distribution of two genes related to brain size~\citeyearpar{Dediu2007}. The authors posit that there might be a causal link between the two, with the existence of a certain biological substrate biasing language acquisition or processing towards certain language structures, in this case ones incorporating tone, a bias that would become exacerbated through iterated cultural transmission.

Many other suggestive correlational patterns have since joined this result, with the purported causality going either from environmental and social factors triggering changes or adaptations in linguistic structures~\citep{Hay2007,Lupyan2010}, or conversely from linguistic features to other social behaviours~\citep{Chen2013}. While the lack of concrete evidence for such effects beyond the stating of statistical macro-level correlations has received much criticism, it has also been argued to be an opportunity for both deriving and testing concrete mechanisms by which social and other external factors influence linguistic structure~\citep{Roberts2012,Nettle2012,Roberts2013correlation}. Specific claims as to how social scale can affect structural complexity indirectly, for example by changing the degree of input variability that individual learners receive~\citep{Wray2007}, have only just begun to be tested experimentally, lending no consistent support to any of the hypotheses that have been proposed so far~\citep{Atkinson2016}.
% "Languages evolve, adapting to pressures arising from their learning and use. As these pressures may be different in different sociocultural environments, non-linguistic factors relating to the group structure of the people who speak a language may influence features of the language itself."

% space of all possible languages -> `pointless' changes?

%LCAS suggests complex grammar spaces beyond simple switching back and forth between two conventions such as word order patterns.
%While there is evidence of some circular changes in language, e.g.~in the process of grammaticalisation, in the absence of a dedicated theory of why a pressure fails to kick in to drive the circle forwards for a certain period of time

%by external actuation such as \citet{Ritt2004} TODO or Baumann paper??
%in a similar way to \citet{Yang2002} who explains the rise of periphrastic `do' in Early Modern English through the influence of non-native speaker productions on the learners' primary linguistic data.
%While such accounts offer particular explanations for particular historical changes, they don't speak towards whether a \emph{general} theory of language change is possible.

With this newest branch of research on factors that are believed to be driving language change we come to the end of one long, more or less continuous arc of approaches that seek the source of language change in general, mechanistic pressures that are thought to be acting on languages at all times. Winding back to the early days of language-internal accounts I will discuss a criticism that has long been hauled at general explanations of language change and that is equally applicable to their modern, adaptively-minded incarnations: the actuation problem.

\subsection{The actuation problem \& sporadic language change}\index{actuation!actuation problem|textbf}
\label{sec:actuationproblem}

Returning back to an earlier period of linguistic study, the advent of structuralism that is typically associated with the publication of Saussure's \emph{Cours de linguistique générale}~\citeyearpar{Saussure1916} reset the focus from establishing historical relationships to the description of individual languages for their own sake. %This in turn raised puzzlement over why a working language should change.
Where in 19th century linguistics research diachronic concerns still played a central role, the systematic identification of the various functions served by \emph{synchronic} language structures under the new paradigm cast additional doubts on why language should change at all:

\begin{quote}
the more linguists became impressed with the existence of structure of language, and the more they bolstered this observation with deductive arguments about the functional advantages of structure, the more mysterious became the transition of a language from state to state. After all, if a language has to be structured in order to function efficiently, how do people continue to talk while the language changes, that is, while it passes through periods of lessened systematicity?~\citep[p.100]{Weinreich1968}
\end{quote}

The criticism embodied by the actuation problem is mainly directed at language-internal accounts, or any other account that attempts to explain changes through inherent asymmetries between competing variants.
While the move from the (over-)explanation of particular language changes to their framing in the light of \emph{universal} pressures that allow generalisation meant progress on the question of \emph{why} or \emph{in which direction} languages change, the approach was said to create ``the opposite problem -- of explaining why language fails to change.''~\citep[p.112]{Weinreich1968}:

\begin{quote}
Why do changes in a structural feature take place in a particular language at a given time, but not in other languages with the same feature, or in the same language at other times? This \emph{actuation problem} can be regarded as the very heart of the matter.~(p.102)
\end{quote}

Although there is no precise definition of `actuation' in \citeauthor{Weinreich1968}'s seminal paper and no consistent usage of the term in the literature has emerged since, it is generally understood to mark \emph{the point in time} when an incoming linguistic variant starts to see a consistent increase in its usage at the expense of established competitor variants, i.e.~the detectable \emph{onset} of the diffusion of a change\footnote{The term `actualization' \citep[with a specific meaning in grammaticalisation theory, see][p.24]{Traugott2011} is sometimes used interchangably~\citep{Andersen2008,Kiparsky2014}.}.

%but it may also refer to the \emph{mechanism or process} behind this increase.
% TODO rephrase What is at that point still only a change-in-the-making can in theory still be \emph{interrupted} by seeing its spread curtailed and reversed back to nonexistence, but one typically speaks of a change as \emph{actuating} when it starts to exhibit increase in usage that is consistently spread across speakers or speaker groups.

While language-internal accounts went a long way in terms of identifying general, inherent instabilities in or asymmetries between two language states, they typically fail to explain the inactivity of a pressure until its point of actuation, i.e.~why a change occurs \emph{exactly when it does}. %means it's again hardly falsifiable on an \emph{individual} basis.
This concern is aggravated by the fact that the myriad pressures that have been proposed to act on language provide a large amount of explanatory freedom by allowing researchers to choose from a whole pool of `explanations' more or less at will, a fact that is also acknowledged by some adherents of the paradigm which make explicit reference to the fact that the approach makes it difficult to connect specific pressures to particular changes:

\begin{quote}
%In a nutshell, then, to ask the question: "Is this rule or constraints (or whatever) motivated" is to ask the wrong question.
No rule or constraint has a motivation in and of itself, but only within the total system in which it occurs, and crucially, in the history of that system. \ldots the interplay of explanatory factors is vastly too complex to allow individual motivations to be attached to individual grammatical elements.~\citep[p.313]{Newmeyer2014}
\end{quote}

Formulated almost 50~years ago, the fact that the actuation problem is still frequently referred to even in contemporary literature on language change indicates that it has never been resolved completely. Importantly, the argument applies equally to accounts referring to domain-general or even external adaptive pressures that are the interest of LCAS approaches, which were also shown to be geared towards very general explanations of change. Particularly in iterated learning models we see a strong focus on universal pressures giving rise to \emph{universal features} such as compositionality, recursion, expressivity etc.~\citep{Brighton2002,Kirby2002,Cornish2009,Smith2013}. %That the criticism of the actuation problem applies equally to LCAS accounts which, as I argued above, are typically invoked to explain the \emph{universal}, but not the particular. Given the wide range of pressures invoked by LCAS approaches I want to cover a distinction between two types of language changes based on the source of asymmetry between them, a distinction that is not typically made since it cuts across the usual categorisation of changes depending on their domain of linguistic description, i.e.~sound, lexical, semantic and morphosyntactic changes.
%There are changes which do not change the nature of contrasts, therefore maintaining the system's expressivity. In the former case, the linguistic systems after the change are just as expressive or rich in contrasts as before, whereas in the latter case an argument can be made that the language has seen some overall improvement beyond the function of the particular convention that underwent change.
%This distinction between simple convention changes and system changes is important since they are typically argued to be explained by different biases. On the level of description that simply identifies differences between competing variants or systems, however, both types of changes are simply due to an inherent asymmetry between them, and the problem or criticism of actuation applies equally to both.
% But which will be relevant to counter different claims about the source of asymmetry in LCAS.
When there is room for variable expression of biases, the evolution towards any particular solution is triggered from the outside by experimental manipulation~\citep{Winters2015}, offering no internal account of why any particular bias might suddenly come to outweigh another one, thus triggering a change in an established, working language.
% Externally-triggered social effects are just as puzzling, since they only explain general tendencies but with many counterexamples

While some researchers allude to the `solving' of the actuation problem for particular changes they investigate\citep[e.g.][]{Baker2011,Stevens2013}, especially work dedicated to the study of the general principles of linguistic change acknowledge the issue of actuation much less. This might not be surprising, given that the full, general thrust of the argument often seems to speak against the possibility of any explanation of language change that goes beyond the account of one individual change and its particular circumstances of actuation.
The macro- vs. micro-level approaches to accounting for language change have led to a separation of concerns, where researchers are either dedicated to explaining \emph{which} changes actuate, generally, or otherwise to the question of when \emph{specific} changes actuate in particular. While I will propose a solution to address both those concerns at the same time in Section~\ref{sec:adaptationism}, we first need to discuss another strand of explanations of language change that arose from fieldwork on the dynamics of the spread and diffusion of linguistic changes through speech communities.

\subsection{Social accounts}\index{social accounts}
\label{sec:socialaccounts}

In the same period of the second half of the 20th century in which typological and generative work on morphosyntactic change indicated universal trends and directions of changes, sociolinguistic research %in the empiricist tradition
painted a wholly different picture of the spread of linguistic changes in parallel~(or rather orthogonal) to language-internal accounts.
Studies in what is sometimes also referred to as the \emph{variationist} tradition~\citep{Tagliamonte2015} have shown that there is a vast amount of variation in language use not just between but even \emph{within} individuals of one and the same speech community.
What these investigations across different languages and cultures have confirmed is that linguistic variation is not distributed randomly but in what \citet{Weinreich1968} call \emph{structured heterogeneity}, often reflecting the underlying social or political structure of the community with the usage of specific linguistic variants stratified according to social characteristics of the speakers such as their age, ethnicity, or socio-economic status~\citep{Foulkes2006,Tagliamonte2012}.
The importance of linguistic variation for the study of language change is again based on extrapolating the observed synchronic variation into time, by recognising that linguistic innovations do not spread evenly across geographical space, or even within speaker groups.

As was the case with language-internal accounts, it is difficult to find or give a precise definition of what exactly makes a `social account'. In contrast to language-internal accounts, social accounts are characterised by seeking the `reason' for the adoption or diffusion of a novel linguistic variant not in (features~of) the variant itself but in some external, social factors.
While social accounts of language change are primarily informed by empirical research of the \emph{micro-level} of linguistic variation and individual changes, sociolinguistic thinking goes beyond individual case studies.
Similar to the historical development of language-internal accounts, sociolinguistic research %over the past 60~years or so
has moved from the documentation and ad-hoc explanation of the social stratification of particular changes to the postulation of generalisable pressures that are presumed to hold across languages and cultures.

%Beyond intra-linguistic or functional effects, research have been looking for the effects that \emph{social} influences have on language change.
For the present purpose I will characterise and discuss two types of accounts that encompass most of the thinking on social influences on language change, the first one referring to \emph{mechanics} of social interactions, the other on the more fuzzy topic of the social \emph{meaning} of linguistic variation, and how it affects the spread of language change.

\subsubsection{Social network accounts}\index{interactor selection}
\label{sec:interactorselection}

While it is again hard to find explicit definitions for different types of social accounts, there exists a relatively clear subcategory of social explanations that stresses the important effects of social interactions. 
Inspired by the quantitative study of social networks and work such as the idea of ``weak ties''~\citep{Granovetter1973}, researchers have attempted to reduce the diffusion of language changes to the underlying distribution of social interactions in a speech community, an idea clearly formulated very early on by~\citet[p.476]{Bloomfield1933}:

\begin{quote}
The inhabitants of a settlement [\ldots] talk much more to each other than to persons who live elsewhere. When any innovation in the way of speaking spreads over a district, the limit of this spread is sure to be along some lines of weakness in the network of oral communication.
\end{quote}

The quantitative, rule-like accountability of this ``principle of density'' of communication can again be seen to hark back to the ideal of mechanistic causal explanation, this time due to external, interactional factors~\citep[p.19]{Labov2001}. While the idea of reducing linguistic diversification to discontinuities in the social structures of speech communities has inspired both empirical~\citep{Milroy1985,Herold1997,Trudgill2008} as well as theoretical and modelling work~\citep{Nettle1999,Silva2008,Gong2012,Blythe2012,Pierrehumbert2014}, some of which I will return to in the next chapter.
However, much empirical data suggests that the relevance of such a simple mechanism is limited.
%derived interaction frequencies between individuals~(such as due to social network structures)
\citet[ch.6-10]{Labov2010} in particular presents overwhelming evidence from work on the Atlas of North American English as well as the so-called Northern Cities vowel shift, showing how the linguistic systems of neighbouring communities with plentiful cultural and economic contact between them are in fact often diverging in opposite directions rather than converging. Rather than reducing changes to a simple mechanism of interaction, these results are seen to indicate that the adoption of competing linguistic variants is not just a matter of automatic mutual accommodation, but that each individual's choice of change is a question of the variants' social meaning in use or, more concisely, a question of \emph{prestige}.

% mislearning:
%1. people 'fail' to meet their speech goals
%2. kids hear these errors and decide they're the real goal
%failing is more optimal:
%Todorov and Jordan 2002: "variability which is larger in task-irrelevant directions" in optimal feed-forward synergies
%+corollary 'minimal intervention principle' (2003, p.28)

\subsubsection{Prestige accounts}\index{prestige}
\label{sec:prestige}

Once again, terminological caution is advised, this time in relation to the many possible pressures referred to by \emph{prestige}. While prestige is intuitively understood to stand for a social bias \emph{towards} something, the word leaves open whether this bias is towards the speech of a high-status \emph{individual}, however they may be speaking at the time, or instead towards specific linguistic \emph{variants}~(which might not actually be used by any prestigious individuals, but maybe merely believed to be). While early uses of the word are more typically referring to the (linguistic) status of individuals~\citep{Tarde1903,Fries1949}, today the term is almost exclusively used to refer to \emph{variant prestige}, a development that can be traced back to the influence of the idea of the `linguistic marketplace'~\citep{Bourdieu1977} that has taken a strong hold in sociolinguistic thinking~\citep{Cedergren1987,Tagliamonte2015}.

Unlike the intuitive reading of `prestige', in the sense of an socially explicit positive valuation of a linguistic variant, the term has gathered additional meanings, such as the that of `covert' prestige~\citep{Trudgill1972}, which has made it a bit of a catch-all term that lacks specificity.
While the gist of prestige is still that linguistic changes are adopted (and in this sense \emph{caused} or \emph{triggered}) based on underlying changes in the social structure of a speech community, \citet{Labov2001} acknowledges that

\begin{quote}
the force of Tarde's explanation [of reducing linguistic changes to underlying social pressures] may be considerably weakened if the term "prestige" is allowed to apply to any property of a linguistic trait that would lead people to imitate it. Thus the fact that a linguistic form has prestige would be shown by the fact that it was adopted by others.~(p.24)
\end{quote}

\citep{Sankoff1988,Armstrong2007}

The exact explanatory role of \emph{prestige} in sociolinguistics today is hard to pin down, not just due to the many different strands of research approaches present in the field~\citep{Tagliamonte2015}. At least the mainstream approach laid out by~\citet{Labov2001}, however, is clearly dedicated to some sort of sociolinguistic \emph{reductionism} of linguistic choices to underlying structures. Labov explicitly embraces the standpoint of~\citet{Meillet1926}:
%to ``follow Meillet's position that the sporadic character of language change can only be explained by correlations with the social structure of the speech community in which it takes place (1926:17)''~\citep[p.xv]{Labov2001}


\begin{quote}
From the fact that language is a social institution, it follows that linguistics is a social science, and the only variable element that we can resort to in accounting for linguistic change is social change, of which linguistic variations are only consequences, sometimes immediate and direct, more often mediated and indirect. \ldots We must determine which social structure corresponds to a given linguistic structure, and how in general changes in social structure are translated into changes in linguistic structure.~(translation from \citealt[p.22-23]{Labov2001})
\end{quote}

That this position is not taken by all sociolinguists is evidenced by debates regarding whether social valuation exists prior to language use at all, as reflected in work on language and identity from which the sociolinguistic account of language change draws: \citeauthor{Bucholtz2004} note that ``much work within variationist sociolinguistics assumes not only that language use is distinctive at some level but that such practices are reflective, not constitutive, of social identities'', whereas in much of the linguistic anthropology work ``identity is better understood as an outcome of language use rather than as an analytic prime''~\citep[p.376]{Bucholtz2004}.

While sociolinguistic research has equally contributed to the identification of general, universal pressures that seem to be guiding language changes~\citep[see][for an extensive summary in relation to sound changes in particular]{Labov1994}, most contemporary population-level fieldwork on language variation and change is trying to come to terms with the idiosyncracy of individual changes. The fact that changes sometimes spread from `above' and sometimes from `below' the level of conscious awareness~(ibid.), from females to males or the other way around~\citep{Milroy1985,Eckert1989,Labov2001,Sundgren2001} challenges the idea that a purely mechanistic theory of language change is possible.
But no matter along which social dimension innovations spread first, a change is only a change if it is increasingly adopted by a speech community across the board. While research has shown that %, in contrast to the assumption of a critical period after which language acquisition is in a sense `complete' and language
language change~(or at least the adjustment of variable usage rates) within an individual's life span is indeed possible~\citep{Sankoff2007,Buchstaller2015}, the canonical case of change is still regarded to be the \emph{incrementation} of changes by younger speakers relative to their parental generation~\citep{Labov2001,Tagliamonte2009}. The fact that this incrementation occurs along the same social lines means from generation to generation has raised a particular interest in how the \emph{structured heterogeneity} of sociolinguistic variation is acquired by children and adolescents.

\subsubsection{Incrementation and age vectors}\index{age vector}
\label{sec:agevector}

Most traditional accounts of language change are based on the assumption that linguistic divergence occurs during language acquisition, mostly based on language-internal factors that make learners `mislearn' or `reanalyse' their linguistic input, which causes them to end up with a different target language than that spoken by their caretakers~\citep[see e.g.][]{Salmons2013}. But quantitative research on infant and adolescent speech has painted a much more refined picture of the \emph{target} of child language acquisition~\citep{Labov1989,Labov2012}. Of particular relevance is the question of how individuals acquire sociolinguistic variation, and how this acquisition develops over time. Quantitative studies of the linguistic patterns of different pre-adolescent age cohorts has shown that, while children's usage patterns might mirror the language use of their caretakers up until about age three or four, learners then exhibit a pronounced ``outward-orientation'': shedding most of the influence of their caretaker speech, learners instead turn not just towards their peers, but towards the usage patterns in the wider speech community as a whole~\citep{Labov2014}. % also Labov2001, ch.13

An exemplary case of this behaviour is the data collected from children in the new town of Milton Keynes, England, which provided a natural testbed for the study of the acquisition of a local dialect against the backdrop of massive individual variation: the settlement expanded massively in the 1970s and 1980s, with most residents moving in from other dialect regions~\citep{Kerswill1994,Williams1999}.
Figure~\ref{fig:miltonkeynes} shows the distribution of variable realisations of the \textsc{GOAT} lexical set~\citep[the vowel in English `goat', `boat', `fold' etc., see][]{Wells1982} by children of different ages growing up in Milton Keynes, as well as the usage rates of their caretakers. What is striking about this data is not just the fact that children appear to switch from imitating their caretakers to imitating the wider community usage at some point after age~4, but also the accuracy with which children manage to replicate the usage distributions of the relevant target group.

This same pattern of acquisition is found in a slightly refined fashion in \citet{Sankoff1973}'s study of the acquisition of the future tense marker~`bai' in Tok Pisin, which underwent a change from being stressed to unstressed as it was grammaticalised during the development of the creole. Figure~\ref{fig:tokpisin} shows rates of secondary stress on `bai' by children, alongside the stress rates exhibited by their respective caretaker(s), connected by lines. What is evident is not just that children are producing fewer stressed tokens, but that their stress rates are lowered at similar rates \emph{relative to the stress rates of their caretakers}. This has led researchers to propose that adolescent learners do not just acquire the variable elements of language use according to social and stylistic constraints to a high degree of precision, but that they also advance changes in variable usage along their respective ``vector of change'':

\begin{quote}
In the incrementation of change, children learn to talk differently from their parents and in the same direction in each successive generation. This can happen only if children align the variants heard in the community with the vector of age: that is, they grasp the relationship: the younger the speaker, the more advanced the change.~\citep[p.344]{Labov2001}
\end{quote}



\begin{figure}[t]

{\centering \includegraphics[width=\maxwidth]{figure/miltonkeynes-1} 

}

\caption[With age, Milton Keynes children exhibit increasing alignment to community rather than caretaker speech.]{Phonetic targets of the \texttt{GOAT} vowel for Milton Keynes children by age exhibit increasing alignment to community rather than caretaker speech~(data from~\citealt{Kerswill1994}).}\label{fig:miltonkeynes}
\end{figure}



\begin{figure}[t]

{\centering \includegraphics[width=\maxwidth]{figure/tokpisin-1} 

}

\caption[Rate of secondary stress on the `bai' future tense marker in Tok Pisin]{Rate of secondary stress on the `bai' future tense marker in Tok Pisin. The relative difference between the rates of children~(circles) and their respective caretakers~(triangles) exhibit a similar pattern of increase across pairs~(data from~\citealt[p.425]{Labov2001}).}\label{fig:tokpisin}
\end{figure}



%Recently, speakers' perception of changes has even been proposed as a solution to the problem of \emph{incrementation}

% TODO mention here that this is basically the `apparent time' construct that sociolinguists use

While this quote does not speak to the initial triggering or actuation of a change, it does suggest that being able to detect the direction~(and~rate) of a change could be fundamental to a mechanism which allows speakers to \emph{advance} language changes systematically across generations.
Even though the concept of age vectors has been taken up as an explanatory device to analyse and account for trends in macro-level data~\citep{Labov2012,Sankoff2013,Stanford2014,Driscoll2014}, direct testing of the underlying assumption,%about individual behaviour
i.e.~that ``youth who are engaged in the incrementation of a sound change have some perception of the age vector''~\citep[p.369]{Labov2010}, has been more limited. The most compelling evidence comes from experimental studies of listener adaptation based on sociophonetic knowledge about ongoing changes: \citet{Drager2005} and \citet{Hay2006} showed experimentally that listeners use their implicit knowledge about age-specific speech patterns to adjust phoneme boundaries when classifying vowels depending on the purported age of the speaker, which was manipulated experimentally. % Drager2005 is perceived age of the voice, Hay2006 tests the hypothesis explicitly by presenting identical voices with different age+social class guises % TODO Drager2011? % TODO during an ongoing change

Even though the concept of age vectors was originally conceived of as applying to both continuous and categorical changes~\citep[p.346]{Labov2001}, in contemporary research it is now mainly applied to (continuous) phonetic changes. Here, the `age vector' can quite literally be taken to be a vector in phonetic space. This interpretation can be intuitively derived from traditional representations of phonetic changes in progress, where arrows are drawn in acoustic~(particular vowel) space to indicate the difference in pronounciations between older and younger speakers of a community~\citep[see e.g.][]{Labov2001}.

% such as the one shown in Figure~\ref{fig:agevectors}.

%\begin{figure}
%\includegraphics[width=]{}
%\caption{TODO: some Labov chain shift graph}
%\label{fig:agevectors}
%\end{figure}

This specialisation of the notion of age vectors is merely a consequence of the fact that most sociolinguistic research is based on \emph{sound changes}, particularly ones of a gradual (rather~than categorical) nature, such as the canonical example of vowel shifts. \index{sound change}
While there is consequently also a rich research tradition on individuals' sociophonetic knowledge~\citep{Foulkes2006}, work on sociolinguistic awareness (or indeed sociolinguistic indexicality) in the domain of necessarily categorical \emph{syntactic} change is generally underrepresented, a fact that I will address in Chapter~\ref{ch:questionnaire}. %with relatively less work on purely morphosyntactic variables.

%~\citep{Labov1994} (this fact is best exemplified by the fact that the 3 volumes of \emph{Principles of Linguistic Change} only contain X cases of non-phon(etic|ological) changes, so that a title like \emph{Principles of Sound Change} might actually have been more appropriate.} %\citep{Labov2016underreview}


% TODO explain how age vectors should be conceived of in frequency (distribution) rather than phonetic space

While a more detailed analysis of this very idea of the \emph{amplification of linguistic trends} forms the central part of this thesis, the overview of competing `explanations' of change presented thus far has painted a rather incoherent and divergent picture of research on language change.
Although the diversity of approaches and results makes it difficult to provide anything like a coherent or comprehensive overview of current opinions, I have already suggested the existence of something like a fault line dividing the field, characterised by two opposing views of language change based on distinct research methodologies.
While mechanistic and adaptively-minded accounts draw evidence both from macro-level typological and individual-level experimental data, sociolinguistic work based on social population-level phenomena stress the arbitrary, haphazard selection and diffusion of changes.

Rather than simply pick one side and more or less implicitly disregard the other, this thesis is in pursuit of another higher-level goal, namely to unify these seemingly contradictory approaches and show that they are in fact compatible.
%A constructively-minded linguist might have read through the historical development of the different accounts and approaches, fond of the methodological pluralism.
Before pointing to the evolutionary framework in which the apparent clashes can be reconciled, I will make a final effort to convince even the staunchest optimist that there truly is a schism in the field that warrants talk about the need of unification and reconciliation, by discussing a recurring issue on the question of language change where the differences are as stark as nowhere else: the adaptive nature (or not) of language change.


\section{Language change and language adaptation}\label{sec:adaptationism}\index{adaptation}

%Having covered most if not all of the relevant theories and thinking about the \emph{how} and \emph{why} of language change, 
Only few quantitative linguists today would deny that language \emph{evolves}, i.e.~that it that it undergoes \emph{change by replication}. The question of whether language change is \emph{adaptive} on the other hand is far from settled, or rather it is assumed to be settled in different ways within different subfields studying language change. Interestingly, the fault line does not coincide with the traditional boundaries of linguistic disciplines, such as phonetics, phonology, syntax, etc. Rather, as I have argued, it is largely a question of research methodology as well as the \emph{scope} of generalisation aimed for by different language change researchers.

The opinions held by the two camps -- those who hope to gain insight from framing language change as \emph{optimisation} of some sort vs. those sceptical of or categorically opposed to such a viewpoint -- go back far.
In light of the necessity to argue for some kind of \emph{asymmetry} to explain the shift from one linguistic variant to another, it is comparatively easier to find explicit arguments for change as improvement or adaptation of the lingustic system, such as \citet{Zipf1949}'s \emph{principle of least effort} discussed above.
The orthogonal view that change really is \emph{just change}, with no particular goal or direction, was in a sense the default view not just of laypersons who are generally not found to be fond of linguistic innovations~\citep{Labov2001}. The same view was adopted by many early Indo-Europeanists who, if anything, saw the beauty and perfection of ancient languages tainted and eroded by unsystematic, haphazard changes~\citep[ch.II]{Jespersen1922}. As such, anti- (or at least non-)adaptationist arguments per se are found most explicitly as direct contradictions of adaptationist claims:

\begin{quote}
Taking linguistic change as a whole, there seems to be no discernible movement towards greater efficiency such as might be expected if in fact there were a continuous struggle in which superior linguistic innovations won out as a general rule.~\citep[p.69]{Greenberg1959}
\end{quote}

This point of view was taken even more strongly by researchers not working on cross-linguistic typology, such as in \citeauthor{Postal1968}'s well-known, somewhat more colloquial statement on syntactic change:

\begin{quote}
There is no more reason for languages to change than there is for automobiles to add fins one year and remove them the next, for jackets to have three buttons one year and two the next, etc.~\citep[p.283]{Postal1968}
\end{quote}

Despite very different methodological approaches and research agendas at the time, similar thinking prevailed at the other end of linguistic inquiry into phonetics:

\begin{quote}
I do not think sound change creates any significant improvement or defect in language. There is sufficient redundancy in language that the message which speech encodes gets through as well (or no worse) before and after sound change. \ldots All human languages manifest asymmetry or disequilibrium in some part of their phonology but seem, nevertheless, to function adequately for communication. \ldots There is no need to appeal to ill-defined notions such as "system pressure", "pattern symmetry", "equilibrium", and the like, nor to maintain that the language is any better or any more "fit" as a result of these sounds changes. \ldots It seems languages are never satisfied! More to the point, it seems that languages are not seeking the satisfaction of some "ideal" configuration.~\citep[p.191-192]{Ohala1989}
\end{quote}

Particularly within the study of sound change, this point of view is now widely accepted and essentially become textbook knowledge. For example, the entire introductory chapter of \citet{Labov2001} is dedicated to what he calls the \emph{Darwinian paradox}: \index{Darwinian paradox}

\begin{quote}
There is general agreement among 20th-century linguists that language does not show an evolutionary pattern in the sense of progressive adaptation to communicative needs. \ldots The almost universal view of linguists is the reverse: that the major agent of linguistic change -- sound change -- is actually maladaptive, in that it leads to the loss of the information that the original forms were designed to carry.~(p.10)
\end{quote}

%\citet[p.39]{Croft2000}
%``the empirical evidence indicates that linguistic selection is governed largely if not exclusively by social forces that have little or nothing to do with functional adaptiveness for communication.''

The same view is also adopted by \citet{Croft2006}, who concludes that ``all of the empirical evidence in language change indicates that social factors, not functional ones, are the causal mechanisms for the propagation of a change''~(p.116).

The list of supporters of the viewpoint that language change is at least to some extent adaptive is just as long and illustrious, with dedicated arguments for the cause by \citet{Jespersen1949}, \citet{Vennemann1993} and~\citet{Haspelmath1999,Haspelmath2008}.
Especially within typology the discussion seems to have reached a conclusion in the opposite direction, for example when~\citet{Wichmann2015} finds that ``Presumably language change is only to some extent random, and to a larger extent is adaptive''~(p.221).

%The general pattern seems to be that researchers diverge in their conclusion depending on whether they focus on the macro-level of language change -- \emph{which changes tend to occur} -- versus the micro-level of individual changes -- \emph{how do individual historical changes unfold}. This apparent schism in research questions and conclusions will reappear throughout this thesis, which itself will be concerned with the micro-level dynamics of individual language changes.

%Perhaps a more extensive argument for why actuation problem also undermines the explanatory power of universal pressures is necessary. This can also be made in a conceptual, visual manner. imagine space of all possible languages~(as is often done by functionalists, e.g. Jaeger2010).

%Most changes happen in the middle region and are kind of pointless. The universal pressures are satisfied in this space, so we shouldn't expect there to be strong intra-linguistic pressures driving the selection of particular variants.

%Wedel's functional load result is a case in point: let's assume (I think this would be in the spirit of Wedel) that there is a general laziness/economy/blending bias that would lead all phonemes of a language to merge. Since the 26 minimal pair threshold is a probabilistic rather than a deterministic predictor, we should probably think of the influence of each individual minimal pair as contributing towards an overall functional selection bias against merging the phonemes, a 'distinction-maintaining' bias that is in direct competition with the blending one. Now between those two pressures there is a sweet spot somewhere around 26 minimal pairs where the two biases balance each other out, so that at this point the decision of whether to merge or not to merge could go either way. On the 'macro' level, where we just ask the binary yes/no question of whether the population adopts a merger or not, this might be a satisfying result, but in our opinion it fails to speak to the 'micro' level of individual behaviour: what happens at or around this critical point, is there some non-linear response across individuals where half of them exhibit a tendency to merge the phonemes and half don't? Or, assuming a more gradual effect of the bias, wouldn't we enter the regime of neutral evolution of variants within every individual where the entire population should (in a synchronised fashion) drift around between sometimes merging the phonemes, and sometimes not?

%This question of the 'sweet spot' at which universal functional pressures that have presumably always been there on the individual level without affecting the community language suddenly kick in and start an ordered directed change is what the 'actuation problem' is all about. It should be noted that the quasi-deterministic requirement for a model of language change as quoted by the reviewer that "would predict, from a description of a language state at some moment in time, the course of development which that language would undergo within a specified interval" is not what Weinreich et al. had in mind either, in the immediately following paragraph they reveal this bold goal to be nothing but a straw man, stating that "Our own view is that neither the strong nor the modest version of such theories of language change, as they proceed from current generative grammar, will have much relevance to the study of language history".

%However, in our view a full account of the pressures that drive language change *has* to make specific predictions at the level of individual behaviour, since this is what gives rise to the population-level behaviour and not the other way around. In other words, accounting for a macro-level pattern while completely underspecifying how exactly the probabilistic pressures (as observed from cross-linguistic samples like Wedel's) actually get expressed on an individual level during one particular change won't do. While macro-level accounts (like the results regarding functional load) are of course worthwhile and interesting it can be dangerous to equate them with explanations for individual events. What we are asking for here is not a deterministic account of whether a change occurs or not, we are still dealing with probabilistic models here, but models of individual behaviour that, when put into interaction with other individuals, yield predictions about population-level shifts alongside equally specific predictions about the underlying individual behaviours that give rise to the different population-level scenarios.

A striking difference of focus in these last two quotes can be found in the `causal mechanism' for the propagation of particular changes, which is evidenced at the micro-level of the diffusion of changes, versus the `adaptive' nature of language changes that is observed as a synchronic fact.
What becomes evident here is that the idea of an `explanation' of a change depends on the level at which it is investigated:

\begin{quote}
There is general agreement that the heart of the study of language change is the search for causes. It is what we generally mean by the explanation of change. While we would like to apply to this search the universal principles that govern grammar as a whole, it is also understood, following Meillet~(1921), that no universal principles can account for the sporadic course of change, in which particular changes begin and end at a given time in history. The actuation problem demands that we search for universals in particulars.~\citep[p.90]{Labov2010}
\end{quote}

This quote succinctly summarises two different approaches to the question of causation, or `why' languages change. Firstly, Labov acknowledges that there are in fact two senses of `why', implying separate research interests with diverging priorities regarding the locus of `explanation'. On one hand, for researchers seeking to generalise over language changes, the `why' question is not actually asked of individual changes in particular, but is first posed as a general cross-linguistic `what' or `which', in order to identify universal guiding principles of change, such as the unidirectional patterns that have been shown to hold across languages. The simple `why' then becomes a `why \emph{that}, over and over again?' which, once a universal pressure has been identified, offers an explanation that is taken to generalise to individual instances of the change, without the need to refer to particular features of those instances. The second sense of `why' is the one most pressing for the researcher looking at particular changes, and is best rephrased as `why there and then'? This question has to take the actuation problem straight on, and can only be answered with particular reference to the specific conditions for the change, so as to explain why it actuated exactly when it did.

While diverging in their focus of explanation~(to the degree of incompatibility), the two approaches share a common scientific standard: both seek to explain changes in a deterministic fashion, in the sense of reducing changes to accountable, rule-like development, merely diverging on whether the underlying mechanism of change is based on micro- or macro-level descriptions of the change. What I want to propose here is that the two approaches can in fact be unified, but only by abandoning that very standard of explanation.
Counter to~\citet[ch.5]{Labov2010}, who argues for the existence of language-internal `triggering events' that can account for the actuation of particular changes, I want to suggest that one has to acknowledge the fact that individual changes occur at points in time that are, at least to the level of description available to the linguist, arbitrary and thus unpredictable.

Although this point of view would presumably resonate with many a researcher who is content with revealing generalisable patterns of linguistic changes, my goal is not to discard or even play down the actuation problem. 
What I suggest is to shift the focus away from the negative, seemingly irrefutable criticism of universal pressures to fail to account for the idiosyncratic nature of particular changes and their seemingly arbitrary onset. Instead, the criticism can be transformed into a challenge of constructively explaining the very feature of language change that is the underlying source of the actuation problem, namely its sporadic nature.
A complete theory of language change should not just offer an explanation of the macro-level dynamics and general direction of language change, but also provide a concise account of exactly \emph{why} the actuation of particular changes is unpredictable. %Rather than being oblivious to the question of actuation, a complete theory should not just predict the predictable general direction of change on the macro-level, but also provide an account that explains why the onset of particular changes at the micro-level is so sporadic.

As I showed above, much headway has been made on the question of how and in what way languages change in general, generalisations which are based on universal pressures and asymmetries. If one accepts those pressures which account for the strict uni-directionality of many changes and that seem to dictate the prevalence of linguistic traits across the globe, how can there still be room for systematic uncertainty and indeterminacy in language change?
Even though adaptive accounts of language change today are well-motivated and grounded not just in typological observations but also in theories of individual processing~\citep{Kirby1999,Jaeger2010}, neither of these two levels are sufficient to conclude that specific changes spread \emph{because} they are adaptive. So when \citet{Wichmann2015} writes that language change is to a large extent adaptive, this statement can be re-read somewhat less untuitively as `most changes that spread through populations happen to be adaptive (but that is not necessarily the causal reason why they \emph{do} in fact spread)'.

The crucial point is that the mere fact of adaptation, or even evidence for the preferred spread of adaptive traits, should not be equated with evidence for \emph{selection for the adaptiveness} of that trait or structure. % A case for replicator-neutral theories of language change.
The intention here is of course not to discard functional and adaptive pressures completely, but to raise the question of whether they should be construed as the pressures which drive the diffusion of linguistic variants through a speech community in the individual cases.
Instead, I want to stress how functional factors actually play an important role \emph{within} social accounts of language change and that by taking an evolutionary approach that distinguishes separate pressures behind the \emph{innovation} of linguistic variants and their subsequent \emph{selection}, the two viewpoints can in fact be unified.

Before elaborating on the importance of an evolutionary approach though, I will have to discuss quantitative approaches to language change more generally, as well as the insights into the dynamics of changes which are relatively well understood thanks to centuries of quantitative research on historically attested changes.

\section{Language change: a quantitative framework}

So far, the language changes referred to as brief examples for different accounts above were primarily characterised through qualitative descriptions. Given the vast range of different types of changes -- lexical, morphosyntactic, phonological, phonetic, semantic -- it might even appear ludicrous to speak of `language change' as one phenomenon that should be subsumed by a single framework.
However, quantitative approaches to many different types of historical language change have made use of similar descriptions to characterise the unfolding of changes. While historical linguists were the first to be concerned with the \emph{variation} exhibited by languages as they undergo change from one state to another, the sociolinguistic study of synchronic variation within the individual broadened the idea by introducing the concept of the \emph{sociolinguistic variable}.

\subsection{The sociolinguistic variable}\index{sociolinguistic variable}
\label{sec:sociolinguisticvariable}

Forming part and parcel of the sociolinguistic or general \emph{variationist} research framework today,
``A linguistic variable in its most basic definition is two or more ways of saying the same thing''~\citep[p.4]{Tagliamonte2012}.
In more formal terms, different \emph{variants} of one of the same \emph{variable} have the property that they are ``identical in truth value, but opposed in their social and/or stylistic significance''~\citep[p.271]{Labov1972}. % in Nardy2012, p.256
The identification of a sociolinguistic variable is always based on one particular language or vernacular, where typically one will find one and the same speaker to be using and mixing the variable's two or more different variants in some probabilistic fashion. A canonical example in English is the variable~\texttt{(ing)}, which refers to the variation in how the final nasal in the gerund suffix \emph{-ing} is pronounced, with the two variants \textipa{[n]} and~\textipa{[N]}.

While this particular example concerns a phonetic variable which has seen stable variation in its usage for centuries~\citep{Labov1989}, sociolinguistic variables can equally be constructed for variable usage in other linguistic domains, and with a particular eye on capturing the unfolding of language \emph{change}. In Chapter~\ref{ch:questionnaire} we will be concerned with syntactic changes that have been unfolding in Shetland Scots over the past century. One of the variables, call it~\texttt{(ynq)}, refers to the syntactic realisation of yes/no questions. The variable can be realised by either of two variants, one being the use of the older, verb-initial sentence order, the other a more modern, Standard English-like formulation employing a sentence-initial periphrastic `do'. By tracking the usage rates of different variants of a variable over time, for example by counting their relative occurrence in recordings or corpora at different points, one can capture the temporal dynamics of a change as the novel, \emph{incoming} variant gradually replaces the \emph{outgoing} form.

%While between-individual variation in the English gerund based on geographical location and socio-economic status as well as within-individual variation based on context and speech style

Although language change can be studied on levels other than that of the sociolinguistic variable~\citep[see in particular][p.98]{Croft2006}, it is chiefly changes at this lowest level, namely in the realisation of individual sociolinguistic variables, that will be covered in this thesis. And, since I have set out to delineate a theory of language change with a particular eye on reproducing the micro-level dynamics of the diffusion of novel linguistic variants, we should have a closer look at what is known~(or believed) about how individual language changes spread within communities.

% subsection: s-shaped curves in language change
\input{introduction/scurves}

\section{Language change as language evolution}\label{sec:evolutionaryapproaches}\index{evolution}

It has often been noted that Darwin's idea of the gradual evolution and diversification of biological species was inspired by philological work of the time which described the historical relationship between different languages~\citep{Darwin1871}. Since then, the field of biological evolution has developed a large theoretical toolkit for capturing and evaluating the replication and unfolding of biological populations in a rigorous, quantitative fashion. This apparent progress, again measured in terms of the scientific standard of the natural and physical sciences, has caused researchers in some social and cultural sciences, but especially linguistics, to adopt similar approaches to describe and explain language change~\citep[to list just a few monographs]{Croft2000,Blevins2004,McMahon2005,Ritt2009}.
Crucially, `evolutionary' approaches to language and language change differ immensely in what aspects of biological evolution they try to import or emulate, as well as how~\citep{Croft2000}. While sometimes biological concepts are merely used as a metaphorical basis to explain linguistic processes that they share some superficial similarity with~\citep[e.g.][]{Lass1990}, others are mainly focussed on transferring existing mathematical methodology over to a new field~\citep{BorgerhoffMulder2001,Atkinson2005a,Jager2008}.

The evolutionary approach to language change adopted in this thesis follows the one outlined by~\citet{Croft2000}, which is in turn based on \citeauthor{Hull1988}'s \emph{generalised analysis of selection}. The fundamental principle of this generalsed analysis is that of defining evolution as \emph{change by replication}~\citep[p.410]{Hull1988}. As I already alluded to above, language evolution in this sense is characterised not by any `inherent' change of an entity~(such as a `language'), but by changes to the type and distribution of a population of entities that are replicated individually.% with replication resulting either in finear-identical basis to a higher or lesser degree of fidelity.
Whereas in biological evolution the \emph{replicator}~(typically assumed to be the genome) and the \emph{interactor}~(the corresponding phenotype that is central to the replication of the genome) are united in one and the same biological entity, this is just one special case of `evolution' in \citeauthor{Hull1988}'s general framework. The generalised analysed allows for the replicators and interactors to be separated out, which is the stance taken by \citet{Croft2000} who characterises language change as a process in which passive linguistic structures are replicated by human \emph{interactors} whenever they engage in linguistic behaviour.\index{interactor}
%This has parallels in \citet{Boyd1985}'s models of cultural evolution.

Croft calls his theory the `utterance selection model of language change', stressing the fact that language change should not be sought in individual minds or abstract linguistic identities, but instead in the linguistic \emph{utterance} as the individual speech act used in a specific context. The paradigm \emph{replicator} in linguistic evolution for him are linguistic structures that are contained within utterances, which he calls \emph{linguemes}, and which he identifies with the linguistic variant as studied by variationist sociolinguists~\citep[p.104]{Croft2006}.
%One subtle difference in the rhetoric of a strict evolutionary account is in how language change is framed. In particular, I will avoid adopting any metaphorical framing of linguistic change that somehow frames language as an active entity when, as a group of imitated human actions, they are really linguistic behaviours that are replicated passively.
%Takes that even implicitly place the drive of languages to change as a feature of the language obscures the need to be explicit about the fact that we are looking for are mechanisms of the individual that leads them to alter the set of linguistic behaviours they engage in in some way.

This new concept of the \emph{lingueme} as the fundamental entity of linguistic evolution also entails a specific theory of its \emph{replication}. Parallel to the concepts of \emph{selection} and \emph{mutation} in biological evolution, Croft distinguishes between two distinct evolutionary pressures, namely those of \emph{differential replication} and \emph{altered replication}~\citep{Croft2000}. Differential replication describes changes where the distribution of existing variants changes due to preferential but otherwise `correct' or near-identical copying of existing variants by a human interactor, a process that I will call \emph{selection}. Altered replication on the other hand results in the spontaneous production or \emph{innovation} of a new variant that is either derived from or otherwise completely independent of other known variants.

This characterisation of language change as consisting of two separate mechanisms that can at least to some degree be described independently is not completely new in linguistics.
\citet[ch.XV §11]{Jespersen1922} makes an explicit distinction between the question of how sound changes \emph{originate} in individual speech as opposed to how they \emph{spread} to other individuals, which he frames a matter of imitation, i.e.~replication of existing variants. A similar sentiment is expressed by \citet{Weinreich1968}, who trace replicator-based thinking in language change as far back as~\citet{Paul1880}.
That same micro-level approach is by far not shared by all linguists working on language change, and it should be noted that this is just as true of research done under the \emph{language as a complex adaptive system} moniker, where the term `evolution' is still often used simply as a synonym for `adaptation'~\citep{Croft2000}.\index{language as a complex adaptive system}

To highlight the relevance of this replicator-based evolutionary take in the light of the research results on language change presented so far, it is worthwhile to re-iterate a basic insight from evolutionary biology regarding the relationship between synchronic distributions of traits and the supposed underlying pressures that led to those distributions, namely that ``the prevalence of a particular species in a habitat does not necessarily imply that it is any better adapted to that habitat than its competitors''~\citep[p.2]{Blythe2012copying}.
This quote is a word of caution against not one but two simplistic assumptions that are often made when extrapolating from observed adaptations to the supposed selection of those adaptations. The first is the influence of \emph{neutral evolution}, \index{neutral evolution}
i.e.~the effect that the random fluctuations inherent in the replication of a finite number of replicators can have on causing the widespread adoption of a trait in the absence of any replicative advantage of that trait. But, given the directed nature of the adoption of language changes suggested by their s-shaped trajectories, neutral theories have played only a very minor role in linguistics, a point that I will return to below.
Instead, I want to highlight a second, less obvious concern regarding different possible sources of \emph{asymmetry} that might underlie the skewed synchronic distribution of linguistic traits.

From the evidence discussed above we can conclude that language change looks very directional, but it does so on two distinct levels: firstly, we find unidirectional patterns in the way languages change, suggesting some consistent asymmetry that, when described at the cross-linguistic level, makes changes go in similar \emph{directions} over and over again. At the macro-level we are therefore concerned with a temporally reduced characterisation of a change that encompasses both the emergence as well as ultimate adoption of an \emph{innovation}.
Crucially, this sense of direction should not necessarily be equated with the \emph{directionality} of the individual trajectories of change which exhibit the aforementioned s-shaped pattern as the individual competing variants undergo \emph{differential replication}. %At the micro-level of the change as the differential replication of existing variants, all variants are in a sense created `equal', with the exception of their initial frequency, which is necessarily lower for the novel

This distinction might seem far-fetched at first, but it is a central argument of this thesis that the macro-level sense of direction that one gets from the re-occurrence of similar changes towards similar adaptive goals and the micro-level diffusion of individual variants that forms the building block of language change are due to fundamentally different pressures.
%Taking an evolutionary approach seriously is about taking in consideration that those two asymmetries might be due to different pressures.
As already alluded to above, I will argue that the influence of adaptive and functional factors should be primarily thought of as affecting the domain of \emph{innovation}, where it is responsible for asymmetries in the generation of new variants. The \emph{selection} pressures which are the driving force behind the directed propagation of individual changes, on the other hand, will be identified with arbitrary social biases which do not in principle distinguish between linguistic novelties that are adaptive or maladaptive.
Simply due to the fact that selection pressures can only apply to variants which are already part of the pool of synchronic variation~\citep{Ohala1989}, the reliable re-occurrence of similar changes can be explained due to their relatively higher likelihood of being innovated~\citep{Joseph2013}. Importantly, though, this approach does not completely rule out changes that go in the direction opposite of any pattern of uni-directionality that might have been identified, but merely predict that those changes would be much less frequent. Attributing the selection of changes to arbitrary social pressures simplifies the matter of accounting for such particular, unusual changes that run counter to universal trends, which would otherwise not just require explanation of why a universal pressure did not hold in a particular instance, but also the origin of a presumed selection pressure in the opposite direction.

%``all of the empirical evidence in language change indicates that social factors, not functional ones, are the causal mechanisms for the propagation of a change''~(p.116).
This separation of concerns in a strict evolutionary account solves the problem of accounting for both universal trends and particular, idiosyncratic changes, providing a natural solution for the scientific conundrum encountered not just in linguistics but also in other social historical sciences~\citep{Blute1997}. Even though the explanatory power of this approach has been promoted heavily in particular by~\citet{Croft2000,Croft2006,Croft2008}, the systematic distinction between innovation and selection pressures has not widely caught on in the field. The clear evidence for functional asymmetries in combination with a lack of sound non-teleological mechanisms that could explain the arbitrary and sporadic actuation of particular changes has meant that this second selection step of language change has either been left unaccounted for, or otherwise assumed to be driven by the very same language-internal asymmetries.
The main goal of this thesis is to give additional support to the two-step model of language change by filling in the current gap regarding the second, arbitrary selection step with a symmetric selection mechanism that nevertheless produces strongly \emph{directional} transitions whose actuation, however, is sporadic and temporally underspecified.
While thinking about such mechanisms is rather unusual for the domain of biological evolution, evolutionary approaches to cultural change more generally have given rise to a wider range of selection mechanisms that are worth exploring.

%macro-level studies can predict \emph{which} changes occur, but not \emph{that} they will occur, or \emph{when}.

%What is interesting under such an approach is that, quantitatively, many different kinds of pressures and accounts presented above fall together. Both functional and social \emph{prestige} pressures, for example, are biases towards specific linguistic variants, which means they are both \emph{content biases} under \citeauthor{Boyd1985}'s terminology. So the only way that they could be distinguished in practice is by testing quantitative predictions regarding the underlying origin or source of this bias.

%\subsection{Neutral evolution and random drift}\index{neutral evolution}
% more metaphorical: Sapir
% more mechanical: Bauer etc.

%Primarily due to increased transfer of mathematical tools from biology 

%Previously not considered, now on the table as very mechanistical accounts. Importantly, this (originally biological) notion of drift is not to be confused with \citet{Sapir1921}'s more figurative idea of linguistic \emph{drift}, which I will return to in the final chapter.

\subsection{Tackling the actuation problem}

%I hope to fill in this gap by investigating trend-amplification as a , thereby  that goes beyond mere verbal argument.

Virtually all approaches to explaining linguistic changes discussed so far struggle with one and the same aspect of its dynamics, namely the sudden incrementation or transmission of change to a part of its grammar following long periods of stagnation or stability. The inability to account for this \emph{sporadic} nature of language change is not always acknowledged, and in fact often outsourced to triggers that fall outside the scope of the specific account of change itself, whether implicitly in the case of language-internal~(as criticised by the \emph{actuation problem}), or explicitly in the case of social accounts.
While the LCAS paradigm recognises the complexity of the many interacting pressures that are thought to underlie the spread of specific changes, the exact nature and dynamic of the transitions from stability to change are hardly ever accounted for or explained explicitly.

\subsection{Beyond biological metaphors: regulatory pressures}

Beyond the attempts to explain language change in relatively simple mechanistic terms akin to the straightforward causal explanation of events that is perceived to prevail in some natural sciences, researchers have also acknowledged the inherent limitations of linear approaches in accounting for linguistic systems' behaviour that is in many respects irregular and non-linear~\citep[see e.g.][]{Fortescue2006}.
These very same effects have also been the subject of the quantitative study of cultural change more generally, particularly in relation to traits of fashion and other arbitrary conventions that cannot be easily attributed to external adaptive pressures~\citep{Acerbi2012}. What is particular about these domains of culture is that they exhibit ``a rigid conformity that cannot be broken by small shocks''~\citep[p.991]{Bikhchandani1992}.
Despite formal accounts of pressures and mechanisms %that are particular to cultural evolution
in which \emph{social learning} from other individuals plays a crucial role~\citep[see e.g.][]{Boyd1985}, \citeauthor{Bikhchandani1992} argue that of well-known social mechanisms such as punishment, conformity or explicit negotiation of conventions, none can explain ``why mass behavior is often fragile in the sense that small shocks can frequently lead to large shifts in behavior''~(p.993).

Beyond simple adaptive selection biases that are often associated with the concept of the biological fitness landscape~\citep{Kaplan2008,Gerlee2015}, researchers have been looking for mechanisms which can reproduce the rapid non-linear transitions found in population waves and bandwagon effects. Rather than relying on external triggers, these fads and fashions can be described as spontaneous cascades of behavioural change that emerge solely from individual interactions~\citep{Bikhchandani1998,Goldstone2008}. A specific set of candidate mechanisms behind such cascades which this thesis will hone in on is that of \emph{regulatory pressures}~\citep{Acerbi2014}, i.e.~culturally acquired traits which themselves regulate or steer the acquisition or selection of (other) cultural traits. Of particular interest here is the idea that features of the \emph{temporal dynamics} of the generation and adoption of cultural variants can affect their spread, in a self-reinforcing fashion. At first sight this type of explanation -- that a variant is going to be adopted more precisely because it is being adopted more -- might seem cyclical or tautological, but a mechanism based on the detection and amplification of trends exhibits a quantitative behaviour that matches well onto the dynamics of language change, as I will argue in much more detail in Chapters~\ref{ch:momentummodel} and~\ref{ch:conclusion}.
%noise-activated transitions~\citep{Mitchener2011,Mitchener2016}
Before we turn to computational modelling as a tool to investigate the predictions of such a dynamical model of change, however, we shall have a final look at the origin of the idea that the historical aspect of cultural traits itself can affect their dynamics of adoption, with respect to linguistic change in particular.

\subsection{The time dimension in linguistic thinking}

While evolutionary approaches as described above as well as the general spirit of reusing concepts and methodologies originally devised to study biological evolution to thinking about language change have become widespread~\citep{Atkinson2005}, only few efforts have been made to move beyond the straightforward models of simple mutation and selection which are typically known to laypersons outside the field of mathematical and evolutionary biology.
\citet{Dixon1997} represents an early attempt at explaining sporadic change by transferring the idea of \emph{punctuated equilibria} to language change, but in the absence of a concrete quantitative model applicable to linguists, the idea failed to grow beyond its metaphorical basis.

%\citep{Altmann2013}
While current work on the \emph{incrementation} of language changes by means of amplifying linguistic trends can be traced back to Labo's concept of \emph{age vectors}~\citep[ch.14 in particular]{Labov2001}, the idea that the \emph{history} of linguistic systems beyond just their present state can influence how individuals change the language can also be found in much earlier work. %, even if only very implicitly, or at second reading.
That the temporal stratification of a language constitutes a fundamental aspect of a grammar was for example argued by~\citet{Fries1949}:

\begin{quote}
It is impossible to give a purely synchronic description of a complex mixed system, at one point of time, which shows the pertinent facts of that system; direction of change is a pertinent characteristic of the system and must also be known if one wishes to have a complete description of the language as it is structurally constituted.~(p.42)
\end{quote}

While \citet{Bailey1970} still speaks about building ``rates of change'' into synchronic linguistic descriptions, it appears that the synchronically-focussed reductionist stance inherent in generativist approaches to language largely eradicated such thinking.
In the evolutionary approach to language change inspired by work in population biology, one typically thinks about the diffusion of the incoming \emph{variant} which is, beyond its initially low frequency, not explicitly assumed to be in any way marked relative to the established majority one.
Particularly in the sociophonetic literature though, one finds evidence of the notion of the diffusion of a \emph{change}~(rather than a variant), possibly to be construed as an incoming variant that is defined \emph{relative to} an established one. This becomes particularly apparent in \citet{Labov2001}'s notion of \emph{age vectors}. \index{age vector}
That this notion survived (or re-emerged) in the study of sound change is maybe not surprising, since the continuous nature of phonetic changes lends itself to thinking about changes as incrementing \emph{relative to} an average pronounciation. In this sense it is also possible for individual language users to produce even `more incoming' pronounciations by extending the age vector beyond its current level of advancement. This differs from the incrementation of categorical variants, where the only way for an individual to advance a change is by increasing the relative usage frequency of the incoming variant.

Showing that the same principle of incrementation can also hold for categorical variables is one goal of this thesis. Having gained a stronger sense of the historical origin and subtle continuity of the idea, we shall now face forwards and tackle the question of how the many competing accounts and their underlying \emph{biases}, \emph{pressures} and \emph{explanations} for language change can be evaluated quantitatively.
%Ellucidating the respective roles of the different pressures is one of the goals of this thesis.
The next chapter is dedicated to the role that computational modelling can play to ellucidate the validity of different accounts beyond verbal argumentation.%, and we will return to this point in Chapter~\ref{ch:bigpicture}.

%One concept that has stuck around despite confusion around its exact conceptualisation % meaning
%is \citet{Sapir1921}'s notion of linguistic `drift'~(not go be confused with genetic drift, aka neutral evolution). \index{drift}

% languages do not just diverge when there is social pressure to discriminate, but also have an internal `drive' to change (in place) by themselves (p.160).

%But Sapir also alluded to the fact that underlyingly related but in the meantime isolated languages tend to develop in the same way, or the same direction. 

%\section{Summary}

%that have been put forward it becomes obvious that it is often unclear how exactly these \emph{pressures} are supposed to be understood quantitatively.

%The next two chapters are dedicated to the computational and empirical investigation of a relatively understudied regulatory pressure, namely the detection \emph{trends} in the usage rates of linguistic traits, and their amplification by individuals.
