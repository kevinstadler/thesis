\section{What is language change?}

Examples, a working definition

\section{Explaining language change}

\subsection{Early accounts}

\subsection{Language-internal accounts}

%It is not our intention to discard functional pressures completely of course, the issue we wanted to raise in the introduction section is whether functional factors should be construed as replication pressures which drive the selection of linguistic variants. Wedel's functional load result is a case in point: let's assume (I think this would be in the spirit of Wedel) that there is a general laziness/economy/blending bias that would lead all phonemes of a language to merge. Since the 26 minimal pair threshold is a probabilistic rather than a deterministic predictor, we should probably think of the influence of each individual minimal pair as contributing towards an overall functional selection bias against merging the phonemes, a 'distinction-maintaining' bias that is in direct competition with the blending one. Now between those two pressures there is a sweet spot somewhere around 26 minimal pairs where the two biases balance each other out, so that at this point the decision of whether to merge or not to merge could go either way. On the 'macro' level, where we just ask the binary yes/no question of whether the population adopts a merger or not, this might be a satisfying result, but in our opinion it fails to speak to the 'micro' level of individual behaviour: what happens at or around this critical point, is there some non-linear response across individuals where half of them exhibit a tendency to merge the phonemes and half don't? Or, assuming a more gradual effect of the bias, wouldn't we enter the regime of neutral evolution of variants within every individual where the entire population should (in a synchronised fashion) drift around between sometimes merging the phonemes, and sometimes not?

%This question of the 'sweet spot' at which universal functional pressures that have presumably always been there on the individual level without affecting the community language suddenly kick in and start an ordered directed change is what the 'actuation problem' is all about. It should be noted that the quasi-deterministic requirement for a model of language change as quoted by the reviewer that "would predict, from a description of a language state at some moment in time, the course of development which that language would undergo within a specified interval" is not what Weinreich et al. had in mind either, in the immediately following paragraph they reveal this bold goal to be nothing but a straw man, stating that "Our own view is that neither the strong nor the modest version of such theories of language change, as they proceed from current generative grammar, will have much relevance to the study of language history".

%However, in our view a full account of the pressures that drive language change *has* to make specific predictions at the level of individual behaviour, since this is what gives rise to the population-level behaviour and not the other way around. In other words, accounting for a macro-level pattern while completely underspecifying how exactly the probabilistic pressures (as observed from cross-linguistic samples like Wedel's) actually get expressed on an individual level during one particular change won't do. While macro-level accounts (like the results regarding functional load) are of course worthwhile and interesting it can be dangerous to equate them with explanations for individual events. What we are asking for here is not a deterministic account of whether a change occurs or not, we are still dealing with probabilistic models here, but models of individual behaviour that, when put into interaction with other individuals, yield predictions about population-level shifts alongside equally specific predictions about the underlying individual behaviours that give rise to the different population-level scenarios.

%In summary we never intended to be dismissive of functional factors, and we have now expanded the introduction section to discuss how functional factors play an important role within social accounts.

The levels of language change: Fedzechina graph

child-based language change~\citep[p.109]{Croft2006}

\subsection{Social accounts}\index{social accounts}

Beyond intra-linguistic or functional effects, research have been looking for the effects that \emph{social} influences have on language change. The umbrella term `social' is somewhat unfortunate, as it covers two very different kinds of effects that are very different in their mechanism, one referring to the simple mechanistic effect that interaction frequencies between individuals~(such as due to social network structures) or also preferential copying from high-status individuals can have, in contrast to studies interested in the social \emph{prestige}, in the sense of socially meaningful valuation of linguistic variants. I will discuss the two kinds of pressures in turn.

\subsubsection{Mechanical accounts}\index{interactor selection}

principle of density/social network effects: models by \citep{Nettle1999,Silva2008,Gong2012,Pierrehumbert2014}

BUT results by \citep[was it that? Northern cities stuff]{Labov2001}

% mislearning:
%1. people 'fail' to meet their speech goals
%2. kids hear these errors and decide they're the real goal
%failing is more optimal:
%Todorov and Jordan 2002: "variability which is larger in task-irrelevant directions" in optimal feed-forward synergies
%+corollary 'minimal intervention principle' (2003, p.28)

\subsubsection{Prestige accounts}\index{prestige}

Despite efforts to the contrary, `prestige' has become a bit of an unfortunate cover term for many different presumed pressures within sociolinguistic research. 

Original idea from where?

distinction between overt/covert prestige, making it a bit of a catch-all term

attempts to make it more mechanistic again by talking about changes from above/below

The exact explanatory role of \emph{prestige} in sociolinguistics today is hard to pin down, not just due to the many different strands of research approaches present in the field~\citep{Tagliamonte2015}.

Whether social valuation exists prior to language use at all is an equally open question, as reflected in work on language and identity from which the sociolinguistic account of language change draws: \citeauthor{Bucholtz2004} note that ``much work within variationist sociolinguistics assumes not only that language use is distinctive at some level but that such practices are reflective, not constitutive, of social identities'', whereas in much of the linguistic anthropology work ``identity is better understood as an outcome of language use rather than as an analytic prime''~\citeyear[p.376]{Bucholtz2004}.

\section{The actuation problem}\index{actuation!actuation problem|textbf}

% define fundamental problem of accounts through quotes

``''~\citep[p.?]{Weinreich1968}


``How can speakers be masters of pronunciation change in one area but helpless victims in other cases?''~\citep[p.192]{Ohala1989}


While the definition of the actuation problem might be very old, it is still relevant.

2 problems with LCAS:
1. overpredicts change
2. explains the universal, not the particular
a) arbitrary conventions
b) social effects which only explain general tendencies but with many counterexamples

LCAS accounts and biases can explain the \emph{universal}, but not the particular: imagine space of all possible languages. Most changes happen in the middle region and are kind of pointless. The universal pressures are satisfied in this space, so we shouldn't expect there to be strong intra-linguistic pressures driving the selection of particular variants.

\citet{Nettle1999}'s \emph{threshold problem}

The need for agreement on arbitrary conventions can ``bring about a rigid conformity that cannot be broken by small shocks''~\citep{Bikhchandani1992} even more than in other domains of culture.

\section{Competing pressures and language as a Complex Adaptive System}
\label{sec:competingpressures}\index{competing motivations}\index{pressures!competing pressures}\index{complex adaptive system}

In the exposition of the different accounts of and explanations for language change so far, the pressures were presented as directly set up against each other. But, partly in order to cope with the problem of \emph{actuation}, scholars have already considered the interaction between different, \emph{competing} pressures.

% also within linguistics
\citeauthor{Dixon1997}'s idea of transferring \emph{punctuated equilibria} to language change~\citeyearpar{Dixon1997}.
\citep{Thomsen2006}


interacting pressures~\citep{Smith2013,Winter2016}


the same interaction between ``a tendency towards efficiency and a competing tendency toward redundancy'' was also proposed by~\citet[p.155]{Greenberg1954}~(printed in~\citealt{Osgood1954}).

\subsection{Functional pressures and the Iterated Learning Model}\label{sec:ilm}

% TODO move stuff from ch.6 here

\subsection{Adaptation to non-linguistic pressures}

With the advent of large-scale databases on languages such as the World Atlas of Language Structures~\citep[WALS][]{WALS2013}, large-scale statistical studies have identified correlational relationships between linguistic structures and non-universal external, non-linguistic factors~\citep{Ladd2015}.

external pressures, whether biological, geographical or social.


\citep{Dediu2007}
\citep{Nettle201?}

\citep{Atkinson2016}
% Languages evolve, adapting to pressures arising from their learning and use. As these pressures may be different in different sociocultural environments, non-linguistic factors relating to the group structure of the people who speak a language may influence features of the language itself.

% space of all possible languages -> `pointless' changes?

LCAS suggests complex grammar spaces beyond simple switching back and forth between two conventions such as word order patterns.
While there is evidence of some circular changes in language, e.g.~in the process of grammaticalisation, in the absence of a dedicated theory of why a pressure fails to kick in to drive the circle forwards for a certain period of time
arbitrary complexity

by external actuation such as \citet{Ritt2004} TODO or Baumann paper??

in a similar way to \citet{Yang20??} who explains the rise of periphrastic `do' in Early Modern English through the influence of non-native speaker productions on the learners' primary linguistic data.
While such accounts offer particular explanations for particular historical changes, they don't speak towards whether a \emph{general} theory of language change is possible.

All these accounts assume an internally stable, consistently acquired language whose stability is broken by external perturbation. But in the absence of any external triggers for the reorganisation of a linguistic system, what are the odds of `eternal optimisation' of a language~\citep{Boersma2003}?

\section{Language change and adaptation}\label{sec:adaptationism}\index{adaptation}

Only few quantitative linguists today would deny that language \emph{evolves}, i.e.~that it that it undergoes \emph{change by replication}. The question of whether language change is \emph{adaptive} on the other hand is far from settled, or rather it is assumed to be settled in different ways within different subfields studying language change. Interestingly, the fault line does not coincide with traditional 



The opinions held by both camps -- those who hope to gain insight from framing language change as \emph{optimisation} of some sort vs. those sceptical of or categorically opposed to such a viewpoint -- go back far.

The most well-known exponents of the latter point 

\begin{quote}
Taking linguistic change as a whole, there seems to be no discernible movement towards greater efficiency such as might be expected if in fact there were a continuous struggle in which superior linguistic innovations won out as a general rule. \citep[p.69]{Greenberg1959}
\end{quote}
Or, in a well-known statement on syntactic change, \citet{Postal1968} finds that
\begin{quote}
There is no more reason for languages to change than there is for automobiles to add fins one year and remove them the next, for jackets to have three buttons one year and two the next, etc.~(p.283)
\end{quote}

Syntax finds itself with an unexpected ally in phonetics, where \citet{Ohala1989} argues that
\begin{quote}
I do not think sound change creates any significant improvement or defect in language. There is sufficient redundancy in language that the message which speech encodes gets through as well (or no worse) before and after sound change. (p.191)
\end{quote}

\begin{quote}
``All human languages manifest asymmetry or disequilibrium in some part of their phonology but seem, nevertheless, to function adequately for communication. [\ldots] There is no need to appeal to ill-defined notions such as "system pressure", "pattern symmetry", "equilibrium", and the like, nor to maintain that the language is any better or any more "fit" as a result of these sounds changes. [\ldots] It seems languages are never satisfied! More to the point, it seems that languages are not seeking the satisfaction of some "ideal" configuration.'' (p.192)
\end{quote}

Particularly within the study of sound change, this point of view is now widely accepted and essentially become textbook knowledge. For example, the entire introductory chapter of \citet{Labov2001} is dedicated to what he calls the \emph{Darwinian paradox}~(p.10): \index{Darwinian paradox}
\begin{quote}
There is general agreement among 20th-century linguists that language does not show an evolutionary pattern in the sense of progressive adaptation to communicative needs. [\ldots] The almost universal view of linguists is the reverse: that the major agent of linguistic change -- sound change -- is actually maladaptive, in that it leads to the loss of the information that the original forms were designed to carry.
\end{quote}

\citet[p.39]{Croft2000}
\begin{quote}
the empirical evidence indicates that linguistic selection is governed largely if not exclusively by social forces that have little or nothing to do with functional adaptiveness for communication.
\end{quote}

The list of supports of the viewpoint that language change is (at least in some sense) adaptive is just as long and illustrious, from \citet{Jespersen1949} to \citet{Vennemann1993}.
Evidence from how functional load guides language change~\citep{Wedel2013short}


Especially within typology the discussion seems to have reached a conclusion in the opposite direction, e.g. Haspelmath, and~\citet{Wichmann2015} who finds that ``Presumably language change is only to some extent random, and to a larger extent is adaptive''~(p.221).

This approach is largely echoed by \citet{Croft2006}, who concludes that ``all of the empirical evidence in language change indicates that social factors, not functional ones, are the causal mechanisms for the propagation of a change''~(p.116).



The general pattern seems to be that researchers diverge in their conclusion depending on whether they focus on the macro-level of language change -- \emph{which changes tend to occur} -- versus the micro-level of individual changes -- \emph{how do individual historical changes unfold}. This apparent schism in research questions and conclusions will reappear throughout this thesis, which itself will be concerned with the micro-level dynamics of individual language changes.
I will argue that by taking an evolutionary approach, the two viewpoints can in fact be unified.
But first I will elaborate on the quantitative and evolutionary approach to language change that is taken by this thesis, which are relatively well understood thanks to centuries of quantitative research on historically attested changes.

\section{Language change: a quantitative framework}

Qualitative descriptions

Attempts to characterise different types of changes (lexical, morphosyntactic, phonological, phonetic, pragmatic) in one framework.

Variationist studies introduced the concept of the \emph{sociolinguistic variable}.

\subsection{The sociolinguistic variable}\index{sociolinguistic variable}
\label{sec:sociolinguisticvariable}

\citep[p.]{Tagliamonte2012}

Although language change can be studied on levels other than that of the sociolinguistic variable~\citep[see in particular][p.98]{Croft2006}, it is chiefly changes in the usage of sociolinguistic variables that will be covered in this thesis.

% subsection: s-shaped curves in language change
\input{introduction/scurves}

\subsection{Language change as language evolution}\label{sec:evolutionaryapproaches}

~(\citealt{Weinreich1968} would even attribute similar thinking to \citealt{Paul1920}).

\citet{Croft2000}'s evolutionary account of language change, based on \citeauthor{Hull1988}'s \emph{General Analysis of Selection}~\citeyearpar{Hull1988}, in which passive \emph{linguemes} are replicated by human \emph{interactors} engaging in communicative behaviour.\index{interactor}
This has parallels in \citet{Boyd1985}'s models of cultural evolution.

What is interesting under such an approach is that, quantitatively, many different kinds of pressures and accounts presented above fall together. Both functional and social \emph{prestige} pressures, for example, are biases towards specific linguistic variants, which means they are both \emph{content biases} under \citeauthor{Boyd1985}'s terminology. So the only way that they could be distinguished in practice is by testing quantitative predictions regarding the underlying origin or source of this bias.


\subsubsection{Random drift}\index{neutral evolution}
% more metaphorical: Sapir
% more mechanical: Bauer etc.
Previously not considered, now on the table as very mechanistical accounts. Importantly, this (originally biological) notion of drift is not to be confused with \citet{Sapir1921}'s more figurative idea of linguistic \emph{drift}, which I will return to in the final chapter.

\subsubsection{Beyond biological metaphors: regulatory pressures}

emphasis on not just transferring from biology to cultural studies

Evolutionary approaches and social learning

\citep{Altmann2013,Acerbi2014}

trends: \citep{Bikhchandani1992,Bikhchandani1998}


While the current work on the \emph{incrementation} of language changes by means of amplifying trend can be traced back to \citet[ch.?]{Labov2001}'s concept of \emph{age vectors}, the idea that the \emph{history} of linguistic systems beyond its current state can influence how individuals change the language can also be found in much earlier work, even if only very implicitly, or at second reading.

One concept that has stuck around despite confusion around its exact conceptualisation % meaning
is \citet{Sapir1921}'s notion of linguistic `drift'~(not go be confused with genetic drift, aka neutral evolution). \index{drift}

 languages do not just diverge when there is social pressure to discriminate, but also have an internal `drive' to change (in place) by themselves (p.160).

But Sapir also alluded to the fact that underlyingly related but in the meantime isolated languages tend to develop in the same way, or the same direction. 

The idea that temporal stratification of a language forms a fundamental fact of a grammar is not new, as for example stated by~\citet[p.42]{Fries1949}:
\begin{quote}
``It is impossible to give a purely synchronic description of a complex mixed system, at one point of time, which shows the pertinent facts of that system; direction of change is a pertinent characteristic of the system and must also be known if one wishes to have a complete description of the language as it is structurally constituted.''
\end{quote}

\citep{Bailey1970} on building `rates of change' into synchronic descriptions

Interestingly, it appears that the reductionist, synchronically-focussed stance promoted by generativist approaches to language largely eradicated such thinking.

In the evolutionary approach to language change inspired by work in population biology, one typically thinks about the diffusion of the incoming \emph{variant} which is, beyond its initially low frequency, not explicitly assumed to be in any way marked relative to the established majority one.
Particularly in the sociophonetic literature though, one finds evidence of the notion of the diffusion of a \emph{change}~(rather than a variant), possibly to be construed as an incoming variant that is defined \emph{relative to} an established one. This becomes particularly apparent in \citet{Labov2001}'s notion of \emph{age vectors} discussed in Section~\ref{sec:agevector}. \index{age vector}
That this notion survived or re-emerged in the study of sound change is maybe not surprising, since the continuous nature of phonetic changes lends itself to thinking about changes as incrementing relative to an average pronounciation. In this sense it is also possible for individual language users to produce even `more incoming' pronounciations by extending the age vector beyond its current level of advancement. This differs from the incrementation of categorical variants, where the only way for an individual to advance a change is by increasing the relative usage frequency of the incoming variant.

\section{Summary}

From the many different \emph{biases}, \emph{pressures} and \emph{explanations} for language change that have been put forward it becomes obvious that it is often unclear how exactly these \emph{pressures} are supposed to be understood quantitatively.

Ellucidating the respective roles of the different pressures is one of the goals of this thesis. The next chapter is dedicated to the role that computational modelling can play to ellucidate this, and we will return to this point in Chapter~\ref{ch:bigpicture}.
%The next two chapters are dedicated to the computational and empirical investigation of a relatively understudied regulatory pressure, namely the detection \emph{trends} in the usage rates of linguistic traits, and their amplification by individuals.
