\section{What is language change?}

Examples, a working definition

\section{Explaining language change}

\subsection{Early accounts}

\subsection{Language-internal accounts}

%It is not our intention to discard functional pressures completely of course, the issue we wanted to raise in the introduction section is whether functional factors should be construed as replication pressures which drive the selection of linguistic variants. Wedel's functional load result is a case in point: let's assume (I think this would be in the spirit of Wedel) that there is a general laziness/economy/blending bias that would lead all phonemes of a language to merge. Since the 26 minimal pair threshold is a probabilistic rather than a deterministic predictor, we should probably think of the influence of each individual minimal pair as contributing towards an overall functional selection bias against merging the phonemes, a 'distinction-maintaining' bias that is in direct competition with the blending one. Now between those two pressures there is a sweet spot somewhere around 26 minimal pairs where the two biases balance each other out, so that at this point the decision of whether to merge or not to merge could go either way. On the 'macro' level, where we just ask the binary yes/no question of whether the population adopts a merger or not, this might be a satisfying result, but in our opinion it fails to speak to the 'micro' level of individual behaviour: what happens at or around this critical point, is there some non-linear response across individuals where half of them exhibit a tendency to merge the phonemes and half don't? Or, assuming a more gradual effect of the bias, wouldn't we enter the regime of neutral evolution of variants within every individual where the entire population should (in a synchronised fashion) drift around between sometimes merging the phonemes, and sometimes not?

%This question of the 'sweet spot' at which universal functional pressures that have presumably always been there on the individual level without affecting the community language suddenly kick in and start an ordered directed change is what the 'actuation problem' is all about. It should be noted that the quasi-deterministic requirement for a model of language change as quoted by the reviewer that "would predict, from a description of a language state at some moment in time, the course of development which that language would undergo within a specified interval" is not what Weinreich et al. had in mind either, in the immediately following paragraph they reveal this bold goal to be nothing but a straw man, stating that "Our own view is that neither the strong nor the modest version of such theories of language change, as they proceed from current generative grammar, will have much relevance to the study of language history".

%However, in our view a full account of the pressures that drive language change *has* to make specific predictions at the level of individual behaviour, since this is what gives rise to the population-level behaviour and not the other way around. In other words, accounting for a macro-level pattern while completely underspecifying how exactly the probabilistic pressures (as observed from cross-linguistic samples like Wedel's) actually get expressed on an individual level during one particular change won't do. While macro-level accounts (like the results regarding functional load) are of course worthwhile and interesting it can be dangerous to equate them with explanations for individual events. What we are asking for here is not a deterministic account of whether a change occurs or not, we are still dealing with probabilistic models here, but models of individual behaviour that, when put into interaction with other individuals, yield predictions about population-level shifts alongside equally specific predictions about the underlying individual behaviours that give rise to the different population-level scenarios.

%In summary we never intended to be dismissive of functional factors, and we have now expanded the introduction section to discuss how functional factors play an important role within social accounts.

The levels of language change: Fedzechina graph

child-based language change~\citep[p.109]{Croft2006}

\subsubsection{Evolutionary approaches and language as a Complex Adaptive System}

Including references to general cultural evolution models

\subsection{Social accounts}\index{social accounts}

\subsubsection{Mechanical accounts}\index{interactor selection}

% mislearning:
%1. people 'fail' to meet their speech goals
%2. kids hear these errors and decide they're the real goal
%failing is more optimal:
%Todorov and Jordan 2002: "variability which is larger in task-irrelevant directions" in optimal feed-forward synergies
%+corollary 'minimal intervention principle' (2003, p.28)

\subsubsection{Prestige accounts}\index{prestige}



Whether social valuation exists prior to language use at all is an equally open question, as reflected in work on language and identity from which the sociolinguistic account of language change draws: Bucholtz and Hall note that ``much work within variationist sociolinguistics assumes not only that language use is distinctive at some level but that such practices are reflective, not constitutive, of social identities'', whereas in much of the linguistic anthropology work ``identity is better understood as an outcome of language use rather than as an analytic prime''~\citep[p.376]{Bucholtz2004}.

\subsection{Random drift}\index{neutral evolution}

% more metaphorical: Sapir
% more mechanical: Bauer etc.

\section{The actuation problem}\index{actuation!actuation problem|textbf}

% define fundamental problem of accounts through quotes

While the definition of the actuation problem might be very old, it is still relevant.

LCAS accounts and biases can explain the \emph{universal}, but not the particular: imagine space of all possible languages. Most changes happen in the middle region and are kind of pointless. The universal pressures are satisfied in this space, so we shouldn't expect there to be strong intra-linguistic pressures driving the selection of particular variants.

The need for agreement on arbitrary conventions can ``bring about a rigid conformity that cannot be broken by small shocks''~\citep{Bikhchandani1992} even more than in other domains of culture.

\section{Language change and adaptationism}\index{adaptation}

Only few quantitative researches today would deny that language \emph{evolves}, i.e.~it that it undergoes change \emph{by replication}. The question of whether language change is \emph{adaptive} on the other hand is far from settled, or rather it is assumed to be settled in different ways within different subfields studying language change.
% actually curious alliances: Kiparsky(?)196, Ohala1989
For example, the entire introductory chapter of \citet{Labov2001} is dedicated to what he calls the \emph{Darwinian paradox}~(p.10): \index{Darwinian paradox}
\begin{quotation}
There is general agreement among 20th-century linguists that language does not show an evolutionary pattern in the sense of progressive adaptation to communicative needs. [\ldots] The almost universal view of linguists is the reverse: that the major agent of linguistic change -- sound change -- is actually maladaptive, in that it leads to the loss of the information that the original forms were designed to carry.
\end{quotation}

This approach is largely echoed by \citet{Croft2006}, who concludes that ``all of the empirical evidence in language change indicates that social factors, not functional ones, are the causal mechanisms for the propagation of a change''~(p.116).

Especially within typology the discussion seems to have reached a conclusion in the opposite direction, e.g. Haspelmath, and~\citet{Wichmann2015} who finds that ``Presumably language change is only to some extent random, and to a larger extent is adaptive''~(p.221).

The general pattern seems to be that researchers diverge in their conclusion depending on whether they focus on the macro-level of language change -- \emph{which changes tend to occur} -- versus the micro-level of individual changes -- \emph{how do individual historical changes unfold}. This schism in research questions and conclusions will reappear throughout the thesis, in particular in reference to the \emph{two rates of language change} discussed in Section~\ref{sec:tworates}. For now we will be concerned with the micro-level dynamics of individual language changes, which are relatively well understood thanks to centuries of quantitative research on historically attested changes.

\section{Language change: a quantitative framework}

\subsection{The sociolinguistic variable}\index{sociolinguistic variable}
\label{sec:sociolinguisticvariable}

\citep[p.]{Tagliamonte2012}

Although language change can be studied on levels other than that of the sociolinguistic variable~\citep[see in particular][p.98]{Croft2006}, it is chiefly changes in the usage of sociolinguistic variables that will be covered in this thesis.

% subsection: s-shaped curves in language change
\input{introduction/scurves}

\section{Beyond biological metaphors: meta-regulatory pressures}

Evolutionary approaches and social learning



\citep{Altmann2013,Acerbi2014}

trends: \citep{Bikhchandani1992,Bikhchandani1998}

\section{Summary}

From the many different \emph{biases}, \emph{pressures} and \emph{explanations} for language change that have been put forward it becomes obvious that it is often unclear how exactly these \emph{biases} are supposed to be understood quantitatively.

Ellucidating the respective roles of the different pressures is one of the goals of this thesis, and we will return to this point in Chapter~\ref{ch:bigpicture}. The next two chapters are dedicated to the computational and empirical investigation of a relatively understudied regulatory pressure, namely the detection \emph{trends} in the usage rates of linguistic traits, and their amplification by individuals.
