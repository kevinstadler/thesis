Human languages are not static entities. Linguistic conventions, whose social and communicative meaning are understood by all members of a speech community, are gradually altered or replaced, whether by changing their forms, meanings, or by the loss of or introduction of altogether new distinctions. How do large speech communities go about re-negotiating arbitrary associations in the absence of centralised coordination?
%And \emph{why} would a community even want to change working conventions that are understood by everyone?

This thesis first provides an overview of the plethora of explanations that have been given for language change. %, with a particular focus on different sources of \emph{pressures} that have been proposed to influence or drive changes.
Approaching language change in a quantitative and evolutionary framework, mathematical and computational modelling is put forward as a tool to investigate and compare these different accounts and their purported underlying mechanisms in a rigorous fashion.

The central part of the thesis investigates a relatively recent addition to the pool of mechanisms that have been proposed to influence language change: I will compare previous accounts with a \emph{momentum-based selection} account of language change, a replicator-neutral model where the popularity of a variant is modulated by its \emph{momentum}, i.e. its \emph{change in frequency of use} in the recent past. I will discuss results from a multi-agent model which show that the dynamics of a trend-amplifying mechanism like this are characteristic of language change, in particular by exhibiting spontaneously generated s-shaped transitions. I will also discuss several empirical predictions made by a momentum-based selection account which contrast with those that can be derived from other accounts of language change.

Going beyond theoretical arguments for the role of trends in language change, I will go on to present fieldwork data of speakers' awareness of ongoing syntactic changes in the Shetland dialect of Scots. Data collected using a novel questionnaire methodology show that individuals possess explicit knowledge about the direction as well as current progression of ongoing changes, even for grammatical structures which are very low in frequency. These results complement previous experimental evidence which showed that individuals both possess and make use of implicit knowledge about age-dependent usage differences during ongoing sound changes.%, and that such knowledge is employed during phonetic and phonological processing.

%Finally, I will situate this novel mechanism relative to previously proposed accounts of language change.
%The final part of the thesis brings together the different pressures brought up in the literature, with a particular focus on the \emph{interaction} between different types of pressures.
Echoing the literature on evolutionary approaches to language change, the final part of the thesis stresses the importance of explicitly situating different pressures either in the domain of the \emph{innovation} of new or else the \emph{selection} of existing variants. Based on a modification of the Wright-Fisher model from population genetics,
%as a simple tool for demonstration
I will argue that trend-amplification selection mechanisms provide predictions that neatly match empirical facts, both in terms of the diachronic dynamics of language change, as well as in terms of the synchronic distribution of linguistic traits that we find in the world.

%Humans can replicate linguistic conventions to a high degree of fidelity, sometimes established conventions get replaced by new variants, with the gradual replacement following the trajectory of an \emph{s-shaped curve}. Although previous modelling work suggests that only a bias favouring the replication of new linguistic variants can reliably reproduce the dynamics observed in language change, the source of this bias is still debated.
