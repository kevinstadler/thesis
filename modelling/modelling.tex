\section{Why model?}

Within the field of language evolution (here meant to encompass both research on the evolution of the human language capacity, as well as modelling of the type of cultural, linguistic changes described above), computational models have undergone pronounced trends. % hype phase
During the hayday of computational modelling, the field first produced a plethora of qualitative and quantitative models of the evolution of the language faculty~\citep{Kirby1999,Nowak2001a} as well as the emergence of linguistic conventions, such as the `Naming Game'~\citep{Baronchelli2008}. Around the same time, similar methodologies became popular to study the dynamics of language \emph{change}, i.e.~the replacement of already established conventions, both in general~\citep{Niyogi1995,Niyogi1997,Arita1998,Nettle1999,Kataoka2000,Livingstone2000,Ritt2004,DeOliveira2005,Niyogi2006,Wedel2006,Baxter2006,Wedel2007,Ettlinger2007b,Ettlinger2007,Fagyal2010,Blythe2012,Gong2012,Otero-Espinar2013,Soskuthy2013,Pierrehumbert2014,Enke2016,Kauhanen2015} as well as for some specific historical changes in particular~\citep{Yang2002,Choudhury2006,Choudhury2007,Pearl2007,Troutman2008,Baxter2009,Sonderegger2010,Swarup2012,Ritt2012,Kirby2013,Kirby2013cogsci}.
% Vazquez-Larruscain2006 (nedergard-thomsen book)
% Kauhanen2016
% review: Kroch2005

In terms of external contributions to the study of language change, mathematicians and physicists in particular have brought the formal tools from their own domains to bear on questions of interest to linguists~\citep{Castellano2009,Blythe2015}. Given the complexity of the methods involved, such contributions often fail to have a lasting impact on thinking in the field if they do not form part of a broader linguistically-motivated research programme which makes it accessible to linguists. % Since one such research programme will be presented in-depth in Chapter~\ref{ch:modelling} which also discusses the role of modelling more generally, I will limit myself here to an overview of conceptually motivated proposals from within the field of language change~(as well as cultural change more generally).
Given the relatively abrupt rise of this new methodology, it is not surprising that this hype was followed by several meta-scientific and review papers advocating and/or defending the use of computational models~\citep{Cangelosi2002,Wang2004,DeBoer2006,Baker2008,Jaeger2009,Hruschka2009,Vogt2010,DeBoer2012EvoLang,Smith2016}.
Since most parts of this thesis are going to be concerned with computational modelling, it is worth asking: what \emph{is} the point of having a computational model?

The primary advantage of a formal model is that it allows~(or rather forces) one to step away from pure arm-chair theorising, which can be difficult to treacherous when applied to complex phenomena such as languages, which are affected by the interplay of many interacting parts or language users. Instead of guessing at the effects of micro-level assumptions on the macro-level dynamics of the system, a computational approach forces the researcher to explicitly lay out their assumptions about the individual, interacting parts in a quantitatively measurable (and ideally also well-motivated) way. From there, computational methods take the lead by determining in an objective way how the transparent assumptions about individual behaviour culminate in (potentially) complex interactional phenomena in the population.

Probably the earliest example of computational work on the emergence of communication systems is the so-called \emph{Naming Game}~\citep{Steels1995,Steels1998naminggame,Baronchelli2006}.\index{Naming Game}
In its simplest form a population of agents, each starting off with an empty lexicon, has to come to agree on a unique `name'~(linguistic form) for a referent or speech act. In-depth study of the Naming Game and its dynamics showed not only how a population could come to agree on a shared convention in the absence of any centralised coordination, it also helped shed light on the types of linguistic preferences or mechanisms that individual agents should have to enable de-centralised coordination to unfold seamlessly~\citep{Wellens2012,Spike2016}. 
Crucially, given the very simple problem description of the minimal version of the Naming Game which lacks any risk of referential ambiguity, its outcome is predicted \citep[and in some cases even proven][]{DeVylder2006,Skyrms2010} to be the emergence of a stable communication system. In the absence of any noise or stochasticity, this simple model does not exhibit continuous, ongoing change that is so characteristic of human language. %that would make it a complex adaptive system par excellence. %Rather, early investigations like the naming game provided a simple baseline dynamic on which further research could be built.

% Bloomfield1933 on density, followed by Stanford2013

Going beyond the initial emergence of a symbolic communication system and closer to more realistic cases of language change under noisy transmission conditions,
\citet{Wedel2004,Wedel2006} offered a computational investigation of how simple mechanisms of replication can lead to, amongst other things, phonological category formation as well as contrast maintenance through change in the phonetic dimension. Inspired by models of evolutionary pressures taken from biology, the models were again chiefly a study of general mechanisms from which the aforementioned universal dynamics of language organisation emerged. When idiosyncratic factors or triggers of particular changes were concerned, such as in the study of contrast maintenance under the threat of contrast loss in~\citet{Wedel2006}, the sudden onset of contrast loss in one dimension is again applied externally, falling outside the scope of the theory of general, universal mechanisms that forms the basis of demonstrably adaptive changes whose actuation is reactive to an external trigger.

While computational modelling has also become a standard technique in related empirical fields such cognitive science, the models of individual behaviour that are of interest to psychologists are necessarily of a very different character than the multi-agent models typically employed to study language as a distributed population-level phenomenon.
The differences between the fields are not just limited to the \emph{types} of models though, they also extend to the exact goals of modelling and consequently to how models are evaluated.
While cognitive models are often assessed on a quantitative basis~\citep{Busemeyer2010}, many models of language evolution and change have retained a \emph{proof of concept} like character. This tradition harks back to original work on the evolution of shared communication systems~\citep{Steels1995,Kirby2000} where models are primarily judged based on their exhibiting some \emph{qualitative} feature such as compositionality, rather than by quantitative comparison to other models or to empirical data.
This development can be attributed to the often close~\citep[and sometimes confusing;][]{Haspelmath2016} interlacing of questions regarding the evolution of \emph{Language}, in the sense of the language \emph{capacity}, the emergence of universal features of languages~(such as duality of patterning) by cultural evolution, as well as evolutionary approaches to `mere' language \emph{change}, possibly in combination with the lack of established corpora of historical changes mentioned previously.

The promise that explicitly spelling out the quantitative assumptions of different models would bring clarity to the field is consequently not as straightforward as it might seem. Depending on the precise framing of the same phenomenon, such as the establishment of a shared communication system in the case of the Naming Game, or the emergence of compositional language from repeated interactions, superficially different mechanisms which actually have very similar effects on a behavioural level can be considered competing explanations for years~\citep{Wellens2012,Spike2016}.
Especially when models are explicitly dedicated to comparing the effect of different parameter settings \emph{within} them, the general dynamics (such as the basic learning rules or other parameters like population turnover employed in virtually every social learning model) are often taken for granted, although it is important to note that much of the dynamics are implicit in these basic assumptions themselves.
Without a dedicated effort to replicate existing models and bring them in direct relation to each other, even computational models can run risk of becoming ideological `blackboxes', counter to their original intention to make underlying assumptions more transparent.
The rhetoric with which computational models are presented can further aggravate this situation. Particularly when it is of interest to make computational models more appealing and convincing to the non-modeller, as is the case when the methodology first spreads to a new field, efforts to portray models as a tool for revelation and enlightenment (rather than as obfuscating black magic) run the risk of trivialising either the models themselves, or at least the analyses presented. %~(the latter point will be made clear with an example in Section~\ref{sec:realigriffiths})
%Computational models thus allow us to both directly compare the expected macro-level behaviour of an interacting system under different micro-level assumptions, but also to explicitly test different parameter conditions of the same model to determine the respective influence of different biases or model parameters.

The computational models which have stood the test of time are therefore those which are not one-offs, such as many of the early models which employed bespoke ad-hoc learning rules that are often not grounded in the general learning literature, but models which have undergone intense study and analysis from the ground up. For the case of the evolution of novel inventions, the Naming Game is a case in point~\citep{Baronchelli2008}. For the case of language \emph{change}, i.e.~the continuous replacement of established conventions, probably the most extensive and well-explored model is the Utterance Selection Model~(USM) of language change, which forms the basis of most of the modelling work in this thesis.

%It is with these words of caution in mind that we dive into an in-depth analysis of the (very) basic dynamics of two models

\section{The Utterance Selection Model}\index{Utterance Selection Model}
\label{sec:usm}

%\section{Comparing accounts using the Utterance Selection Model}
%Introduce USM generally, including EWMA-formulation and general learning/alignment properties

The version of the Utterance Selection Model~(USM) discussed here grew out of \citeauthor{Croft2000}'s more general formulation of language change as evolutionary competition between utterances. While in its original, theoretical formulation in~\citet{Croft2000} it is truly full \emph{utterances} which are undergoing replication, in its mathematical-computational incarnation the USM is best understood as a quantitative model of the competition between different variants of one sociolinguistic variable, as described in Section~\ref{sec:sociolinguisticvariable}.\index{sociolinguistic variable}

At its core, every agent in the USM is completely characterised by its variable use over variants, specified by the proportions with which each variant is used, all of which together sum to~1. For sake of simplicity we will limit ourselves to the canonical case of two competing variants, where the behaviour of an agent~$i$ can be captured by a single variable~$0<x_i<1$ representing its relative usage level of the incoming variant, with that of the competing variant taken to be~$1-x_i$.

The primary contribution of the computational USM is that it provides a well-defined and rich framework to study the dynamics of these internal usage levels as they are influenced by observing realisations of the same linguistic variable in interactions with other speakers in a population. % TODO The USM 

\subsection{Model parameters of the USM}

\subsubsection{Learning rate $\lambda$}

Following an interaction, the agents update their internal frequency according to the following USM update rule, which is again applied for both agents~\citep[p.4]{Baxter2006}:
%\cite[p.4]{Baxter2006}:
\begin{equation}\label{eq:usm1}
x_i' = \frac{x_i+\lambda\cdot y_i}{1+\lambda}\;,
\end{equation}
where $y_i$ is the subjective \emph{perceived frequency} of the variable usage rate, whose computation will be discussed below.

%$$x_i' = \frac{x_i+\lambda((1-H_{ij})n_i+H_{ij}n_j)}{1+\lambda} = \frac{x_i}{1+\lambda} + \frac{\lambda(\dots)}{1+\lambda}$$% H_{ij} \in [0,1]$$
%$$H_{ij} = \lambda h$$

Perhaps the most important model parameter is the agents' learning rate~$\lambda$, which is by default assumed to be the same for all agents. What the USM's update rule in Equation~\ref{eq:usm1} does is change an agent's internal frequency~$x_i$ by shifting it a small step towards the relative perceived frequency that it observed in its most recent interaction.
The higher the learning rate, the larger the step towards this target frequency: at~$\lambda=0$ there is no learning and the agent remains at their initial frequency forever, as~$\lambda\rightarrow\infty$, the agent approaches a regime in which they instantly adopt exactly those usage frequencies observed in their last interaction.
While there are instantiations of the USM in which the learning rate for individual agents is not constant but \emph{decreases} over time to imitate the effect of increasing rigidity of language use with age~\citep{Baxter2016}, this thesis will be concerned with the simpler case of a constant learning rate that is identical for all agents in the population.
Since we are mostly interested in reliable model behaviour that exhibits gradual assimilation rather than abrupt and erratic changes in individual usage levels, like most investigations of the USM we will limit ourselves to low values in the range of~$\lambda\le0.01$).

It should be acknowledged that the particular form of the learning rule was partly chosen due to its mathematical properties, which make it amenable to analysis using tools from statistical physics~\citep[see in particular][]{Baxter2006}. To get a more intuitive understanding of what the update rule does in terms of agents' learning dynamics, it is worth noting that it is equivalent to defining an agent's usage levels as an exponentially weighted moving average~(EWMA) over its learning input data series of perceived frequencies~$\vec{y}$.
EWMAs themselves are a generalisation of Bush-Mosteller learning~\citep{Bush1955} for non-discrete input data points which, rather than employing a fixed time window to average over, always gives relatively more weight to the most recent data points, with the absolute contributions of individual learning samples decaying over time.
Upon receiving a new data point~$y$ indicating a certain usage level observed in an interaction, the agent updates their own usage level~$x$ according to

\begin{equation}
x' = (1-\alpha)\cdot x + \alpha\cdot y\;.
\end{equation}

This representation of the learning rule makes it clear that the agent's own usage level is simply a moving overage over the perceived frequencies it observes in interactions, where $\alpha$ controls the relative weight of the newest data point toward that moving average.
%At time $t$ datum $y_{t-i}$ has weight $\alpha(1-\alpha)^{i-1}$,
This formulation is equivalent to the original USM updating rule in Equation~\ref{eq:usm1} given

\begin{equation}
\alpha = \frac{\lambda}{1+\lambda}
\end{equation}
\begin{equation}
\lambda = \frac{\alpha}{1-\alpha}\;,
\end{equation}
the only difference being a rescaling of the parameter space from $\lambda\in[0,\infty)$ to $\alpha\in[0,1]$, as shown in Figure~\ref{fig:mapping}.



\begin{figure}[htbp]

{\centering \includegraphics[width=\maxwidth]{figure/mapping-1} 

}

\caption[Mapping between the two USM learning rate parameter spaces]{Mapping between the $\alpha$ and $\lambda$ parameter spaces, $\alpha = \frac{\lambda}{1+\lambda}$ or $\lambda = \frac{\alpha}{1-\alpha}$, respectively. $\lambda = 0$ corresponds to $\alpha = 0$, $\lambda = 1$ to $\alpha = 0.5$, and $\alpha = 1$ to the limit of $\lambda\rightarrow\infty$.}\label{fig:mapping}
\end{figure}



The USM's dynamics beyond the simple update rule are controlled by a number of other parameters which will be briefly introduced here, before their individual effects are explained in more detail in the following Sections.
Firstly, at every point in time a new pair of distinct agents~$i, j$ has to be chosen from the population, which consists of a fixed number of~$N$ agents total. Interacting agents are randomly drawn based on a matrix~$G$ which specifies the probabilities of interacting for all pairs of agents. 
Whenever an agent~$i$ with an internal frequency of use~$x_i$ is chosen to engage in an interaction with another speaker~$j$, they each produce and exchange~$T$ tokens of the variable under investigation by taking a sample from the corresponding Binomial distributions~$Bin(T, x_i)$ and~$Bin(T, x_j)$ respectively. Based on the samples~$n_i$ and~$n_j$ taken from each of the distributions, the agents combine the relative frequencies~$\frac{n_i}{T}$ and~$\frac{n_j}{T}$ into \emph{perceived frequencies}~$y_i, y_j$ according to the following formula:

\begin{equation}\label{eq:perceivedfrequency}
y_i = (1-H_{ij})\cdot f(\frac{n_i}{T}) + H_{ij}\cdot f(\frac{n_j}{T})\;.
\end{equation}

In other words, the perceived frequency is based on a weighted sum of the agent's own productions and that of their interlocutors, and is calculated separately for the other agent~$j$ by exchanging all the indices~$i, j$.
Here the high degree of modularity of the model becomes evident in the number of parameters, only some of which will be of interest to us here, but which it is worth going through in turn.

\subsubsection{Population size $N$}\index{production-perception loop}

Like virtually all models of language change, the USM is a \emph{multi}-agent model, i.e.~it simulates a \emph{population} of agents that engages in interactions. While a dynamic population with changing population size would be possible, most investigations are limited to assuming a fixed number of agents~$N$ that remain in the population the entire time~\citep[again see][for an exception]{Baxter2016}. This simplifying assumption lends the USM to more general analysis and enables to connect it to evolutionary models from other domains. In particular, \citet{Blythe2007divided} showed the USM's equivalence to \citeauthor{Wright1931}'s island model~\citeyearpar{Wright1931}, where the population size~$N$ corresponds to the number of biological subpopulations or `islands' between which only limited exchange of replicators takes place. The effect of different values of~$N$ on the dynamics of the USM depend on several of the other model parameters, and will be explored in more detail below.

\subsubsection{Social network structure/interaction probability matrix $G$}

The parameter~$G$ is a square matrix of size~$N\times N$ which specifies the probabilities for every pair of agents to be chosen to interact with each other, so that the sum over \emph{distinct} pairs $\sum_{\langle i,j\rangle}G_{ij}=1$. This parameter can not just gradually alter the frequency or density of interactions between different agents or agent groups. By setting a specific~$G_{ij}=0$ one can completely `disconnect' two agents $i,j$ in the interaction network, thereby creating the same effect that \emph{social network structure} has in many other multi-agent models of language change.
As I discussed above, the exact role that networks of social interactions have on the diffusion of language changes is still debated, with equally conflicting results over whether network structure matters fundamentally~\citep{Blythe2007divided,Fagyal2010,Gong2012,Pierrehumbert2014,Kauhanen2016} or only marginally~\citep{Nettle1999,Baxter2008,Blythe2009,Stadler2009}, with the results obtained from computational models again largely dependent on many other underlying assumptions and the particular learning models used.

Since this thesis will not investigate the effect of either network structures or nonuniform interaction probabilities, we will abdicate the many degrees of freedom bestowed by the this parameter matrix by always assuming a fully connected network of~$N$ agents with equal interaction probabilities, setting $G_{ij}=\frac{1}{N-1}$ for all~$i\ne j$.

\subsubsection{Accommodation/alignment matrix $H$}

The parameter~$H$ in Equation~\ref{eq:perceivedfrequency} above is a square matrix which specifies the weights that all ordered pairs of individual agents give to each others' productions, with $H_{ij} \in [0,1]$. At the extreme of $H_{ij}=0$, agent~$i$ completely discards any input it receives from agent~$j$ and its perceived frequency~$y_i$ is consequently completely determined by its own productions.
A value of $H_{ij}=0.5$ would give equal weight to both the speaker's and the listener's production in an interaction. By employing different values in the cells of $H$~(in particular by setting pairs of agents' mutual accommodation parameters $H_{ij}, H_{ji}$ unequal), the matrix can be used to model asymmetries in adoption structures in a population, as well as increased influence of some individuals' usage levels as a form of individual~(rather than variant) prestige, a mechanism that will be explored below.

Beyond using~$H$ to introduce individual differences, it is also possible to set uniform accommodation behaviour by setting all matrix values $H_{ij}$ to the same fixed constant~$h\in[0,1]$. The degree of accommodation only affects the USM's dynamics when there are systematic differences in usage levels within the population, which could be due to inter-individual differences such as age-stratified populations or differing variant selection biases~\citep{Baxter2016} or otherwise due to clusters or differences in the degree of connectivity in a social network, cases which have only seen limited investigation so far~\citep{Blythe2007divided,Michaud2017}.
Since this thesis will not be concerned with inter-individual differences or stratified network structures, all simulations will be performed so that agents are set to only align with their interlocutor and not to their own productions, equivalent to~$h=H_{ij}=1$ for all~$i,j$.

\subsubsection{Production sample resolution $T$}

$T$, a positive integer, is the aforementioned sample size which determines the `resolution' with which agents can observe the variable use of different variants of an agent with usage rate $x$ in an interaction by randomly sampling from a binomial distribution~$Bin(T,x)$. For sake of simplicity we will only be concerned with the case of two competing variants, but the definition generalises to~$k\ge3$ variants in which samples are taken from the multinomial distribution~$Mult(T,\vec{x})$, where an agent's usage probabilities over the $k$~variants are specified by a vector~$\vec{x}$ of length~$k-1$.

The parameter $T$ is rather unusual, in the sense that no comparable parameter features in most other computational models of language change. Among the many models referenced above, most can be assigned to one of two groups based on when and how learning, in the sense of inferring or updating a linguistic property or system from data, occurs. One group, in which agents remain in the population and learning occurs \emph{incrementally}, agents typically receive one data point at a time for each learning event they are sampled to partake in, for example in the Naming Game. In the other group of models there is explicit reference to a \emph{sample size} of the learning data, but there is typically only one learning event at the beginning of an agent's lifetime, such as in the case of the Iterated Learning model. %Moreover, in this case the sample size often refers to 
A combination of both, multiple learning events throughout an agent's life time each of which with a learning sample of more than one data point, is not normally considered, but turns out to be a crucial aspect of an evolutionary model of language change as described above. This has to do with the fact that the continued differential \emph{selection} of linguistic variants relies on the existence of variation in the population, variation which can only be attested in learning samples of sizes~$T>1$. This point will become more apparent when we discuss the core parameter that determines how agents derive the \emph{perceived} usage frequencies of variants in interactions, through the bias function~$f(.)$.

\subsubsection{Bias function $f(.)$}

While the numerical parameters so far all control some aspect of the population or interactions, the bias function~$f(.)$ is where \emph{selection} of specific variants comes into play. Its role is to alter the \emph{objective} relative frequency of tokens produced in the interaction, $\frac{n}{T}$, to an agent's subjective \emph{perceived} frequency.
$f(.)$ is simply a function that maps from the frequency interval~$(0,1)$ to~$(0,1)$. While in principle any arbitrary function could be plugged in here, most analyses are limited to mappings that obey some reasonable criteria, in particular that they are \emph{monotonically increasing} within the interval, so that relatively higher objective frequencies are always mapped to higher (or equal) perceived frequencies~\citep{Blythe2012}.

Special attention should be drawn to the fact that the bias function is only defined for~$(0,1)$ and that, per definition, $f(0)=0$ and $f(1)=1$. These two equivalences are imposed because~$f(.)$ embodies the \emph{differential selection} mechanism of the USM. While the bias function can alter the variation observed in individual samples in one way or another, the function must not indicate the presence of a variant when it is not attested in the sample. More generally, this constraint also stops the bias function from introducing new variants into the population, and thus a strict requirement for any evolutionary model that distinguishes the selection of existing variants from pressures of innovation through altered replication.

While the USM's original definition in~\citet{Baxter2006} also incorporated parameters for the spontanous generation of unattested variants, most studies of the model so far have been concerned with the analysis of the diffusion and selection of traits that are already established at a low level across the population. With the exception of the final chapter, this thesis will also primarily be concerned with \emph{selection} mechanisms, of which many different ones can be implemented through the function~$f(.)$.

%This function can be used to implement a mechanism of \emph{regularisation}, discussed in Section~\ref{sec:usmregularisation}, as well as \emph{replicator selection} mechanisms which prefer some of the competing variants over others, a case which will be discussed in Section~\ref{sec:usmreplicator}.
%and different functions for this parameter will be the main subject in the remaining sections.

%[Comparing accounts with the USM]
\section{Comparing accounts with the Utterance Selection Model}

Having covered the general mechanism of the USM, we can now investigate the predicted dynamics under the presence (or absence) of different biases. This section recapitulates the in-depth study of several different USM biases by~\citet{Blythe2012} while contributing an additional model of asymmetric replicator selection in Subsection~\ref{sec:usmreplicator}. The motivation for the present analysis is to address the question of which accounts or presumed pressures would predict s-shaped transitions of variant use (and under which conditions) when compared in one unified framework, which necessarily also includes a detailed study of the model's baseline behaviour in the absence of any pressures.

%Based on the general formulation of the utterance selection model presented above, \citet{Blythe2012} set out to address ongoing discussions on different social accounts of change to see whether the `mechanical view' and prestige 
\subsection{Neutral evolution}\index{neutral evolution}
\label{sec:usmneutral}

%Neutral evolution and the dynamics of the USM's minimal assumptions: interactions of $\lambda$, $x$, and new input $n$

While the USM's updating rule given in Equation~\ref{eq:usm1} is very general and allows for a vast number of modifications through the additional parameters, it is interesting to analyse the model's learning dynamics in the absence of any pressures of either innovation or differential replication.
This \emph{neutral evolution} condition, so-called because it is based on completely neutral replication of existing traits from the population according to their current prevalence~\citep{Blythe2012neutral}, is achieved by using the identity function
\begin{equation}
f(u) = u
\end{equation}
as the USM's bias function, which means that the agents' \emph{perceived} frequency~$y$ in an interaction can be directly derived from their interlocutor's productions, i.e.
\begin{equation}
y_i = \frac{n_j}{T}\;.
\end{equation}

Using this simple assumption, we can investigate when the agents' internal~$x$~value changes most. Since a lot of the dynamics stem from the basic learning rules, the exact roles of the basic parameters and their behaviour at different moments in the model should be studied in detail.
%For sake of thoroughness and to better understand the baseline dynamics of the updating rule, we can investigate the interaction between the agent's internal usage level~$x$ and how it changes after an interaction.
%Since a lot of the dynamics stem from the basic learning rules, the exact roles of the basic parameters and their behaviour at different moments in the model should be studied in detail.
Firstly, Figure~\ref{fig:singledxpern} shows the point change away from a internal usage proportion~$x=0$ for different input data points~$y$ as a function of the agent's learning rate~(plots are provided for both the $\alpha$ as well as the~$\lambda$ formulation of the learning rate).
The equal spacing between the curves for different~$y$ means that the impact of different input data points is proportional to their difference to the agent's internal value~$x$.

\begin{figure}[htbp]

{\centering \includegraphics[width=\maxwidth]{figure/singledxpern-1} 

}

\caption[Absolute point change to the agent's usage rate for different learning rates with the same initial value ]{Absolute point change to the agent's usage rate for different learning rates with the same initial value $x=0$ given different input datapoints~$y$. Left: absolute point change as a function of the learning rate~$\alpha$. Given an x value at one extreme and input data at the other, the maximum change to x is equal to $\alpha$. Right: absolute point change as a function of the learning rate~$\lambda$.}\label{fig:singledxpern}
\end{figure}



\begin{figure}[htbp]

{\centering \includegraphics[width=\maxwidth]{figure/singledxperx-1} 

}

\caption[Absolute changes for the same input data ]{Absolute changes for the same input data $y=1$ for different values of $x$ given a range of learning rates $\alpha$~(left) and $\lambda$~(right). Given an x value of $0$ at one extreme and input data $y=1$ at the other, the maximum change to x is equal to $\alpha$.}\label{fig:singledxperx}
\end{figure}



In fact, an identical picture emerges in the case of a fixed input data point~$y=1$ that is incorporated into different internal values~$x$, as shown in Figure~\ref{fig:singledxperx}. %Moreover, the straight lines What this tells us is that there is no interaction at all between the learning rate $\alpha$ and input data $y$, both of which have a strictly linear effect on the outcome of individual learning events given the same starting frequency~$x$.
Generally, given our EWMA update rule we find that
\begin{equation}\label{eq:usmdiff}
\Delta x = x'-x = \alpha\cdot n + (1-\alpha)\cdot x - x = \alpha\cdot(n-x)\;,
\end{equation}
i.e.~the point change to~$x$ is always directly proportional to the difference between the agent's current usage level~$x$ and the input data~$y$.
The USM's individual agent update dynamics therefore follow a general learning framework that is free from nonlinearities and which, in the absence of any biasing, has been shown to be equivalent to several other models of neutral evolution in biology~\citep{Blythe2007divided}, and whose dynamics are not significantly affected by setting the neutrally copying agents in many different types of structured networks~\citep[but see \citealt{Kauhanen2016}]{Blythe2010,Blythe2012copying}.
%, but how are the individual interactions and their resulting perceived frequencies~$y_i$ and~$y_j$ determined?

Due to the complete lack of asymmetries in the neutral evolution condition its underlying mechanism, often referred to as ``random copying'', has scarcely been proposed to be the underlying force of language change, or even cultural change more generally~\citep{Mesoudi2009}.
Upon numerical inspection, neutral evolution exhibits its characteristic dynamics which include ``large fluctuations and a tendency for an upward or downward trend to reverse one or more times before an innovative variant goes extinct or wins out''~\citep[p.285]{Blythe2012}.
\citet{Blythe2012neutral} in particular argues that neutral evolution should in fact be taken as a \emph{null model} against which competing accounts of language change should be compared, as a baseline similar to those underlying the neutral theory of molecular evolution in biology~\citep{Kimura1983}. Apart from using it as such a null condition for the model presented in Chapter~\ref{ch:momentummodel}, I will also return to an in-depth study of the dynamics of neutral evolution in Section~\ref{sec:realigriffiths}. But for now we will focus on replication mechanisms in the USM that actually implement \emph{selection} of some kind.

\subsection{Replicator-neutral selection}\index{regularisation}\index{selection}\label{sec:usmregularisation}
While the term selection is often associated with a preference for particular variants, it really covers \emph{differential replication} of any kind, and can therefore also be used to implement \emph{symmetric} selection functions which are neutral regarding the different variants or replicators. One example is the case of \emph{frequency-dependent selection} biases which can be used to systematically favour variants not based any inherent \emph{a priori} property but based on their current attested frequency in the population, a feature which itself changes over time~\citep{Boyd1985}. As the simplest symmetric, frequency-dependent selection mechanism \citet{Blythe2012} suggest the function

\begin{equation}\label{eq:usmregularisation}
f(u) = u + a\cdot u\cdot (1-u)\cdot(2u-1)\;,
\end{equation}
where values of the parameter~$a>0$ lead to a \emph{boost} of variants whose relative frequency is already greater than~$50\%$, while settings of~$a<0$ implement selection in favour of any variants currently in the minority.
The two regimes of this non-linear selection function are displayed in Figure~\ref{fig:usmregularisation1}, giving a visual sense of the function's symmetry: mirroring the plots along both axes yields an identical curve, an indication that, mathematically, the function satisfies the symmetry criterion $f(u) = 1 - f(1-u)$~\citep{Blythe2012}.

\begin{figure}[htbp]

{\centering \subfloat[The frequency-dependent selection mapping function from Equation~\ref{eq:usmregularisation} against the baseline of neutral evolution, $f(u)=u$, indicated by the dotted line. \emph{(i)}~conformity copying with $a=1$: all input frequencies $u>.5$ are mapped to even higher perceived frequencies so that~$f(u)>u$. \emph{(ii)}~anti-conformity copying with $a=-.5$, leading to an effect in the opposite direction.\label{fig:usmregularisation1}]{\includegraphics[width=\maxwidth]{figure/usmregularisation-1} }
\subfloat[Mean expected change to an agent's usage level~$x$ for different values of~$T$, assuming learning rate~$\lambda=1$.\label{fig:usmregularisation2}]{\includegraphics[width=\maxwidth]{figure/usmregularisation-2} }

}

\caption{The dynamics of frequency-dependent selection as implemented in the utterance selection model through Equation~\ref{eq:usmregularisation}.}\label{fig:usmregularisation}
\end{figure}



What is crucial to understand about the USM is that these mapping functions~$f(.)$ affect the selection dynamics only somewhat indirectly. First of all, according to the definitions above the choice of resolution parameter~$T$ constrains the points in the $[0,1]$ range at which~$f(.)$ is actually ever evaluated, namely only at the fractions which can be sampled from the underlying binomial distribution, i.e.~$\{\frac{n}{T}\;|\;n=0\dots T\}$.
%These intermediate plots~(call them~$g_T(u)$) are crucial in understanding the analytical solutions that can be derived for the trajectories for different $f(u)$. The mathematical appendix to~\citet{Blythe2012} lays out how we can derive these solutions: given a function of the general form $f(u)=u+\lambda g(u)$
%The mean change of speaker frequency in one interaction and the average overall trajectory is simply dependent on the average of the function $g(u)$ over the distribution of token productions for a given current population mean~$x$. Assuming that tokens are produced independently, this is a binomial distribution with~$n=T$ and~$p=x$.
Particularly at low values, $T$~can therefore have a drastic impact on the dynamics. To visualise the effect of this parameter, we can determine the \emph{typical change} to an agent's usage rate~$x$ by a specific selection function~$f(.)$ for different values of~$x, T$. To do so we first calculate the mean perceived frequency~$\bar{y}$ over all possible sample outcomes~$n\sim Bin(T,x)$,
\begin{equation}\label{eq:meanperceivedfrequency}
\bar{y} = \sum_{n=0}^T P(n;T,x)\cdot f(\frac{n}{T})\;.
\end{equation}

Using Equation~\ref{eq:usmdiff} we can then determine the average change to an agent's usage level for some constant learning rate, which is plotted in Figure~\ref{fig:usmregularisation2}. The first striking observation is that, for~$T=2$, the mean expected change~$\Delta x$ is~0 for the entire range of values of~$x$, i.e.~the model is equivalent to the random copying of the neutral evolution condition. This result can be explained by inspecting the mapping functions directly above: with~$T=2$, the functions are only ever evaluated at $0, 0.5$ and $1$, all values for which they are identical to neutral copying, i.e.~where~$f(u)=u$. For higher values of~$T$, however, both regimes exhibit the expected influence on the agent's usage rate, with conformity-copying~(left panel) decreasing the usage of infrequent variants while further boosting the frequency of those which are already used more than $50\%$~of the time, and the opposite for anti-confirmity copying~(right panel). Generally, the higher~$T$, the stronger the impact of the function~$f(.)$, since bigger samples allow more evidence for variation\footnote{In particular, less probability mass on the two homogeneous samples $n=0$ and~$n=T$ to which $f(.)$ does not apply.}, a necessary ingredient for differential replication.

\citet{Blythe2012} report that with values of~$a>0$ any minority variants are rapidly eliminated from the population leading to the fixation of just one variant, while values~$a<0$ give rise to stable co-existence of all different variants, with the stability of co-existence dependent on the community size and learning rate of the individual agents~(p.286). These two regimes show that a frequency-dependent selection mechanism of this kind can be used to implement a bias for \emph{regularisation} (as well as~\emph{de-regularisation}) of competing linguistic variants. \index{regularisation}
However, neither of these two scenarios lead to directed transitions from the introduction of a novel variant to its complete adoption, as is the case in language change. While we will revisit the mechanism of regularisation based on \emph{innovation} rather than selection in a different evolutionary model in Section~\ref{sec:realigriffiths}, we now turn to the dynamics of USM configurations that implement \emph{asymmetries} of some kind.

\subsection{Weighted interactor selection}\index{interactor selection}\label{sec:usminteractorselection}\index{asymmetry}

Rather than rely on asymmetries in the \emph{replicators}~(the linguistic variants) that is implied by language-internal and variant prestige accounts, the more \emph{mechanical} social accounts discussed in Section~\ref{sec:interactorselection} seek the cause for the preferential spread of linguistic innovations in features of the social networks, such as differential interaction densities or the skewed influence of specific individuals or nodes in the network.
As discussed above, at least under the assumption of pure random copying the social network structure alone is not sufficient to alter the dynamics of neutral evolution to yield reliable directed transitions.
The idea of differential \emph{individual prestige} or influence however corresponds to a wholly separate mechanism, namely that of \emph{interactor selection}, where the asymmetry leading to the differential replication of variants is not due to the bias function~$f(.)$ but instead determined by the matrix~$H$ which controls the weight given to the samples obtained from different individuals in their interactions with others.

By adjusting the values in~$H$ accordingly, one can thus create an interaction structure where the production levels of some group of linguistically `leading' individuals is preferentially imitated by another group of `followers', but not vice versa. Proposals of this kind have been suggested to influence the diffusion of language changes through different \emph{adopter groups} of varying complexity~\citep{Labov2001,Rogers1962,Milroy1985}.
\citet{Blythe2012} use the utterance selection model to investigate the quantitative predictions made by different assumptions regarding the structure of adopter groups. Initiating only the `leading' group with high usage rates of an otherwise not established variant, they find that only a highly unrealistic staging of an entire chain of adopter groups, with the respective size of the groups following an exponential pattern, lead to s-shaped transitions. Having exhausted all other mechanisms of differential replication, they finally turn to the most direct way in which to affect the model dynamics, namely through a direct asymmetry between the linguistic replicators.

\subsection{Replicator selection}\index{replicator selection}\index{asymmetry}\label{sec:usmreplicator}

The most straightforward way to achieve a directed increase of one variant at the expense of the other is by implementing a bias function which consistently boosts the perceived frequency of that variant at the expense of all others. In this way, ``$f(u)>u$ for all frequencies $u$ between zero and one, and hence the listener perceives the innovation to be at a higher frequency than it was actually produced at, and overproduces accordingly''~\citep[p.291]{Blythe2012}. The simplest asymmetric linear function used by \citeauthor{Blythe2012},

\begin{equation}\label{eq:usmmultiplicative}
f(u) = u\cdot(1+b_m)\;,
\end{equation}
with a selection bias applied for all $b_m>0$, has an interesting property, namely that it is asymmetric in two ways: not only does it skew the likelihood of adoption towards one of the two competing variants, the strength with which this bias is applied also increases for higher frequencies of the innovative variant, as can be seen in Figure~\ref{fig:replicatorselection1}(i).
To confirm their finding that ``an S-shape is easily obtained through replicator selection''~(p.291) with little sensitivity to the precise selection function used, I will also be investigating a second model of replicator selection that employs an \emph{additive} instead of a multiplicative bias which exhibits equal strength across the entire trajectory,~i.e.

\begin{equation}\label{eq:usmadditive}
f(u)=u+b_a\;,
\end{equation}
again capped at the maximum value of 1.0, as can be seen in Figure~\ref{fig:replicatorselection1}(ii).
%To test whether this incremental amplification of the bias is necessary to produce s-shaped curves, I modified the replicator selection model to apply
In order to achieve a comparable bias strength for the two types of replicator selection, for all later figures the respective bias values  will satisfy $b_a=\frac{b_m}{2}$, a choice which results in equivalent strength selection at the mid-point as well as a (roughly) equal amplification relative to the neutral evolution condition~$f(u)=u$ across the entire trajectory.

From these bias functions~$f(.)$ we can again derive the average perceived frequency for different settings of~$T$ as well as the consequent expected change to the agents' usage frequencies, which are shown in Figure~\ref{fig:replicatorselection2}. The asymmetry of replicator selection is immediately evident from the fact that the expected change is always greater than zero, meaning that the incoming variant is boosted across the entire frequency range.
By solving the differential equations defined by these functions we can also calculate the average trajectories that would be produced as the selected for variant spreads through the population, which are shown in Figure~\ref{fig:replicatorselection3}~\citep[for an in-depth explanation of the approach see the appendix to][mathematical derivations of the results for both models of replicator selection are provided in Appendix~\ref{app:usm}]{Blythe2012}.

While for both models of selection the average changes to~$x$ at~$T=2$ result in growth patterns that are equivalent to the canonical s-shaped logistic function, \index{logistic growth}
the relative intensity of selection across the trajectory diverges for higher values of~$T$. For the original, multiplicative bias model shown in Figure~\ref{fig:replicatorselection3}(i), increasing the sample resolution leads to an extended period of initial growth in which the amount of change increments linearly past the half-way mark at which simple logistic growth characteristically starts to slow down again, making the growth pattern more and more exponential as the sample resolution~$T\rightarrow\infty$.
Consequently, this selection function predicts the maximum rate of growth, indicated by the peak in Figure~\ref{fig:replicatorselection2}(i), to occur later during the trajectory for higher settings of~$T$, while also exhibiting relatively greater levels of selection and thus faster transitions given the same bias strength~$b_m$.

A different picture emerges for the additive bias as shown in Figure~\ref{fig:replicatorselection2}(ii). While the overall selection pressure imposed by the bias function also increases for higher~$T$, the pattern of selection stays symmetric around the mid-point, with a relative acceleration of transitions achieved by a greater degree of selection in the other frequency regions.
For $T=2$ as well as $T=3$ the selection dynamics can be shown to be identical to logistic growth as displayed in Figure~\ref{fig:replicatorselection3}(ii), with growth rates of $2b_a$ and $3b_a$ respectively~(see Appendix~\ref{app:usm}). No general solution for the average expected trajectory is provided here but, since the function describing the mean change to~$x$ shown in Figure~\ref{fig:replicatorselection2}(ii) approaches a constant value throughout the interval~$(0,1)$ as $T$ increases, we should expect the dynamics to start resembling those of steady linear growth as~$T\rightarrow\infty$.

% Filling in the three different replicator biases and solving the differential equations for them we can thus deduce the average trajectories:

\begin{figure}[htbp]

{\centering \subfloat[Objective to perceived frequency mapping under replicator selection. \emph{(i)}~multiplicative replicator bias presented in~\citet{Blythe2012}, with $b_m=0.1$. \emph{(ii)} additive bias with equivalent bias strength~$b_a=b_m/2=0.05$. All mapping functions are capped at 0 and 1 and impose $f(0)=f(1)=0$. The dotted line indicates the unbiased mapping characteristic of neutral evolution, i.e.~$f(u)=u$.\label{fig:replicatorselection1}]{\includegraphics[width=\maxwidth]{figure/replicatorselection-1} }
\subfloat[Average bias applied with multiplicative and additive replicator selection for different values of~$T$, assuming~$\lambda=1$. The asymmetry of bias functions leads to an increase of the innovative variant across all frequency ranges, with generally stronger selection for higher~$T$. Mathematical derivations for the functions can be found in Appendix~\ref{app:usm}.\label{fig:replicatorselection2}]{\includegraphics[width=\maxwidth]{figure/replicatorselection-2} }
\subfloat[Typical average trajectories resulting from applying the multiplicative and additive replicator biases. For higher~$T$ the multiplicative bias extends the initial period of exponential growth, while the additive bias remains symmetric around the mid-point with fast growth from the very start of the trajectory.\label{fig:replicatorselection3}]{\includegraphics[width=\maxwidth]{figure/replicatorselection-3} }

}

\caption[Analysis of the selection dynamics for the original, multiplicative (left column) as well as additive (right column) replicator selection bias]{Analysis of the selection dynamics for the original, multiplicative (left column) as well as additive (right column) replicator selection bias.}\label{fig:replicatorselection}
\end{figure}



With this analysis we come to the end of an overview of some of the most relevant results that have been obtained from the computational version of the Utterance Selection Model originally due to~\citet{Baxter2006}.
Starting from an investigation of its baseline learning dynamics, which were shown to implement a model of \emph{neutral evolution}, we recapitulated the survey of pressures provided in \citet{Blythe2012} to determine which evolutionary replication mechanisms could lead to directed, s-shaped trajectories. \citeauthor{Blythe2012} concluded that only an inherent asymmetry between variants, implemented as a preference for the innovative variant that is shared by the majority of interacting agents, can reliably produce directed transitions. This conclusion can help us rule out some of the proposed pressures which, while not completely ineffective, do not appear to have the necessary leverage to be the main driving force behind the adoption of individual language changes, at least as far as one accepts the model's underlying assumptions regarding the behaviour of individual agents and their interactions.
% in particular those based on social network effects as well as individual prestige.
But, curiously, the result does not speak to the two biggest contenders to explaining language change: both language-internal and social, \emph{variant prestige} accounts are based on an underlying preference for an innovative over an outgoing variant, as implemented by a replicator selection bias. While this last Section showed how slightly different assumptions regarding the strength of the asymmetry and force of the selection pressure across different attested frequencies of the variants can yield slightly different predictions regarding the resulting trajectories, a look at the quantitative investigations of language changes presented in Chapter~\ref{ch:review} indicated that the empirical data might be too sparse to make any strong claims or perform conclusive comparisons between models and data~\citep[although see][]{Altmann2013,Ghanbarnejad2014}.

Despite the fact that the USM features a larger number of parameters than most other models of language evolution, every single one is both transparent and grounded in the USM's dedication to a concrete evolutionary framework~\citep{Croft2000}.
The sample resolution parameter~$T$, for example, while unusual in terms of its absence from other models of language change, is well-motivated by the model's strict separation of innovation and selection pressures.
The persistence with which the model has been analysed, both via analytical methods and numerical simulation, means that the sensibility of its dynamics~(or lack thereof) with regard to different parameters and parameter combinations are much better understood than for most other models~\citep{Blythe2007divided,Baxter2009,Blythe2009,Blythe2012neutral,Baxter2016,Michaud2017}.

Before I go on to expand the studies of the USM through another selection mechanism intended to tackle the question of how asymmetries between variants might emerge out of the replication dynamics itself, the main contribution of Chapter~\ref{ch:momentummodel}, I will first introduce a simpler, yet related, modelling framework that has also been used to investigate s-shaped curves, and which will also resurface again in Chapter~\ref{ch:conclusion}.

%Similarly, since $f(u)$ can never go above 1, the impact of any asymmetric selection functions are necessarily diminished when more and more datapoints that come in are at or close to $1$, which happens more often the closer the population average is to $1$. Thus, the current population average value of variant use also affects how exactly~$f(u)$ affects learning. This interaction is shown in the next Figure. 


\section{A Markov chain model of neutral evolution}
\index{Bayesian inference}\index{Iterated Learning}
\label{sec:realigriffiths}

One formal model of neutral evolution (i.e.~copying of linguistic traits in the absence of any replicator or interactor selection) that makes particular reference to the temporal dynamics of changes is Reali \& Griffiths model of regularisation by Bayesian learners~(\citeyear{Reali2009,Reali2010}).

%\subsection{Model description}
At its core, \citeauthor{Reali2009} present a model of frequency learning by Bayesian inference. In their particular framing, an individual is trying to infer the relative frequencies $\theta_i \in [0,1]$ of different variants $i=1\ldots n$ based on some input data as well as prior beliefs about what the true values of $\theta_i$ are likely to be. These prior beliefs act as \emph{inductive biases} and are captured by the \emph{prior}, represented by a probability distribution $f(\vec{\theta})$ defined over all possible values of $\vec{\theta}$.

For the simple case of two competing variants, even though the individual is technically inferring two complementary relative frequencies $\theta_1, \theta_2$, we can limit our analysis to the problem of inferring $\theta_1$, since trivially $\theta_2=1-\theta_1$. The model can easily be extended from the \emph{binomial}~(two-variant) outcome to \emph{multinomial} outcomes, i.e. with three or more competing variants but, without loss of generality, we will limit our demonstration to the case of two competing variants. To simplify notation we will henceforth also write $\theta$ to refer to $\theta_1$.

While any continuous probability distribution over the interval $[0,1]$ could serve as a prior, the authors choose the \emph{Beta distribution}, whose probability density function is defined as
\begin{equation}
f(x;\alpha,\beta) = \frac{1}{B(\alpha,\beta)} x^{\alpha-1}(1-x)^{\beta-1}\;,
\end{equation}
where $B(.)$ is the \emph{Beta function}.

Because we are interested in a \emph{neutral} model that is not a priori biased in favour or against either of the competing variants, the shape of the prior distribution over the support will have to be \emph{symmetric}: the prior probability density of $\theta$ taking a certain value, $f(\theta)$, should be the same as its probability of taking the complementary value $f(1-\theta)$. This can be achieved by setting the Beta distribution's shape parameters $\alpha,\beta$ to the same value. Consequently the authors use prior distributions of the form
\begin{equation}
\Theta \sim Beta(\frac{\alpha}{2},\frac{\alpha}{2})\;.
\end{equation}
with just a single parameter, $\alpha$, which controls the degree of \emph{regularisation}. Figure~\ref{fig:priors} shows the effect of this parameter on the prior distribution. For a value of $\frac{\alpha}{2}=1$ the prior distribution is uniform: not only is the individual not biased towards any of the variants (the distribution is symmetric), their estimate of the underlying frequency~$\theta$ is not biased towards any particular frequency region in~$[0,1]$ either. The same isn't true when $\frac{\alpha}{2}\ne 1$: for values~$<1$, the inference of $\theta$ is explicitly geared towards more extreme relative frequencies closer to $0$ or $100\%$ usage -- the model implements a \emph{regularisation bias}. The opposite is the case when~$>1$ which \emph{a priori} favours values of $\theta$ around the $0.5$ mark. Agents employing such a setting are inclined to infer more mixed usage of the competing variants than suggested by their learning data alone.

% In the case of $K\ge3$ competing variants the Beta distribution prior with shape parameters $\frac{\alpha}{2}$ is simply replaced by a Dirichlet prior with parameters $\frac{\alpha}{K}$.


\begin{figure}[htbp]

{\centering \subfloat[Prior distributions for three different levels of~$\alpha$.\label{fig:priors1}]{\includegraphics[width=\maxwidth]{figure/priors-1} }
\subfloat[Posterior distributions after observing $N=10$ data points with various input distributions ($x=5,2,0$), regularisation parameters as in (a).\label{fig:priors2}]{\includegraphics[width=\maxwidth]{figure/priors-2} }

}

\caption[Examples of Beta distribution priors and posteriors with three different levels of the regularisation parameter~]{Examples of Beta distribution priors and posteriors with three different levels of the regularisation parameter~$\alpha/2$.}\label{fig:priors}
\end{figure}



The particular choice of prior distribution~(Beta or Dirichlet for the multinomial case) has elegant mathematical properties: when a learner receives an input sample of size~$N$, where $0\le x \le N$ of the tokens were instances of the variant whose frequency $\theta$ they are trying to infer, then the posterior is again a Beta distribution, namely
\begin{equation}
\label{eq:posterior}
\Theta|x \sim Beta(x+\frac{\alpha}{2}, N-x+\frac{\alpha}{2})\;. % TODO fix notation
\end{equation}

Following this inference step, there is still the question of how the posterior distribution is translated into actual production behaviour, which provides us with testable predictions of the model. % after all, we need production behaviour to make a full iterated learning model.
Here, we will consider three different ways for an individual to generate their own productions~$x'$ based on the learning sample~$x$ that they themselves received. % deriving a specific value~$\theta$ from the posterior distribution,
The first two were also treated by \citet{Reali2009}, the third covered by \citet[p.156]{Ferdinand2015}:

\begin{description}
\item[Sampling from the posterior:] when generating new productions directly from the posterior, the probability that a \emph{sampling learner} produces a particular variant $x'$~times out of a total of $N$~productions is distributed according to a \emph{Betabinomial distribution} with the same parameters as the posterior distribution in Equation~\ref{eq:posterior}, i.e.
\begin{equation}\label{eq:usmsampling}
X'|x \sim BB(x+\frac{\alpha}{2}, N-x+\frac{\alpha}{2}, N) \;.
\end{equation}

\item[Adopting the \emph{average} of the posterior:] instead of sampling from the posterior for every production, an individual could deterministically select the mean of the posterior distribution, which is
\begin{equation}\label{eq:usmaveraging}
\hat{\theta}=\frac{x+\frac{\alpha}{2}}{N+\alpha}\;.%  close to the observed relative frequency $\frac{x_1}{N}$.
\end{equation}
The productions of a Bayesian learner who deterministally chooses the parameter $\hat{\theta}$ are then distributed according to a Binomial %(or, in the case of multinomial outcomes, Multinomial)
distribution,
\begin{equation}\label{eq:usmmap}
X'|x \sim Bin(N, \hat{\theta}) \;.
\end{equation}
While \citet{Reali2009} call this a `MAP' learner, we will refer to this mechanism of selecting a hypothesis as the \emph{averager} strategy.

\item[Adopting the \emph{mode} of the posterior (\emph{maximum a posteriori}):] The posterior distribution's mode, where the probability density function is highest, can be found at
\begin{equation}
\theta_{MAP} = \arg\max_{\theta} f(\theta|x) = \frac{x+\frac{\alpha}{2}-1}{N+\alpha-2}\;,
\end{equation}
except when~$x=0$ or~$x=N$, in which case the resulting posterior Beta distribution is \emph{j-shaped}, with the mode falling on~$0$ or~$1$, respectively. When such a \emph{MAP learner} has adopted the mode as their production probability then their own productions are distributed according to a Binomial distribution with $p=\theta_{MAP}$, i.e.
\begin{equation}
X'|x \sim Bin(N, \theta_{MAP}) \;.
\end{equation}
\end{description}

One way in which the impact of these different ways of sampling data (either directly from the posterior or by first deterministically selecting a $\theta$) can be exemplified is by visualising the \emph{average production} of a learner who is inferring the underlying distribution based on the input sample they just observed. This data is shown in Figure~\ref{fig:meanmapping}, which maps the different possible input distributions~(along the x-axis) to the average output productions~$\pm$~their standard deviation. The identity function $x=y$, equivalent to pure probability matching, is shown for reference. In this graphical representation, a mapping function that leads to increased \emph{regularisation} should map input proportions between $0$~and~$50\%$ to even \emph{lower} output proportions, while input proportions~$>50\%$ should yield output proportions even closer to~$100\%$.

What is evident from Figure~\ref{fig:meanmapping} is that the only method which on average leads to regularisation at every iteration is the \emph{maximum a posteriori} method with $\alpha\le 1$. None of the other mapping functions are consistently regularising. Rather, as was pointed out by~\citet[p.176]{Ferdinand2015} both data production methods discussed by \citeauthor{Reali2009} rely on mechanisms that merely increase the sample variability \emph{in either direction}, until the system drifts into a state of categorical presence of one variant only. This contrasts with the regularising mapping functions of the Utterance Selection Model shown in Section~\ref{sec:usmregularisation}, which systematically increase the proportion of whichever variant is currently more prevalent. We will return to a critique of the present regularisation model in the next section. % \citep{Spike??}
% TODO point out the 1/10 -> 2/10 transition probability

\begin{figure}[htbp]

{\centering \subfloat[Input/mean output mapping when sampling from the posterior.\label{fig:meanmapping1}]{\includegraphics[width=\maxwidth]{figure/meanmapping-1} }
\subfloat[Input/mean output mapping when selecting the \emph{average} of the posterior as the hypothesis. As pointed out by \citet{Ferdinand2015}, the \emph{mean} output of this model is identical to that of the sampler shown above, only that the averager exhibits different amounts of sampling error about this mean, depending on the input frequency.\label{fig:meanmapping2}]{\includegraphics[width=\maxwidth]{figure/meanmapping-2} }
\subfloat[Input/mean output mapping when selecting the \emph{maximum} of the posterior as the hypothesis~(MAP). With~$\frac{\alpha}{2}=1$~(middle panel) this strategy is identical to pure frequency matching, while MAP with $\frac{\alpha}{2}<1$~(left panel) is the only strategy that, \emph{on average}, leads to regularisation in one iteration.\label{fig:meanmapping3}]{\includegraphics[width=\maxwidth]{figure/meanmapping-3} }

}

\caption[Input to mean-output mapping for the three ways of producing data from the posterior and three levels of the regularisation parameter]{Input to mean-output mapping for the three ways of producing data from the posterior and three levels of the regularisation parameter. The three settings of $\alpha$ capture inductive biases ranging from regularisation~($\frac{\alpha}{2}=.25$, left column) to de-regularisation ($\frac{\alpha}{2}=5$, right column).}\label{fig:meanmapping}
\end{figure}



% With these three different models up our sleeve, we can turn to actual productions.

\subsection{Representing Bayesian Iterated Learning as a Markov chain}\index{Markov model}
\label{sec:markovmodel}

While the model presented above captures frequency learning by Bayesian inference within one individual, it is interesting to ask how the productions of a sequence of such learners would develop over time when one individual's output serves as the learning input of another. To do this, we can analyse the interactions between repeated learning input and production output as a \emph{Markov chain}, a simple modelling tool for understanding systems which can be in one of a finite number of states that they switch between probabilistically.

More formally, a Markov model can be defined by specifying conditional transition probabilities $P(X_{t+1}=x'|X_t=x)$ between a number of discrete states $x,x'\in\mathcal{S}$, which we call the Markov model's \emph{state space}. The Markov model is completely described by a function $P: \mathcal{S}\times\mathcal{S} \rightarrow [0,1]$ where the transition probabilities \emph{out} of any given state have to sum to one, i.e.
\begin{equation}
\sum_{x'\in\mathcal{S}} P(X_{t+1}=x'|X_t=x) = 1 \qquad\forall\; x \in \mathcal{S}\;.
\end{equation}

In the case of the Bayesian inference model above, there are two equally valid ways in which it could be translated into a Markov model, based on how the state space $\mathcal{S}$ is construed. The logical alternation between learning parameter $\theta$ and production of $x$ tokens of a specific variant out of $N$~total productions allows for both a characterisation of the Markov model as transitioning from one individual's posterior distribution $f(\theta|x)$ to another or, alternatively, from one individual's number of productions~$x$ to the next.

%The state space of the respective Markov models would be either defined by the set of all possible posterior distributions, or alternatively by the set of all possible productions. In the former case the respective transition probabilities would then capture the probability of an individual's posterior distribution resulting in the following individual having a particular posterior distribution or, in the case of the production-centric view, simply the probability of one individual producing a certain number of tokens of variant~1 given how many tokens of that type the previous learner produced.
%For sake of simplicity and increased interpretability, 

To define the state space, we have to set a fixed size of productions~$N$, from which a new learner has to infer the underlying production frequency~$\theta$.

An example of such a transition matrix for $N=10, \alpha=0.5$ is found in Table~\ref{tbl:transitionmatrixsample}. This particular matrix is created based on the assumption that learners sample their data directly from the posterior distribution they computed from the input they received.

Compare this to Table~\ref{tbl:transitionmatrixmap}, which is based on a chain of learners that deterministically select the mode $\theta_{MAP}$ of the posterior distribution $f(\theta|x)$ as their estimate of $\theta$. Their data production probabilities are consequently distributed according to a Binomial distribution with $p=\theta_{MAP}$, so the rows of this transition matrix are equivalent to this Binomial distribution.

% latex table generated in R 3.2.3 by xtable 1.8-2 package
% Thu Aug 25 13:55:50 2016
\begin{table}[htbp]
\centering
\begin{tabular}{rrrrrr}
  \hline
 & $x'=0$ & $x'=1$ & $x'=2$ & $x'=3$ & $x'=4$ \\ 
  \hline
$x=0$ & 0.8379 & 0.1156 & 0.0347 & 0.0099 & 0.0019 \\ 
  $x=1$ & 0.3756 & 0.3005 & 0.1932 & 0.0985 & 0.0322 \\ 
  $x=2$ & 0.1352 & 0.2318 & 0.2659 & 0.2318 & 0.1352 \\ 
  $x=3$ & 0.0322 & 0.0985 & 0.1932 & 0.3005 & 0.3756 \\ 
  $x=4$ & 0.0019 & 0.0099 & 0.0347 & 0.1156 & 0.8379 \\ 
   \hline
\end{tabular}
\caption[Markov chain transition matrix for the Bayesian Iterated Learning model with sampling from the posterior]{Markov chain transition matrix for the Bayesian Iterated Learning model with $N=4$ and $\alpha/2=0.25$. The rows represent the probabilities of producing any of the given samples, assuming that the production is sampled from the posterior.} 
\label{tbl:transitionmatrixsample}
\end{table}
% latex table generated in R 3.2.3 by xtable 1.8-2 package
% Thu Aug 25 13:55:50 2016
\begin{table}[htbp]
\centering
\begin{tabular}{rrrrrr}
  \hline
 & $x'=0$ & $x'=1$ & $x'=2$ & $x'=3$ & $x'=4$ \\ 
  \hline
$x=0$ & 1.0000 & 0.0000 & 0.0000 & 0.0000 & 0.0000 \\ 
  $x=1$ & 0.6561 & 0.2916 & 0.0486 & 0.0036 & 0.0001 \\ 
  $x=2$ & 0.0625 & 0.2500 & 0.3750 & 0.2500 & 0.0625 \\ 
  $x=3$ & 0.0001 & 0.0036 & 0.0486 & 0.2916 & 0.6561 \\ 
  $x=4$ & 0.0000 & 0.0000 & 0.0000 & 0.0000 & 1.0000 \\ 
   \hline
\end{tabular}
\caption[Markov chain transition matrix for the Bayesian Iterated Learning model selecting the mean of the posterior]{Markov chain transition matrix for the Bayesian Iterated Learning model with $N=4$ and $\alpha/2=0.25$. The rows represent the probabilities of producing any of the given samples, equivalent to $Bin(x';N,p=\theta_{MAP})$} 
\label{tbl:transitionmatrixmap}
\end{table}


The system that we describe by specifying the transition probabilities between individual states is a random process called a \emph{Markov chain}. Stochastic systems of this kind are said to obey the \emph{Markov property}, which means that the probability of entering a particular state only depends on the system's \emph{current} state, but not on any other prior states or state sequences that preceded the current one. This image of a \emph{chain} maps neatly onto the Iterated Learning model, where every new learner receives input from their parent generation who they then replace.

Importantly for us, the characterisation of a stochastic system as a Markov chain allows for straightforward analyses of different kinds. For example, assuming that our system would run indefinitely, we can calculate the probability of this infinite chain of states to occupy a particular state \emph{in the limit}. The so-called \emph{stationary distribution}~$\pi$ of a Markov chain transition matrix~$T$ is a probability distribution over its states~$S$, i.e.~it must satisfy
\begin{equation}
\pi\ge0,\; \sum_{s\in S} \pi_s = 1\;.
\end{equation}

In mathematical terms, the stationary distribution has the property that performing another iteration of the chain must map the distribution onto itself, i.e.\index{stationary distribution}

\begin{equation}\label{eq:stationarydistribution}
\pi = \pi\cdot T\;.
\end{equation}

Based on these definitions, it is possible for a given Markov chain to have more than one stationary distribution. This is generally only the case when the state space consists of subpartitions that cannot be reached from each other, as is the case when there is more than one absorbing state. The stationary distributions of the different systems whose input/mean-output mapping we visualised previously in Figure~\ref{fig:meanmapping} are shown in Figure~\ref{fig:stationarydistribution}.

\begin{figure}[htbp]

{\centering \subfloat[Stationary distribution for chains of learners who are sampling from the posterior.\label{fig:stationarydistribution1}]{\includegraphics[width=\maxwidth]{figure/stationarydistribution-1} }
\subfloat[average\label{fig:averagerstationarydistribution}\label{fig:stationarydistribution2}]{\includegraphics[width=\maxwidth]{figure/stationarydistribution-2} }
\subfloat[Stationary distribution for \emph{maximum a posteriori}~(MAP) learners. The different colours indicate that for $\alpha/2\le 1$ the Markov chain has two absorbing states, corresponding to categorical usage of either of the variants.\label{fig:stationarydistribution3}]{\includegraphics[width=\maxwidth]{figure/stationarydistribution-3} }

}

\caption[Stationary distributions of the Markov chain transition matrices]{Stationary distributions of the Markov chain transition matrices.}\label{fig:stationarydistribution}
\end{figure}



The stationary distributions confirm that the parameter~$\alpha$ indeed works as intended: when $\alpha/2<1$, the chains spend most of their time in the extreme states corresponding to categorical usage of either of the two competing variants. %As the sample size $N$ increases 
When $\alpha/2>1$, on the other hand, the chains mostly consist of learners who mix the variants evenly. The behaviour with intermediate values $\alpha/2\approx1$ falls in between, with the exact distribution also depending on the type of learners.

The MAP learner, not considered in the original \citeauthor{Reali2009} papers, deserves special attention: as already hinted at above, only this learning strategy looks like a proper \emph{regulariser} in the sense that an input proportion will, \emph{on average}, result in an output proportion that is in fact more regular than the input. It is also the only learning strategy which, for any $\alpha/2\le1$, does not introduce variation when there isn't any in the input, i.e.~learners who receive homogeneous input will never spontaneously introduce variation into their output. Figure~\ref{fig:stationarydistribution3} shows that, as a consequence, chains of such learners will end up in either of two \emph{absorbing states} corresponding to categorical usage of a variant, and remain there indefinitely.

\subsection{Neutral evolution and s-shaped curves}

So far our analysis of the stationary distribution limits us to describing the expected state of a model, but abstracted away from time.
One particular claim of \citet{Reali2010} concerning temporal dynamics is that even the neutral evolution model described by Bayesian regularisers will produce s-shaped curves. While we would not expect completely symmetric replication such as produced by neutral evolution to produce particularly directed transitions, they argue that this depends on which data is considered. In particular, since historical linguists only (or at least primarily) describe changes which have gone to completion, our assessment of whether a model produces s-shaped curves should equally be limited to data of this kind. They consequently go on to analyse only those chains that start off in a state where the first generation uses one of the competing variants categorically, while the last ends up in the opposite state where its productions contain only the other variant.

In order to get a better understanding of the underlying dynamics of our Markov model, we will therefore need to switch to an analysis that allows us to condition the Markov chains to be in specific states at specific points in time. One tool to do exactly this are \emph{Hidden Markov Models}~(HMMs). As the name suggests, HMMs are closely related to the Markov models described above. While in `normal' Markov chains the state sequence is directly visible to the observer, Hidden Markov Models allow us to specify a certain level of uncertainty over the model's state at any given point in time. Of particular importance to is that, instead of just randomly generating state sequences, HMMs allow us to make probabilistic inferences about the most likely states or state sequences that our model is likely to be in.

In what follows, I used R's \texttt{HMM} package~\citep{HMM1.0} to both replicate and extend the results reported in~\citet{Reali2010}. Firstly, Figure~\ref{fig:naiveconditioning} shows a replication of the original analysis from their paper. All four subplots show the state probability distribution for Markov chains of length $50$ where the input data presented to the first generation consisted of $50$ instances of only one variant. The probability distribution is represented as a heat map where, for any specific generation, darker colors indicate a higher probability of being in a state at that time. The probabilities of all states per generation sum to 1. The particular probability distributions shown here were calculated for chains of learners which use the inferred mean $\hat{\theta}$ of the posterior distribution to sample data for the following generation, but results for learners sampling directly from the posterior are qualitatively similar.

\begin{figure}[htbp]

{\centering \subfloat[Results with learners accepting the \emph{mean} of the posterior as their hypothesis for~$\theta$ with $\alpha=0.5$.\label{fig:naiveconditioning1}]{\includegraphics[width=\maxwidth]{figure/naiveconditioning-1} }
\subfloat[Results with learners accepting the \emph{mean} of the posterior as their hypothesis for~$\theta$ with $\alpha=10$.\label{fig:naiveconditioning2}]{\includegraphics[width=\maxwidth]{figure/naiveconditioning-2} }

}

\caption[State probability distribution for all Markov chains where the input to the first generation consists of tokens of only one variant]{State probability distribution for all Markov chains of length 50 where the input to the first generation consists of tokens of only one variant. The dashed white line indicates the trajectory through the `average' states that the chain is in at any given point in time. \emph{(i)}~conditioning on the first generation's input only \emph{(ii)}~conditioning on both the first and final generations' data.}\label{fig:naiveconditioning}
\end{figure}



Subfigures~(i) on the left show the development of the chains when conditioning on this initial state only. These two plots, which differ only in their setting of~$\alpha$, neatly highlight the contrast between the two different regimes of the regularisation parameter $alpha$: in Figure~\ref{fig:naiveconditioning1} we set $\alpha=0.5$, corresponding to \emph{regularisation}. In this setting, chains of learners are drawn to produce either of the two variants (near-)categorically. Note that, even though the system starts off with only one variant as its input, the chance introduction of tokens of the competing variant leads some chains to eventually regularise in the `other direction': whenever tokens of the other variant accumulate through random sampling, the chains start to be equally drawn towards the other fully regular state, i.e.~categorical usage of the formerly unattested variant.

While even after 50~generations the majority of chains is still at or near the usage frequency that was presented to the first participant, increasingly chains will start to `bunch up' against the top-most state corresponding to categorical usage of the other variant. Indeed, in the limit we should expect the the right-most `slice' of Figure~\ref{fig:naiveconditioning1}(i) to become completely symmetric around the halfway-mark, as it approaches the Markov chain's stationary distribution shown above in Figure~\ref{fig:stationarydistribution2}(i).

In contrast, the left panel of Figure~\ref{fig:naiveconditioning2} with $\alpha=10$ represents the \emph{de-regularisation} regime, where individuals prefer to use both variants equally. This is borne out by the fact that chains of such learners are quickly drawn towards the middle states, indicating mixed usage.

Subfigures~(ii) on the right-hand side show the expected distribution of states when conditioning on both the initial and final states of the chain, where the last individual only produces tokens of the competing variant that was not attested in the first generation's input data. While the probability distribution over possible states at most intermediate generations is extremely wide, \citeauthor{Reali2010} point to the \emph{average trajectory}~(shown in white) that is calculated by computing the \emph{average state} of all chains at any given generation. They point out that, intriguingly, the shape of this average trajectory is dependent on the regularisation parameter $\alpha$. In particular, the model produces s-shaped trajectories exactly when chains are geared towards regularising input, which experimental evidence suggests is in fact a feature of human language learning~\citep{HudsonKam2005,Reali2009,Smith2010}.

It is crucial to point out here is that this \emph{average} of all transitions is not necessarily representative of the model's \emph{typical} transitions~\citep{Blythe2012neutral}. In order to get an idea of what individual trajectories of Iterated Learning chains actually look like, we can simply generate state sequences of the underlying Markov model randomly and filter them according to the start and end conditions~(see Appendix~\ref{app:markovmodel} for the code).

Figure~\ref{fig:naiveconditioningtransitions} shows three randomly generated chains that fulfill both the start and end condition specified above. The trajectories were generated using exactly the same parameter setting as the one underlying the s-shaped average trajectory shown in Figure~\ref{fig:naiveconditioning}. Already here we can see that individual trajectories are much more noisy, less directed and s-shaped than the numerically computed `average transition' above suggests.

\begin{figure}[htbp]

{\centering \includegraphics[width=\maxwidth]{figure/naiveconditioningtransitions-1} 

}

\caption[Three randomly generated Markov chains initiated at $0/N$ and terminating at $N/N$ after 50 iterations.]{Three randomly generated Markov chains initiated at $0/N$ and terminating at $N/N$ after 50 iterations. \emph{(i)}~learners sampling from the posterior distribution~$p(\theta|x)$. \emph{(ii)}~learners accepting the \emph{mean} of the posterior as their hypothesis for~$\theta$.}\label{fig:naiveconditioningtransitions}
\end{figure}



What is also evident is that not all of the `transitions' are actually of the length that we specified: many chains either remain at the initial state for some time, or otherwise converge on categorical usage of the other variant early and remain there until the remaining generations have passed. This points to another more general problem, namely that termination after exactly 50~generations is not actually well-motivated. %Even a single-generation jump from from 0/50 to 50/50 has a non-zero~(if extremely small) probability and would arguably not be s-shaped.
To understand the dynamics of this model even better we should therefore take a closer look at the expected duration of transitions.

\subsubsection{Expected number of generations for a transition to complete}

In order to get a more accurate picture of the typical trajectory exhibited by regularising Iterated Leaners, we first need to know the likelihood of a transition completing in a given number of generations. Figures~\ref{fig:naiveconditioningprobabilities} shows both the per-iteration probability as well as the cumulative probability of a chain of Iterated Learners reaching categorical usage of the initially non-existent, incoming variant over time.

\begin{figure}[htbp]

{\centering \includegraphics[width=\maxwidth]{figure/naiveconditioningprobabilities-1} 

}

\caption[Probability of transitions from categorical usage of one to categorical usage of the other variant]{Probability of transitions from categorical usage of one to categorical usage of the other variant, for learners accepting the \emph{mean} of the posterior as their hypothesis for~$\theta$. \emph{(i)}~probability of completing first transition after the given number of generations \emph{(ii)}~cumulative probability of having completed at least one transition.}\label{fig:naiveconditioningprobabilities}
\end{figure}



For the \emph{averaging} learner with parameters $\frac{\alpha}{2}=0.25$ and $N=50$ as above, the chain is most likely to first reach categorical usage of the incoming variant at the distribution's mode after 149 generations, while on average the first transition takes 444 iterations to complete.

The distribution of the expected duration of a transition by a chain of learners sampling directly from the posterior distribution is qualitatively similar. Using the same parameter settings as above, the most likely and mean duration until completion of the first transition are 87 and 310 respectively).

%Knowing this we can now generate some of these more `typical' trajectories.

\subsubsection{Average trajectory of transitions that have the exact same duration}

As pointed out above, the number of generations until a new variant has fixated isn't actually representative of the \emph{duration} of a transition. Since chains might remain at their initial state for a few iterations before picking up, or also return back to the initial state before picking up again. If we are interested in the length of the actual transition (i.e.~we only start to measure the duration of a transition when the new variant is first innovated) the distribution of transition durations looks quite different, as shown in Figure~\ref{fig:complexconditioningprobabilities}.

\begin{figure}[htbp]

{\centering \subfloat[Results with learners sampling from the posterior distribution~$p(\theta|x)$.\label{fig:complexconditioningprobabilities1}]{\includegraphics[width=\maxwidth]{figure/complexconditioningprobabilities-1} }
\subfloat[Results with learners accepting the \emph{mean} of the posterior as their hypothesis for~$\theta$.\label{fig:complexconditioningprobabilities2}]{\includegraphics[width=\maxwidth]{figure/complexconditioningprobabilities-2} }

}

\caption[Probability of having completed a transition in exactly the number of generations without ever reverting back to the initial state]{Probability of having completed a transition in exactly the number of generations without ever reverting back to the initial state.}\label{fig:complexconditioningprobabilities}
\end{figure}



An immediately obvious difference between this and the earlier distribution of transition durations in Figure~\ref{fig:naiveconditioningprobabilities} is that the cumulative probability in subfigure~(i) never reaches 1. Under \citet{Reali2010}'s original condition on the final state only, which allowed all possible intermediate trajectories, all chains would eventually reach the target state at some point.

Not so when conditioning on transitions which have to last an exact number of generations: Figure~\ref{fig:complexconditioningprobabilities} only considers transitions that, from their moment of actuation, actually reach the target state without ever `failing' (i.e.~returning back to the categorical initial state) in between. For a chain of learners who take the mean of their posterior distribution as their hypothesised underlying frequency~$\theta$, only about $1.46\%$ of initial introductions of a new, competing variant actually lead to successful transitions without any interruptions.

In terms of the distribution of durations of those transitions which \emph{are} successful, the number of generations until completion are expectedly much lower than in Figure~\ref{fig:naiveconditioningprobabilities} above. For the \emph{averaging} learner, the most like exact duration of a successful transition is much lower at 75~generations, with the mean duration at around~135 generations.
For the \emph{sampler} the values are even lower~(mode 36, mean~58).

Figure~\ref{fig:doubleconditioned} shows the state probability distribution as well as average trajectory of the Markov chains which are conditioned on introducing the initially unattested variant in the very first generation, as well as on only tokens of that variant at the maximum number of generations (and no earlier), without ever returning to the initial state.
Results are shown for both \emph{sampling}~(Figure~\ref{fig:doubleconditioned1}) as well as \emph{averaging}~(Figure~\ref{fig:doubleconditioned2}) learners for two different representative durations, the most likely duration of a transition~(the mode of the distributions in Figure~\ref{fig:complexconditioningprobabilities}) and the~(higher) \emph{mean} duration. %for a transition using the given parameters)
The Figure shows that the average of all transitions, again indicated by the dashed white line, is actually more like an~$S$ bent in the `wrong' direction. In other words, unlike what we find in empirical data on language changes, some of the slowest rates of growth occur at the mid-point of the change, similar to the average transition of chains of \emph{de-regularising} learners shown in Figure~\ref{fig:naiveconditioning2}.

\begin{figure}[htbp]

{\centering \subfloat[Results with learners sampling from the posterior distribution~$p(\theta|x)$, most likely duration and average duration of a transition are 36 and 58 generations, respectively.\label{fig:doubleconditioned1}]{\includegraphics[width=\maxwidth]{figure/doubleconditioned-1} }
\subfloat[Results with learners accepting the \emph{mean} of the posterior as their hypothesis for~$\theta$, most likely duration and average duration of a transition are 75 and 135 generations, respectively.\label{fig:doubleconditioned2}]{\includegraphics[width=\maxwidth]{figure/doubleconditioned-2} }

}

\caption[State probability distribution for all Markov chains exhibiting a transition with the exact same duration.]{State probability distribution for all Markov chains exhibiting a transition with the exact same duration. The dashed white line shows the \emph{average} trajectory, while the white dots indicate one of the \emph{most likely} transition paths. The duration is set to be equal to \textit{(i)}~the most likely duration of a transition and \textit{(ii)}~the average duration of all completed transitions respectively, as computed for parameters $N=50, \alpha=0.5$.}\label{fig:doubleconditioned}
\end{figure}



What Figure~\ref{fig:doubleconditioned} also shows up, however, is that even using this arguably more accurate conditioning on exact start and end points of the transitions as well as on a more realistic time scale, the average trajectory is still not an accurate representation of a typical trajectory. Marked by the white dots is one of the \emph{most likely} individual trajectories that the Markov chain passes through on its way from the initial to the final state. This trajectory is determined using the \emph{Viterbi algorithm}~\citep{Jurafsky2008}, a dynamic programming algorithm for Hidden Markov Models that allows one to infer the most likely sequence of states given a sequence of observations which only reveal partial information about the likely underlying states. The algorithm can be used for our purposes by providing it with a sequence of observations that indicate categorical usage of one variant at the start and categorical usage of the other at the end of the sequence, with a fixed number of observations representing an unspecified degree of mixed usage in between~(the source code as well as a more detailed description of the approach can be found in Appendix~\ref{app:markovmodel}).
The sequence of underlying state transitions which has the highest overall likelihood of all possible paths given these observations is one that rapidly crosses the mixed-usage area in 10-15~generations, and remains hovering at near-categorical usage of either variant for the rest of the time. (It should be noted that the exact position of this fast transition along the time axis is irrelevant, in fact all transitions \emph{parallel} to the one indicated by the dots, i.e.~ones with the same shape but actuating at earlier or later generations, have the exact same probability of occurring.)

%Even the average of these more strictly conditioned trajectories doesn't look particularly s-shaped anymore. Moreover, while shorter transitions (like ones of the length of the mode, left figure) still have the fastest rate of change at the mid-point, \emph{longer} 

To finish our study of the individual transitions generation by this model, we randomly generate a final set of transitions, limiting ourselves to only those that first complete after \emph{exactly} the specific number of generations, i.e.~we exclude ones that reach a frequency of~50 of the incoming variant early and stay there. Three such example transitions can be seen in Figure~\ref{fig:complexconditioningtransitions}.

\begin{figure}[htbp]

{\centering \subfloat[Results with learners sampling from the posterior distribution~$p(\theta|x)$.\label{fig:complexconditioningtransitions1}]{\includegraphics[width=\maxwidth]{figure/complexconditioningtransitions-1} }
\subfloat[Results with learners accepting the \emph{mean} of the posterior as their hypothesis for~$\theta$.\label{fig:complexconditioningtransitions2}]{\includegraphics[width=\maxwidth]{figure/complexconditioningtransitions-2} }

}

\caption[Three randomly generated transitions which first exhibit categorical usage of the new variant exactly after the average number of generations it takes a chain to complete a transition]{Three randomly generated transitions which first exhibit categorical usage of the new variant exactly after the average number of generations it takes a chain to complete a transition. The duration of transitions is equal to \emph{(i)}~the most likely and \emph{(ii)}~the \emph{average} duration of a transition given the parameter settings~($N=50, \alpha=0.5$).}\label{fig:complexconditioningtransitions}
\end{figure}



\subsection{Effect of sample size on the duration of transitions}

No matter what the shape of the average trajectory might be, for the sake of cross-validating the general results of the neutral evolution models as implemented here as well as by the Utterance Selection Model, we can compare the two models' predictions regarding how the expected duration of transitions develops as a function of the `population size'.

While in the Utterance Selection Model the `population size' refers explicitly to the size of the \emph{speech community}~(i.e. it is a measure of the number of interacting individuals), \citeauthor{Reali2009} are more implicit about the precise meaning of their model parameter~$N$.
In~\citet{Reali2010} they show that a chain of learners employing a specific sample size~$N$ that accepts the average of the posterior as their hypothesis for the underlying frequency~$\theta$ is identical to the Wright-Fisher model of neutral evolution with symmetric mutation rates, an equivalence that will be discussed more in-depth in Section~\ref{sec:realigriffithsequivalence}.
Taking the equivalence of these two models literally would mean that the parameter~$N$ in the present model corresponded to the population size of a group of Bayesian learners, each of which uses either of the variants categorically, with the probability of adopting either variant given by~$\hat{\theta}$.
% TODO unpack

Another way to construe the meaning of parameter~$N$ corresponds to how it is mapped onto an Iterated Learning experiment on humans in~\citet{Reali2009}. Here, the model is fit to a chain of single individuals, each of which first receives and then produces a sample of $N$~tokens. While not exactly specifying a feature of the individual, the function that $N$ fulfills in this context is to control the resolution at which the data is presented to and produced by individual participants in the chain. In this sense, the parameter fulfills a function very similar to the $T$ parameter of the Utterance Selection Model described above.

On the other hand, the fact that the model does not allow for continuous updating of the internal representations once they are acquired, but is instead based on a one-time learning event of sample size~$N$, means that the set of possible posterior distributions~$p(\theta|x)$, as well as the resolution of possible values of $\theta$ for strategies that adopt one value deterministically, is completely constrained by~$N$. % The size of learning outcomes is finite / But, due to the one-off nature of learning in an Iterated Learning model
As a consequence, the parameter inadvertently acts as something like a \emph{memory capacity} of the individual which, unlike the USM's sample resolution~$T$, also limits the individual agents' representational resolution of the frequency distribution they are trying to acquire. % a strange convolution of variable frequency, memory capacity

\begin{figure}[htbp]

{\centering \includegraphics[width=\maxwidth]{figure/realipopulationsize-1} 

}

\caption[Mean and mode of the duration of transitions as a function of the parameter~]{Mean and mode of the duration of transitions as a function of the parameter~$N$, with $\alpha=0.5$.}\label{fig:realipopulationsize}
\end{figure}



Whichever way the parameter is to be construed, its setting does not just affect the likelihood of transitions occurring, but also the transitions' duration and shape. The parameter's effect on the average as well as most likely \emph{duration} of completed transitions in chains of learners is shown in Figure~\ref{fig:realipopulationsize}. In all cases, $N$~shows a linear relationship with the time until fixation for all measures with varying slopes, a result that is in line with findings for expected diffusion times obtained from other general models of neutral evolution~\citep{Kimura1969}. %Blythe2007divided

%Even under this very different type of model we can reproduce the results from the neutral evolution regime of the Utterance Selection Model as discussed in Section~\ref{sec:usmneutral}: the duration of transitions from (near)-categorical usage of one variant to another under neutral evolution, where neither variant has any selection bias acting in their favour, increases linearly with the size of the population.

\subsection{Summary}

To complement the study of various different replication regimes implement in the USM framework earlier, I presented a replication of \citeauthor{Reali2009}'s Markov model of neutral evolution with symmetric innovation, a model that has been used to make concrete claims about the possibility of s-shaped transitions in the absence of asymmetry between variants~\citep{Reali2010}.
However, neither the dynamics of individual transitions, nor a closer investigation of the \emph{average} trajectories under different conditioning assumptions suggests that this model of neutral evolution based on regularising Bayesian learners exhibits curves that are particularly directed, instead producing noisy transitions with frequent reversals and restarts.
Also, in agreement with other models of neutral evolution, the expected duration of a transition from categorical use of one variant to categorical use of another increases linearly with the population/memory size parameter.
Another important conclusion regarding modelling more generally is that, when one is interested in the \emph{temporal dynamics} of a system it is indispensable to look not only at the end states or average dynamics as a shortcut, but that a more exhaustive analysis of the actual dynamics and \emph{typical} transitions is required.


%\section{A conclusion on modelling}

\section{Trend-amplification and momentum-based selection}

The literature summarised above in Chapter~\ref{ch:review} as well as the pressures investigated in this chapter cover the bulk of the established accounts and theories about language change.
While the direct comparison of different pressures in the Utterance Selection Model just recapitulated indicates that an asymmetry between variants is necessary to account for the directed trajectories found in language change, a survey of the literature shows that there is no universal agreement on where exactly those asymmetries should be found. Adaptive pressures, most of which can be characterised as being language-internal, provide good explanations of the macro-level patterns of language change found cross-linguistically, but they are subject to a methodological flaw known as the \emph{actuation problem}. \index{actuation!actuation problem}
While general, universal pressures can be invoked to account for universal properties of human languages, they fail to explain why some linguistic features are only selected for occasionally in specific languages, thus leading to the actuation of particular language changes.
In other words, none of the selection mechanisms investigated so far offer a theory of how asymmetries could emerge spontaneously and sporadically, as appears to be the case in language change.
The remainder of this thesis is dedicated to the exploration of a relatively novel selection pressure that was briefly alluded to previously, namely that of \emph{trend amplification} by the individual. In the spirit of an exhaustive model comparison, the next chapter will investigate the dynamics of such a mechanism, implemented as a \emph{momentum-based} selection pressure within the Utterance Selection Model.

%\citep{Labov2001,Gureckis2009}
