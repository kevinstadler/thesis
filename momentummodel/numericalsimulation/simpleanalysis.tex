\documentclass[a4paper]{article}
\usepackage{lmodern}
\usepackage{amssymb,amsmath}
\usepackage{ifxetex,ifluatex}
\usepackage{fixltx2e} % provides \textsubscript
\ifnum 0\ifxetex 1\fi\ifluatex 1\fi=0 % if pdftex
  \usepackage[T1]{fontenc}
  \usepackage[utf8]{inputenc}
\else % if luatex or xelatex
  \ifxetex
    \usepackage{mathspec}
    \usepackage{xltxtra,xunicode}
  \else
    \usepackage{fontspec}
  \fi
  \defaultfontfeatures{Mapping=tex-text,Scale=MatchLowercase}
  \newcommand{\euro}{€}
\fi
% use upquote if available, for straight quotes in verbatim environments
\IfFileExists{upquote.sty}{\usepackage{upquote}}{}
% use microtype if available
\IfFileExists{microtype.sty}{%
\usepackage{microtype}
\UseMicrotypeSet[protrusion]{basicmath} % disable protrusion for tt fonts
}{}
\usepackage[margin=1in]{geometry}
\ifxetex
  \usepackage[setpagesize=false, % page size defined by xetex
              unicode=false, % unicode breaks when used with xetex
              xetex]{hyperref}
\else
  \usepackage[unicode=true]{hyperref}
\fi
\hypersetup{breaklinks=true,
            bookmarks=true,
            pdfauthor={Kevin Stadler},
            pdftitle={Momentum-based language change},
            colorlinks=true,
            citecolor=blue,
            urlcolor=blue,
            linkcolor=magenta,
            pdfborder={0 0 0}}
\urlstyle{same}  % don't use monospace font for urls
\usepackage{graphicx,grffile}
\makeatletter
\def\maxwidth{\ifdim\Gin@nat@width>\linewidth\linewidth\else\Gin@nat@width\fi}
\def\maxheight{\ifdim\Gin@nat@height>\textheight\textheight\else\Gin@nat@height\fi}
\makeatother
% Scale images if necessary, so that they will not overflow the page
% margins by default, and it is still possible to overwrite the defaults
% using explicit options in \includegraphics[width, height, ...]{}
\setkeys{Gin}{width=\maxwidth,height=\maxheight,keepaspectratio}
\setlength{\parindent}{0pt}
\setlength{\parskip}{6pt plus 2pt minus 1pt}
\setlength{\emergencystretch}{3em}  % prevent overfull lines
\providecommand{\tightlist}{%
  \setlength{\itemsep}{0pt}\setlength{\parskip}{0pt}}
\setcounter{secnumdepth}{0}

%%% Use protect on footnotes to avoid problems with footnotes in titles
\let\rmarkdownfootnote\footnote%
\def\footnote{\protect\rmarkdownfootnote}

%%% Change title format to be more compact
\usepackage{titling}

% Create subtitle command for use in maketitle
\newcommand{\subtitle}[1]{
  \posttitle{
    \begin{center}\large#1\end{center}
    }
}

\setlength{\droptitle}{-2em}
  \title{Momentum-based language change}
  \pretitle{\vspace{\droptitle}\centering\huge}
  \posttitle{\par}
  \author{Kevin Stadler}
  \preauthor{\centering\large\emph}
  \postauthor{\par}
  \predate{\centering\large\emph}
  \postdate{\par}
  \date{Fri Nov 13 13:03:31 2015}


% Redefines (sub)paragraphs to behave more like sections
\ifx\paragraph\undefined\else
\let\oldparagraph\paragraph
\renewcommand{\paragraph}[1]{\oldparagraph{#1}\mbox{}}
\fi
\ifx\subparagraph\undefined\else
\let\oldsubparagraph\subparagraph
\renewcommand{\subparagraph}[1]{\oldsubparagraph{#1}\mbox{}}
\fi

\begin{document}
\maketitle

\subsection{Simple numerical investigation of the
transitions}\label{simple-numerical-investigation-of-the-transitions}

Let us begin with a simple criterion of what a `transition' is. Assuming
a threshold (say 5\%), we can define the two extreme areas where the
mean population usage level of the minority variant is below this
threshold as the two regions of `near-categorical use' of either
variant. A transition, then, is the period in which the mean usage
levels of the population cross from near-categorical use of one to
near-categorical use of the other variant. Taking our grand total of
\(124416\) runs, which were all initialised with mean usage levels
between 1 and 3\%, we find that only 37094 (\textasciitilde{}30\%) ever
exceeded the 5\% mark. And of those only an even smaller fraction,
namely 16285 (\textasciitilde{}13\%) ever exhibited a transition
(i.e.~crossed over into near-categorical usage of the other variant).

\begin{figure}[htbp]
\centering
\includegraphics{simpleanalysis_files/figure-latex/ttransitions-1.pdf}
\caption{Successful transitions generated for 4 settings of the sample
resolution parameter \(T\). The remaining parameters were held fixed at
\(a=0.01\), \(N=10\), \(b=2\), \(m=2\). The total number of attested
transitions for each setting of \(T\) is 21, 34, 65, 50, with 144 runs
per condition.}
\end{figure}

This surely tells us something.

\begin{figure}[htbp]
\centering
\includegraphics{simpleanalysis_files/figure-latex/mtransitions-1.pdf}
\caption{Successful transitions generated for 6 settings of parameter
\(m\). The remaining parameters were held fixed at \(a=0.01\), \(N=10\),
\(b=2\), \(T=5\). The total number of attested transitions for each
setting of \(m\) is 51, 50, 78, 91, 97, 89, with 144 runs per
condition.}
\end{figure}

This surely tells us something.

\begin{figure}[htbp]
\centering
\includegraphics{simpleanalysis_files/figure-latex/btransitions-1.pdf}
\caption{Successful transitions generated for 6 different settings for
the momentum bias strength \(b\). The remaining parameters were held
fixed at \(a=0.01\), \(N=10\), \(m=2\), \(T=5\). The total number of
attested transitions for each setting of \(b\) is 0, 6, 37, 58, 50, 94,
with 144 runs per condition.}
\end{figure}

This surely tells us something.

\begin{figure}[htbp]
\centering
\includegraphics{simpleanalysis_files/figure-latex/ntransitions-1.pdf}
\caption{Successful transitions generated for 6 different population
sizes \(n\). The remaining parameters were held fixed at \(a=0.01\),
\(N=10\), \(b=2\), \(m=2\), \(T=5\). The total number of attested
transitions for each setting of \(n\) is 70, 78, 50, 63, 59, 66, with
144 runs per condition.}
\end{figure}

The neutral evolution transitions are \emph{really} rare. Here's
\emph{all} of them:

\includegraphics{simpleanalysis_files/figure-latex/neutralevolutiontransitions-1.pdf}
\includegraphics{simpleanalysis_files/figure-latex/interruptions-1.pdf}
\includegraphics{simpleanalysis_files/figure-latex/interruptionsperparameter-1.pdf}

\begin{verbatim}
## [1] 0.5510204
\end{verbatim}

\begin{verbatim}
## [1] 0.965966
\end{verbatim}

\begin{figure}[htbp]
\centering
\includegraphics{simpleanalysis_files/figure-latex/interruptionsamples-1.pdf}
\caption{Example of a successful and an interrupted transition created
in runs with identical parameter settings: N=2, b=2.5, T=4, x0=0.03,
i=1, a=0.01, m=4}
\end{figure}

\begin{verbatim}
## [1] 1
\end{verbatim}

\begin{figure}[htbp]
\centering
\includegraphics{simpleanalysis_files/figure-latex/largeninterruptions-1.pdf}
\caption{N=30,b=1.5,T=2,x0=0.01,i=0,a=0.01,m=4}
\end{figure}

\begin{verbatim}
## [1] 1
\end{verbatim}

\end{document}
