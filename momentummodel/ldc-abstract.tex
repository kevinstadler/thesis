Like other socially transmitted traits, human languages undergo cultural evolution. While humans can replicate linguistic conventions to a high degree of fidelity, sometimes established conventions get replaced by new variants, with the gradual replacement following the trajectory of an \emph{s-shaped curve}. Although previous modelling work suggests that only a bias favouring the replication of new linguistic variants can reliably reproduce the dynamics observed in language change, the source of this bias is still debated. In this paper we compare previous accounts with a \emph{momentum-based selection} account of language change, a replicator-neutral model where the popularity of a variant is modulated by its \emph{momentum}, i.e. its \emph{change in frequency of use} in the recent past. We present results from a multi-agent model that are characteristic of language change, in particular by exhibiting spontaneously generated s-shaped transitions that do not require externally triggered actuation. We discuss several empirical questions raised by our model, pertaining to both momentum-based selection as well as other biases and pressures which have been suggested to influence language change.
