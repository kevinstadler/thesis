\section{Modelling the interaction of different pressures}
\label{sec:asymmetricvariation}

%One argument put forward in the previous sections was that, in order to explain asymmetries, it is not necessary to construe the prevalent conventions as having been selected for for that purpose.

In this section I will present a simple model to investigate how a symmetric momentum bias, a selection bias favouring trending conventions whether they are beneficial or deterious, interacts with asymmetric \emph{generation} of variants. The model is an extension of the Markov model of Bayesian Iterated Learning that was analysed in-depth in Section~\ref{sec:realigriffiths}.\index{Markov model}

To recapitulate, \citet{Reali2009} proposed a model of regularisation by Iterated Learning. In their model, a Bayesian learner infers the underlying relative frequencies of several competing variants based on a sample of productions they observe from a teacher. By iteratively passing one learner's output on to the next and analysing the \emph{stationary distribution} of this Markov chain, they showed that by setting the parameter~$\alpha$ to values that lead individual learners to slightly increase the variability of their own productions, the chains end up \emph{regularising} the input distribution: over time, the population ends up mostly producing input distributions where one of the competing variables is used categorically.

Moreover, in~\citet{Reali2010} the authors showed that under some circumstances this model of Bayesian Iterated Learning is equivalent to the Wright-Fisher model of biological evolution with mutation~\citep[see e.g.][]{Hartl2007}.
The Wright-Fisher model is a general tool from population genetics which is used to predict the expected change in the frequency of competing alleles in a population in the presence or absence of different pressures such as \emph{mutation} and \emph{selection}.

%given specific ~(henceforth referred to as \emph{innovation}) probabilities

\subsection{The Wright-Fisher model}\index{Wright-Fisher model}
\label{sec:wrightfisher}
In its very simplest form the Wright-Fisher model~\citep[for its original formulation see][]{Wright1931} describes the dynamics of competition between two variants in a finite population of size~$N$.
%TODO The case that is closest to cultural evolution is one where every individual only possesses one of the variants~(a \emph{haploid} population in biological terms). ~(akin to annual plants)
Call the two variants 0 and 1 and their respective absolute frequencies $n_0$ and~$n_1$. The entire population of~$N$ individuals is replaced at discrete time steps so that it is always the case that $n_0 + n_1 = N$. To simplify notation and assuming a constant population size~$N$, we can again drop the indices referring to individual variants: we will henceforth write $n_t$ to mean $n_1$ at time \emph{t}, from which the frequency of the competing variant can be trivially computed.

The state of the population can be described simply by the relative frequency~$x$ of that variant,~i.e.
\begin{equation}
x_t = \frac{n_t}{N}\;.
\end{equation}

The Wright-Fisher model assumes that the individual generations of the population as it evolves over time are \emph{non-overlapping}. In other words, the generation makeup at the following timestep, $n_{t+1}$, is determined by creating a new population of~$N$ tokens, each generated by replicating a randomly selected `ancestor' from the previous generation.
The probability distribution over the likely frequencies~$n_{t+1}$ in the next generation is consequently distributed according to a Binomial distribution

\begin{equation}
\label{eq:binomialsampling}
n_{t+1} \sim Bin(N, f(x_t))\;.
\end{equation}

In the absence of any other pressures, the probability of replicating an instance of a particular variant is simply equivalent to the relative frequency of that variant in the previous generation, i.e.
\begin{equation}
f(x) = x\;.
\end{equation}

The behaviour of this simplest form of the Wright-Fisher model describes the dynamics of completely neutral drift which was not only fundamental in informing the \emph{neutral theory} of genetic evolution~\citep{Kimura1968,Kimura1983}, but still forms an important baseline for evaluating the presence or absence of evolutionary pressures in empirical data sets to this day. % TODO reference?

Without any mechanisms that influence the selection of existing variants or mutations that would introduce new ones, this is a very simple model of the \emph{diffusion} of traits through replication. Before we turn to more complex versions of the Wright-Fisher model which incorporate the influence of mutation and selection pressures, it is insightful to have a closer look at the similarity between diffusion models from biology and the Utterance Selection Model as described in Section~\ref{sec:usm}.\index{diffusion}

\subsubsection{Relationship to the Utterance Selection Model~(USM)}

The similarity between Equation~\ref{eq:binomialsampling} and the data production function of the USM in Equation~\ref{eq:usmsampling} is not just superficial: in its most general formulation the Wright-Fisher model is in fact identical to the trivial case of a single USM agent with a learning rate~$\alpha=1$ and USM sample resolution~$T=N$ that is engaged in a production-perception loop.\index{production-perception loop}

More fully-fledged versions of the USM with populations of speakers also find parallels in more complex models of biological evolution. \citet{Blythe2007divided} showed how the learning and alignment dynamics of the USM are in fact identical to Wright's \emph{island} model, an extension of the simple Wright-Fisher model above that is used to study the diffusion of variants between subdivided populations with limited migration between them~\citep{Wright1943}.

While models of pure diffusion in complex population structures are interesting in themselves and many results concerning fixation times and fixation probabilities can be derived from them analytically~\citep[see e.g.][]{Imhof2006,Baxter2008,Blythe2012copying,Michaud2016}, we shall return to the simpler model of just one population with fixed turnover, which greatly simplifies the analysis of different innovation and selection pressures that we are particularly interested in here.

%While the diffusion model interesting in itself as a baseline or \emph{null model} of change, we are currently more interested in the different ways in which \emph{biases} or \emph{pressures} can act on a population of replicating units.

\subsubsection{The Wright-Fisher model with mutation}\index{innovation}
\label{sec:realigriffithsequivalence}

% best reference it seems: http://numerical.recipes/whp/notes/wrightfisher.pdf

Under the Wright-Fisher model with \emph{mutation}~(henceforth referred to more generally as \emph{innovation}), the probability of producing an instance of variant~$0$ at the next generation is not completely equivalent to its current relative frequency~$x$. Instead, with probability~$\mu_{1}$, one of the $n_0$ type~0 variants present in the population will spontaneously mutate into an instance of variant~1 during its replication. Conversely, there is a probability of~$\mu_{0}$ that one of the instances of the other variant~(of which there are $n_1$) will spontaneously mutate into a variant~0. Under this assumption, the relative probability of producing an instant of variant~1 at the next generation is
\begin{equation}\label{eq:mutation}
f(x) = x\cdot(1-\mu_0) + (1-x)\cdot\mu_1 \;.
\end{equation}

It is this version of the Wright-Fisher model that is mathematically equivalent to \citeauthor{Reali2010}'s model of Bayesian inference, in particular to the version of the model where the learners derive a hypothesis from their input sample of size $N$ by deterministically adopting the mean~$\hat{\theta}$ of the posterior distribution~$p(\theta|x)$ as their production probability. In this case, the learner's production behaviour is identical to that of a population of~$N$ individuals who are replicating with innovation rates set to
\begin{equation}\label{eq:realigriffithsequivalence}
\mu_0 = \mu_1 = \frac{\alpha}{2\cdot(\alpha+N)}\;.
\end{equation}

%The fact that equivalence can only be established as a function of both of the Bayesian model's parameters is slightly awkward, as it means that, to achieve the same regularisation behaviour~(by means of setting~$\alpha$) in a population, 
% according to both the regularisation parameter~$\alpha$ as well as the population size~$N$,

Reframed in terms of biological evolution, the values of the Bayesian model parameter that are associated with regularisation behaviour,~$\alpha < 1$, all satisfy the limit of low mutation, $\mu \ll 1/N$~\citep{Tarnita2014}.

%This equivalence comes in the wake of other studies which have highlighted that evolution by natural selection can be seen as an inference dynamics~\citep{Harper2009}.

The Wright-Fisher model with innovation allows for pressures in favour of specific variants, by setting the innovation rates $\mu_1\ne\mu_0$.
But it does not support \emph{selection} of variants, as would be needed for a replicator or momentum-based selection bias.

%In particular, by setting only one of the rates to 0 one can simulate the accumulation of `noise'

\subsubsection{The Wright-Fisher model with innovation and selection}

To allow selection on top of innovation, we have to move on to the Wright-Fisher model with innovation and selection, which takes the form

\begin{equation}\label{eq:mutationselection}
f(x) = \frac{x\cdot(1+s)\cdot(1-\mu_0) + (1-x)\cdot\mu_1}{x\cdot(1+s) + 1 - x} \;.
\end{equation}

The parameter $s\ge 0$ in this equation represents a \emph{selection coefficient} which causes the $n_1$~tokens of variant~$1$ that are present in the population to be preferentially selected, i.e.~their likelihood of replication is increased at the expense of the competing variant. In contrast to the innovation pressures~$\mu$, the effectiveness of the selection coefficient depends on the current prevalence of the selected variant in the population. In particular, the selection pressure in favour of an advantageous variant is completely ineffective as long as no tokens of that variant are present in the population. In other words, the selection coefficient cannot introduce new variants. This can be appreciated best when we consider the force of selection in isolation, by setting the probability of randomly generating new variants $\mu_0=\mu_1=0$. In this case we arrive at the Wright-Fisher model with selection only,

\begin{equation}\label{eq:selection}
%f(x) = \frac{x\cdot(1+s)}{x\cdot(1+s) + 1 - x} \;.
f(x) = \frac{x+s\cdot x}{1+s\cdot x} \;.
\end{equation}

\subsection{Wright-Fisher model dynamics for infinite population size}

At this point we can already get a glimpse of how these two different types of pressures differ in their dynamics as well as how they interact. Assuming an infinite population size, random sampling effects are completely washed out and we recover an idealised image of the dynamics of the Wright-Fisher model.

\subsubsection{The dynamics of innovation}

Figure~\ref{fig:mutationselection} shows the relative impact of innovation and selection under the assumption of an infinitely large population size. The graphs plot the relative change~$\Delta x$ to the frequency of one of the variants as a function of its current prevalence~$x$ in the population. As can be seen from panels~(i) and~(ii) in Figure~\ref{fig:mutationselection1} the impact of innovation, as measured by how much it affects the current frequency of variants, is stronger when there are fewer tokens of the innovated variant in the population. More than anything else, innovation is a pressure \emph{away} from homogeneous population states.

\subsubsection{The dynamics of selection}

The dynamics of selection look very different: from Figure~\ref{fig:mutationselection1}(iii) we can see that selection is ineffective as long as there are no instances of the selected-for variant present, and it acts strongest when there is \emph{most} variation -- in the case of two competing variants this is when both variants are equally represented in the population.

The panels underneath in Figure~\ref{fig:mutationselection2} demonstrate how these pressures affect the distribution of variants over time in the limiting case of an infinitely large population, an idealised condition in which the replication dynamics are unaffected by the noise of random sampling. Asymmetric innovation pressures lead to rapid spread of the frequently innovated variant initially, but the growth then tails off towards some asymptotic frequency, which corresponds to the points in the upper panels where the line indicating the relative change to the variant frequency crosses the y-axis, i.e.~where~$\Delta x=0$.

In contrast to this \emph{r-shaped} growth, for the temporal dynamics of the selection pressure shown in Figure~\ref{fig:mutationselection2}(iii) we recover the logistic growth pattern that we also found for the replicator selection regime of the Utterance Selection Model in Section~\ref{sec:usm}. \index{logistic growth}
Also, unlike innovation pressures, the function indicating the relative impact of selection in Figure~\ref{fig:mutationselection1}(iii) crosses the $\Delta x=0$ line twice, meaning that the selection regime possesses two stable, asymptotic states, at $x=1$ as well as $x=0$. Unlike the innovation pressure, the s-shaped growth pattern cannot escape the variationless state at $x=0$ but requires some low level of variation in order to `kick in'.



\begin{knitrout}
\definecolor{shadecolor}{rgb}{0.969, 0.969, 0.969}\color{fgcolor}\begin{figure}[htbp]

{\centering \subfloat[Influence of innovation and selection pressures as a function of the current population makeup. \emph{(i)}~innovation of one variant only, $\mu_1=0.005$. \emph{(ii)}~innovation of both variants at different rates, $\mu_1=0.005, \mu_0=0.001$. \emph{(iii)}~selection only, $s=0.02$.\label{fig:mutationselection1}]{\includegraphics[width=\maxwidth]{figure/mutationselection-1} }
\subfloat[Temporal development of the population given the corresponding pressures from the panels directly above, assuming initial states $x=0.01$ and $0.99$.\label{fig:mutationselection2}]{\includegraphics[width=\maxwidth]{figure/mutationselection-2} }

}

\caption[Dynamics of innovation and selection in the Wright-Fisher model]{Dynamics of innovation and selection in the Wright-Fisher model.}\label{fig:mutationselection}
\end{figure}


\end{knitrout}

\subsubsection{The interaction of innovation and selection}\index{replicator selection}\index{pressures!interaction of}

Now that we have an idea of when the different types of pressures of innovation and selection affect the evolution of a system most strongly, we can go on and ask how the two interact, i.e.~what happens when both pressures apply simultaneously?

Assuming the same constant selection pressure as above, there are still three different scenarios of interaction to consider. Figure~\ref{fig:mutationselectioninteraction} shows the infinite population size dynamics assuming (i)~selection of a variant that is also preferentially innovated, (ii)~selection on top of \emph{unbiased}~(symmetric) innovation, and (iii)~selection \emph{against} a preferentially innovated variant.

In all cases we can see that, while innovation of the selected for variant helps it spread more quickly open first introduction, innovation of the competing variant actually stops the selection pressure from taking over the entire population. Only as the innovation rates~$\mu$ diminish towards zero do we recover the logistic growth dynamics of selection that we saw above, which lead to complete dominance by the preferred variant in the limit.

%Crucially, the current analyses all rely on the assumption of an infinite population size, an idealisation which is not subject to the effects of random sampling. In order to see how robust the two pressures are in finite populations, as well as how their interaction works out in that case, 
While this simple analysis gives us an idea of when innovation and selection are expected to impact most strongly on an evolving systems dynamics', understanding the \emph{interaction} of these different types of pressures in finite populations requires more in-depth study for which we will again turn to the Markov model framework.

\begin{knitrout}
\definecolor{shadecolor}{rgb}{0.969, 0.969, 0.969}\color{fgcolor}\begin{figure}[htbp]

{\centering \subfloat[Influence of innovation and selection pressures as a function of the current population makeup. All panels assume selection of the incoming variant at $s=0.02$. \emph{(i)}~preferred innovation of the selected for variant, $\mu_1=0.005, \mu_0=0.001$ \emph{(ii)}~symmetric innovation of both variants, $\mu_0=\mu_1=0.005$ \emph{(iii)}~antagonistic pressures, where selection competes against the preferentially innovated variant, $\mu_1=0.001, \mu_0=0.005$.\label{fig:mutationselectioninteraction1}]{\includegraphics[width=\maxwidth]{figure/mutationselectioninteraction-1} }
\subfloat[Temporal development of the population given the corresponding pressures from the panels directly above, starting at initial state $x=0$.\label{fig:mutationselectioninteraction2}]{\includegraphics[width=\maxwidth]{figure/mutationselectioninteraction-2} }

}

\caption[Dynamics of the interaction of innovation and selection pressures in the Wright-Fisher model]{Dynamics of the interaction of innovation and selection pressures in the Wright-Fisher model.}\label{fig:mutationselectioninteraction}
\end{figure}


\end{knitrout}

%In this way they are quite similar to the case of antagonistic innovation and selection, which has been of particular interest to biologists working with the Wright-Fisher model. Given a certain deleterious innovation which occurs with probability~$\mu_1$, one can determine how strong the selection pressure \emph{against} this innovation must be in order to stop the variant from spreading through the population.

%If~$x$ is the proportion of the population possessing the novel, deleterious variant. Under the assumption that the probability~$\mu_0$ of backwards-mutation to the original trait is negligible, then at every time step a fraction~$\mu_1\cdot(1-x)$ of the population will spontaneously exhibit the innovation. At the same time, the selection pressure of strength~$s$ will act to select \emph{against} any individuals who already possess the variant. The two pressures balance each other out when the term
%\begin{equation}
%\Delta x = \mu_1\cdot(1-x) - s\cdot x\cdot (1-x)
%\end{equation}
%is equal to $0$, which is the case when
%\begin{equation}
%x = \frac{\mu_1}{s}\;.
%\end{equation} % \citep{Barton2007}

%As long as the innovation rate is not greater than the selection coefficient, the Wright-Fisher model predicts that we achieve \emph{mutation-selection balance}~\citep{Spencer2001} \index{selection!mutation-selection balance}\index{mutation!mutation-selection balance}
%where the proportion of individuals having the deleterious variant to stabilise at the \emph{equilibrium frequency}~$\mu_1/s$.

\subsection{Innovation and selection in finite populations}

While the assumption of an infinite population size allows for an idealised study of the effects of innovation and selection as well as an idea of when they balance each other out, the instances of biological and cultural evolution that we can observe empirically all play out in finite populations, where changes to the distribution of variants over time are subject to random sampling effects. In order to disentangle the relative roles of the two pressures of innovation and selection in finite populations, we will make use of some of the same analytical tools that were already used to study the model of \citeauthor{Reali2009} in Section~\ref{sec:markovmodel}.

Following the same order of pressures as above, we will be investigating their dynamics using two different tools: firstly, in order to get an idea of the relative frequencies at which the competing variants are present across time we will again be looking at the \emph{stationary distributions} of the Markov chains that correspond to the Wright-Fisher models with the respective innovation and selection parameters.

Like before, we will focus on parameter combinations which lead the models to preferrably occupy states of (near-)categorical usage of either of the variants, as is the case for low mutation rates. Since we are again also interested in the nature of the \emph{transitions} between those two extreme states, we will have to move beyond merely looking at the stationary distributions, since this type of analysis generalises over the temporal aspect of the model dynamics. To get an idea of the nature of the transitions we will primarily look at the \emph{completion probabilities} of actuated changes, as measured by the probability of the incoming variant \emph{diffusing} through the entire population. To explain the concept, we will start with the simplest possible model of pure \emph{diffusion} in which both of the those pressures are completely absent.

\subsubsection{Diffusion without innovation or selection}\index{diffusion}

In the absence of any innovation of new variants, the two extreme states that correspond to categorical usage of either variant are \emph{absorbing} states. Starting off with a population that exhibits variation, repeated replication of a finite population will eventually lead one variant to \emph{diffuse} through the entire population, with all of the other competing variants being eliminated. The exact probability of either variant diffusing across the entire population depends on the initial state of the population. Figure~\ref{fig:diffusionselection1} shows the probability distribution across all model states, based on three different initial conditions. This distribution, computed through numerical simulation, shows that the probability of a variant diffusing is equal to its initial frequency, a finding that is in agreement with analytical results~\citep{Clifford1973}.

A more efficient way to look at this is by simply considering the \emph{diffusion probability} of the incoming variant given a range of initial states, all in one plot. The left panels in Figure~\ref{fig:diffusionselection2} show the diffusion probability for pure neutral drift given different starting states of the model for increasing population sizes. The probabilities lie on the diagonal of the chart in each case, showing that the result that the probability of diffusion is identical to the variant's initial relative frequency is independent of population size.

\begin{knitrout}
\definecolor{shadecolor}{rgb}{0.969, 0.969, 0.969}\color{fgcolor}\begin{figure}[htbp]

{\centering \subfloat[Stationary distribution of diffusion without selection given different initial states~$x_0$.\label{fig:diffusionselection1}]{\includegraphics[width=\maxwidth]{figure/diffusionselection-1} }
\subfloat[Diffusion probabilities for different strenghts of the selection coefficient~$s$ and population sizes~$N$. The baseline diffusion probabilities given purely neutral drift are indicated by the dotted line.\label{fig:diffusionselection2}]{\includegraphics[width=\maxwidth]{figure/diffusionselection-2} }

}

\caption[Diffusion outcomes for different population sizes ]{Diffusion outcomes for different population sizes $N$ and selection coefficients $s$.}\label{fig:diffusionselection}
\end{figure}


\end{knitrout}

\subsubsection{The dynamics of selection}\index{replicator selection}

In contrast to purely neutral diffusion, the presence of a selection bias for individual variants~(a \emph{replicator bias}) leads to a systematic increase of the selected variant as soon as it is introduced. How a selection pressure favouring the variant alters the dynamics of diffusion can be seen in Figure~\ref{fig:diffusionselection2}. In contrast to the pure diffusion case~($s=0$), a positive selection coefficient causes the relative probability of the selected for variant to diffuse to the entire population to \emph{increase} above its baseline probability which is indicated by the dotted line.
While a higher selection coefficient~$s$ increases the diffusion probability of the variant, the reliability of the selection bias in terms of guaranteeing the variant to win out depends crucially on the size of the population: the smaller the population, the more the dynamics are influenced by random sampling which, in the absence of mutation, increases the chance of driving even a positively selected for variant to extinction. Particularly for larger population sizes a sufficiently high selection coefficient can almost guarantee the diffusion of the variant, as long as it manages to avoid extinction during the fragile initial low frequency region.

\subsubsection{Symmetric innovation as (de-)regularisation}\index{regularisation}

Before considering the case of interacting pressures, we should also investigate the dynamics of the spontaneous innovation of variants acting by itself. The case of symmetric innovation, where the probabilities of spontaneously introducing either of the competing variants are equal~($\mu_0=\mu_1$), was shown to be equivalent to Bayesian Iterated Learning chains of \emph{averaging} learners by \citeauthor{Reali2010} and an in-depth quantitative analysis of the dynamics under such settings was provided in Section~\ref{sec:realigriffiths}. To briefly recapitulate, under moderately low innovation rates the stationary distribution is similar to the one of the pure diffusion model, only that the non-zero probability of randomly producing an unattested variant means that variation is never fully eliminated from the system. High innovation rates on the other hand will lead the populations to mostly consist of an even mix of all possible variants, with the exact cutoff point between the behaviours depending both on the innovation rate~$\mu$ and the population size~$N$ as well as the relationship of the two~(see Section~\ref{sec:realigriffithsequivalence}).

A summary of the different regimes of the stationary distribution is shown in Figure~\ref{fig:symmetricmutation1}. It should be noted that, given the same innovation rate~$\mu$, the shape of the stationary distribution depends on the population size~(or, in \citeauthor{Reali2009}'s framing, the size of the sample provided to the learner). Calculating the innovation rate based on a regularisation parameter~$\alpha$ which scales with the population size according to Equation~\ref{eq:realigriffithsequivalence} offers a more reliable way of picking an innovation rate that leads to the regime of stationary distributions that are of interest to us, i.e.~ones that primarily occupy states of categorical usage of either competing variant. For the remainder of the analysis we will therefore be specifying innovation rates using this parameter~$\alpha$.

\begin{knitrout}
\definecolor{shadecolor}{rgb}{0.969, 0.969, 0.969}\color{fgcolor}\begin{figure}[htbp]

{\centering \subfloat[Stationary distributions of the Wright-Fisher model under different symmetric mutation rates~$\mu_0=\mu_1>0$ and population sizes~$N$.\label{fig:symmetricmutation1}]{\includegraphics[width=\maxwidth]{figure/symmetricmutation-1} }
\subfloat[Probability of the incoming variant diffusing through the entire population as a function of the current frequency of the variant.\label{fig:symmetricmutation2}]{\includegraphics[width=\maxwidth]{figure/symmetricmutation-2} }

}

\caption[The dynamics of symmetric innovation in the Wright-Fisher model]{The dynamics of symmetric innovation in the Wright-Fisher model.}\label{fig:symmetricmutation}
\end{figure}


\end{knitrout}

In order to calculate either variant's probability of diffusion in a model with innovation, we have to adjust the definition of \emph{diffusion} slightly. Whenever there is a non-zero probability to spontaneously innovate either of the variants, neither of the model states corresponding to categorical usage of a variant are absorbing states, since even a variant that's been eliminated from the population can always be innovated anew and spread to completion. To nevertheless capture the influence of initial states, we define the diffusion probability of the incoming variant simply as the probability of first reaching a state of categorical usage of that variant, as opposed to first reaching a state where the variant is completely unattested, and we will be using this definition for the remainder of the analysis.

The relative probability of first reaching either state of categorical usage given low symmetric innovation rates is shown in Figure~\ref{fig:symmetricmutation2}. As already indicated in Section~\ref{sec:realigriffiths}, \citeauthor{Reali2009}'s model exhibits unusual behaviour for a model of regularisation in the sense that the probability of diffusing to the currently \emph{less} attested variant is actually slightly \emph{raised} above the baseline probability given by neutral diffusion, a baseline which is approached as~$\alpha\rightarrow0$.

\subsubsection{Asymmetric innovation as a model of the accumulation of errors}\index{accumulation of errors}

While symmetric innovation probabilities can be conceived of as a bias for regularisation, the probabilities can also be set unequal to introduce a bias that favours one of the variants over the other. This is an alternative way to introduce asymmetry between variants, apart from the selection bias discussed above. This configuration of the Wright-Fisher model parallels one of the earliest and simplest theories of language change discussed in Chapter~\ref{ch:review}: the assumption that a linguistic variant is accidentally `misproduced' as another some of the time, for example due to coarticulation effects, maps neatly onto a scenario with moderately low mutation rates, where the innovation probability in one direction vastly outweighs the one in the other. %~(which might in fact be 0). % ~(in the most extreme case by setting one of them to~0)

Such changes through the \emph{gradual accumulation of errors} are more typically considered for phonetic changes in a \emph{continuous} dimension, but the same process could also apply to categorical variables, such as is the case for syntactic patterns that are in competition.
While models of (syntactic) grammar competition typically consider more complex~(external) triggers for actuation, we can nevertheless study the dynamics of the spread of variants assuming that asymmetric innovation probabilities are in place in which the incoming variant is preferrably innovated, i.e.~$\alpha_1>\alpha_0$.

In order to avoid raising the preferred innovation rate to unrealistically high levels, all comparisons between different degrees of innovation asymmetries will be based on holding the preferred innovation probability $\alpha_1$ constant and setting the opposite rate~$\alpha_0$ to a diminishing fraction of the baseline rate. In this way, the degree of asymmetry can simply be expressed by the fraction~$\alpha_1/\alpha_0$, with higher numbers corresponding to greater asymmetries.

That even small asymmetric innovation probabilities can have a strong effect on the expected synchronic distribution of variants can be seen from the stationary distributions in Figure~\ref{fig:asymmetricmutationprobabilities1}, showing domination of the preferrably innovated variant to become increasingly more likely as its relative likelihood of innovation over competitor variants increases.
While this suggests that asymmetries in innovation could be just as influential as an asymmetric selection bias, we also find evidence against the strongly directed nature of individual transitions predicted under the infinite population size assumption in Section~\ref{sec:wrightfisher}. Under this simplifying assumption it was demonstrated that changes in one direction actuate instantly and are not s-shaped, but rather exhibit \emph{r-shaped} or \emph{logarithmic growth} that starts off rapidly but slows down as it approaches saturation.

%<<asymmetricmutation, fig.scap="Dynamics of the Wright-Fisher model with strong mutation in one direction", fig.cap=paste("Dynamics of the Wright-Fisher model with strong mutation in one direction ($\\mu_1=", mu[2], ", \\mu_0=", mu[1], "$) for three different population sizes~$N=", paste(Ns, collapse=", "), "$.", sep=""), fig.subcap=c(paste("Stationary distribution for different population sizes given the same asymmetric mutation rates $\\mu_0=", mu[1], "\\mu_1=", mu[2], "$.", sep=""), "Development of the state probability distribution over time, eventually converging to the stationary distributions above.")>>=
\begin{knitrout}
\definecolor{shadecolor}{rgb}{0.969, 0.969, 0.969}\color{fgcolor}\begin{figure}[htbp]

{\centering \subfloat[Stationary distribution\label{fig:asymmetricmutationprobabilities1}]{\includegraphics[width=\maxwidth]{figure/asymmetricmutationprobabilities-1} }
\subfloat[Probability of the incoming variant diffusing through the entire population as a function of the current frequency of the variant.\label{fig:asymmetricmutationprobabilities2}]{\includegraphics[width=\maxwidth]{figure/asymmetricmutationprobabilities-2} }

}

\caption[Dynamics of the Wright-Fisher model with asymmetric innovation rates]{Dynamics of the Wright-Fisher model with asymmetric innovation rates. The innovation probability of the biased for variants is held fixed at $\alpha_1=0.2$.}\label{fig:asymmetricmutationprobabilities}
\end{figure}


\end{knitrout}

However, a look at the diffusion probability of the incoming variant in Figure~\ref{fig:asymmetricmutationprobabilities2} shows that the directedness of growth derived from the infinite population size approximation above is not actually representative of the dynamics of innovation in finite populations. In contrast to asymmetries through selection, diffusion probabilities do not stray far from the neutral evolution baseline, suggesting that typical transitions are very similar to those of pure diffusion, i.e.~noisy and far from strongly directed.
While the stationary distribution shows that this model is much more likely to remain in the categorical preferred variant state for a long time, the diffusion probabilities show that it typically takes a large number of initiated transitions before one actually succeeds in spreading to the entire population, even when the variant has already managed to spread to a large part of it.
%In contrast to the selection bias~$s$ which exhibits an ever increasing impact on the diffusion probability of the preferred variant, an asymmetry in innovation does not have a strong effect,
At least as long as the innovation rates are relatively low in absolute terms, even an increase in asymmetry does not significantly raise the probability of the favoured variant diffusing without interruption.

A comparison of the dynamics of selection and innovation in Figures~\ref{fig:diffusionselection} and~\ref{fig:asymmetricmutationprobabilities} highlights that, while the expected stationary distributions of the different scenarios of asymmetric are very similar in that they predict a strong \emph{dominance} of the preferred variant, the typical trajectories of each might differ wildly.
The latter point is of particular interest since, as discussed above, the population size can interact with the effectiveness with which different asymmetric biases express themselves in the evolutionary dynamics. As a next step we can therefore analyse the effect that a \emph{bottleneck} in terms of a constrained population or sample size has on the two types of pressures as well as the interaction between the two.
% What this tells us is that, if we assume that there is a constant selection pressure in favour of one of the variants applying, then we should (almost) always only ever find this variant attested in language data that we collect synchronically.

%In reality we are dealing with finite population sizes though, where random sampling increases the impact of neutral drift on the population makeup at the expense of both innovation and selection pressures. But which pressure is affected more?
% Show asymmetric completion probabilities of changes of the two variants given different initial conditions.
% TODO plot different selection and mutation pressures and pick mutation level based on that (selection shouldn't be overpowered)

\subsection{Innovation, selection and the bottleneck in Iterated Learning}\index{Iterated Learning}\index{bottleneck}\index{pressures!interaction of}

%As can be seen in Figure~\ref{fig:mutationselectionpopulationsize},
As could be seen above, the \emph{size} of a population plays a crucial rule in how effective different kinds of pressures are. In the same way that completely neutral \emph{diffusion} of a variant is much more likely in smaller populations, %the impact of one and the same innovation rate is also relatively stronger the smaller the population is, whereas
the relative power of one and the same selection bias declines the smaller the population gets. This characterisation casts new light on the workings of the Iterated Learning Model~(ILM) that was discussed in Section~\ref{sec:ilm}, particularly on the role of the \emph{bottleneck}.
One of the core ideas of the ILM is that, by imposing a limit on the amount of information or learning data that is transmitted between generations of learners, biases become exacerbated and express themselves more quickly in the linguistic systems produced by iterated learning chains.

The purpose of this section is to attempt to shed light on the bottleneck result by teasing apart the relative impact of innovation and selection pressures under the effect of drift. The motivation for this is that, while several results showing the mathematical equivalence between Bayesian inference and specific models of biological evolution exist \citep{Reali2010,Harper2009}, there are still important conceptual differences between the two types of models. In the Bayesian approach any and all types of bias or pressure are amalgamated in the prior distribution~$p(h)$, as well as to some degree in the production probabilities~$p(d|h)$. This formulation does not allow us to distinguish between biases for innovation and selection in a principled way, as can be done in more biologically-minded frameworks such as the Wright-Fisher model.

Based on \citeauthor{Reali2010}'s demonstration of the equivalence between their iterated learning chain of \emph{averaging} learners and the Wright-Fisher model, we want to investigate the respective role that innovation and selection pressures have as a bottleneck is imposed. To do so we will compare the \emph{relative effectiveness} of different types of asymmetric biases given different values of the population/sample size parameter~$N$. %While none of the values for the parameters~$N, s, \alpha$ used here have a strong empirical motivation
%In order to be able to compare the behaviour of the model for different values of~$N$ we will use the same regularisation parameter~$\alpha$, which adjusts the innovation rate~$\mu$ according to Equation~\ref{eq:realigriffithsequivalence}~(qualitatively similar results are obtained when the same low innovation rate is used).
We will generally limit ourselves to instances of the model with low innovation rates which exhibit regularisation behaviour, i.e.~$\alpha\ll 1$.
As could be seen from the various stationary distributions above, asymmetric pressures in combination with diffusion through a finite population typically lead to situations where either one or the other variant prevails in the population. Based on these stationary distributions we can calculate the average relative frequency that the preferred variant has over the other in terms of its expected synchronic distribution,
\begin{equation}\label{eq:averagefrequency}
\hat{x} = \sum_{i=0}^N \frac{i}{N} \cdot \pi_{x=i} \;,
\end{equation}
where $\pi$ again refers to the stationary distribution %as computed from the Wright-Fisher model's corresponding Markov chain (
as defined earlier in Equation~\ref{eq:stationarydistribution}. Since we are only looking at cases with low innovation rates~$\alpha$ corresponding to regularising behaviour, the stationary distributions are bimodal, with the populations mostly remaining in regions of near-categorical use of either of the competing variants. This means that intermediate values of~$\hat{x}$ that we compute do not indicate the frequency at which the selected for variant is \emph{typically} present in the population, but rather \emph{how much of the time} that variant is used near-categorically, as opposed to its competing variant.
If the replication of the two variants was completely unbiased we would expect the average frequency to be at the~0.5 mark meaning that, across time, we would find both variants to be equally frequent. The further away the average frequency moves from this neutral mark, the more \emph{effective} that bias is at expressing itself in the expected synchronic distribution of variants.
%One way to investigate the respective role of innovation and selection pressures given different population/sample sizes is by calculating their \emph{relative effectiveness} or influence on the overall distribution of variants.
Based on this measure we can now compare how the two different types of asymmetric biases are affected by different population/sample sizes~$N$.

\subsubsection{The effectiveness of selection}

The first type of asymmetric bias considered here is that of \emph{selection}, which is implemented by setting the selection coefficient $s>0$. This asymmetric pressure applies on top of some low \emph{symmetric} innovation probability corresponding to~$\alpha/2=0.1$, which we impose to stop both variants from simply diffusing to the entire population.
As can be seen in Figure~\ref{fig:populationsizeeffectiveness}(i), in smaller populations the impact of selection pressures is increasingly \emph{reduced}, with the exact point at which the effect of the bias starts to falter depending on the magnitude of the selection coefficient. As the population size decreases towards its minimum at~$N=1$, the expected frequency of the selected for variant approaches~$50\%$, indicating no preference for either of the competing variants.

This effect can be explained based on the dynamics of selection demonstrated above: when modelling discrete~(quantitative) traits, selection pressures rely on \emph{variation} in the population and act most strongly when variation is highest. As the population size decreases the impact of neutral \emph{drift} through random sampling effects increases, diminishing the force of selection that is reliant on the relatively stable maintenance of variation.

%<<selectionstationaries, fig.cap=paste("Stationary distribution of the Wright-Fisher model with symmetric mutation and selection, $s=", s, "$ for two different population sizes.", sep="")>>=
%tightmargin(mfrow=c(2, 2), pty="s")
%for (N in Ns) {
%  for (i in c(-0.5, 0.5))
%    repl.mut.eq(N=N, m0=mu[1.5+i], m1=mu[1.5-i], b=s) %>%
%    binomialsampling.markov.matrix %>%
%    plotstationary(., ylim=0:1)
%}
%@

\begin{knitrout}
\definecolor{shadecolor}{rgb}{0.969, 0.969, 0.969}\color{fgcolor}\begin{figure}[htbp]

{\centering \includegraphics[width=\maxwidth]{figure/populationsizeeffectiveness-1} 

}

\caption[Relative frequency of variants as a function of population size given different regimes of innovation and selection pressures]{Relative frequency of variants as a function of population size given different regimes of innovation and selection pressures. \emph{(i)}~selection of symmetric innovation, $\alpha=0.01$ \emph{(ii)}~asymmetric innovation only, $\alpha_1=0.01$ \emph{(iii)}~asymmetric innovation with selection against the preferentially innovated variant, $\alpha_0=0.01$, $\alpha_1=0.0005$.}\label{fig:populationsizeeffectiveness}
\end{figure}


\end{knitrout}

\subsubsection{The effectiveness of asymmetric innovation}

It was shown above that asymmetric innovation rates which favour the spontaneous production of one of the variants can have a strong effect on their expected synchronic distribution. Figure~\ref{fig:asymmetricmutationprobabilities} already indicated that the influence of innovation is robust to changes in population size, a result that is also borne out by the present effectiveness measure. Rather than being affected by smaller population sizes, Figure~\ref{fig:populationsizeeffectiveness}(ii) shows that systems tend to converge towards some consistent low usage level of the less frequently innovated variant, with that usage frequency dependent on the two innovation rates, but not population size\footnote{Identical results are obtained when using a set mutation rate~$\mu$ rather than the population size-dependent regularisation rate~$\alpha$, assuming the same asymmetry between innovation of the two variants. While the same absolute level of $\mu$ leads to more temporal instability in smaller populations, the mean value of the stationary distributions only depends on the ratio between the two innovation rates.}. In other words, unlike with selection, asymmetric pressures due to biased \emph{innovation} appear to be robust to bottlenecks.

%Based on this measure we can now investigate the effectiveness of innovation and selection biases given different population or sample sizes~$N$. Figure~\ref{fig:populationsizeeffectiveness} shows the average frequency of the biased-for variant as calculated from the stationary distribution.

\subsubsection{The interaction of innovation and selection}

While we have discovered a change in the effectiveness of selection pressures for reduced population sizes, neither of the pressures taken individually has seen an \emph{increase} in how much it is expressed synchronically, so how can the bias-amplifying effect observed in iterated learning chains be explained? Figure~\ref{fig:populationsizeeffectiveness}(iii) shows the expected frequency of variants based on pressures of innovation and selection that work \emph{antagonistically}: while the selection coefficient favours the incoming variant at the rates indicated, there is also an underlying asymmetry in the generation of variants that preferrably innovates the competing variant.
%with~$\alpha_0=2*alphahalved$ while~$\alpha_1=2*alphahalved/divs[1]$.
In this configuration we see an inversion of the frequency distribution at a population size that is again dependent on the strength of selection. Where for larger population sizes the selection pressure prevails, the innovation pressure is more robust and thus wins out whenever a sufficiently small bottleneck is imposed.

The present analysis of the relative force of innovation and selection pressures sheds the amplification of biases in iterated learning in a new light: adopting \citeauthor{Reali2009}'s equivalence between the Wright-Fisher model and iterated learning chains, the imposition of a bottleneck corresponds to a reduction of population size in the biological sense. Based on our characterisation of the effectiveness of different asymmetric pressures in the Wright-Fisher model I conclude that, at least when it comes to the general interpretation of bottlenecks causing an amplification of biases, the biases that see themselves relatively amplified during iterated learning experiments are more likely of the \emph{innovation}, rather than the \emph{selection} type, which see a decrease in their impact. % They are conventions that are more likely to be invented.

Most studies in the Language as a Complex Adaptive System tradition are based on the assumption that biases which become exacerbated in simulation and experimental settings are of the \emph{selection} type~(e.g.~\citealt[ch.6]{Kirby1999}, \citealt{Chater2010evolution}), without taking into account the effect that other evolutionary pressures, in particular innovation and drift, have on the systems under investigation~\citep[see][p.127-9]{Henrich2008}.

%However, the present analysis does not contradict all iterated learning results
Recall that the original finding of iterated learning models was that the bottleneck triggers \emph{generalisation}, i.e.~the emergence of new items that are consistent with other known elements of a production system. In situations where learners have to infer productive \emph{systems}, such as reconstructing recursive language~\citep{Kirby2002} or function learning~\citep{Griffiths2013}, %on the micro-level of individual signs or utterance,
the pressure that is amplified is one towards the systematic \emph{innovation} of new signals.
Especially when extrapolating the role of the bottleneck to the acquisition of single, holistic traits, our analysis predicts that the effect seen in experimental work is unlikely to be an amplification of selection, but rather the \emph{relaxation} of other selective pressures that hold in real life language use, causing asymmetries that are due to preferential innovation biases to prevail.

At this point two more caveats regarding modelling are in order: firstly, it should be noted that this result regarding the ineffectiveness of selection pressures does not just hold for asymmetric selection as implemented by the Wright-Fisher model's selection coefficient~$s$. In other words, the analysis does not rely on the unusual configuration of directly antagonistic innovation and selection pressures. \emph{Symmetric} selection pressures, such as the frequency-dependent regularisation biases proposed for the USM in Section~\ref{sec:usm} as well general confirmity pressures, are equally affected by the imposition of a bottleneck. So while iterated learning can help reveal human biases in directed innovation, the methodology does not so far speak to how these innovations manage to overcome conformity pressures to help them spread through real language communities.

Secondly, the present analysis of the dynamics of selection is based on the discrete replicator view implied by evolutionary approaches such as the USM proposed by \citet{Croft2000}.
%The present analysis therefore applies specifically to the micro-level of individual utterances and conventions. If one were to adopt a higher level of analysis, in particular one investigating entire linguistic \emph{systems}
%Again, the conclusion of which type of pressure we are looking at depends on the level of the relative level of analysis that we take on the system under investigation.
% at least in the strict sense of evolution as `change by replication'.
% it's ok to take macro-level view, but just because a description is right on the macro-level doesn't mean it's correct, it could just be generalising/summing over a lot of interactions on the lower level that are of a completely different type. having a successful explanation at the low level that predicts the higher level is superior!
It should be noted that the nature of selection in this framework, as a process that merely favours the replication of instances of a variant that are already present in the population, is not the only possible way to frame selection.
In fact, the model predictions might look very different if we considered \emph{continous} rather than \emph{discrete} replicators~\citep[e.g.][akin to the modelling of \emph{quantitative traits} in biological evolution]{Wedel2006}, or if we considered the effect of \emph{guided variation}, a cultural selection pressure proposed by~\citet[pp.136]{Boyd1985} that is independent of the amount of variation present in the population. \index{guided variation}

So while other analyses are possible, which of these theoretical models of selection comes closest to reality is a matter that can only be established empirically, a task for which an effort must be made to more systematically disentangle the mechanisms of innovation and selection in experimental work on humans. % In the absence of strong empirical evidence for either model,
The approach in this chapter was taken both because it is based on a concrete model of iterated learning~\citep{Reali2009}, but also because it offers a clear explanation of the exact nature of the asymmetric biases that are amplified by a bottleneck, whether that bottleneck is construed in terms of limited learning input sample or population size.

\section{Momentum-based selection in the Wright-Fisher model}\label{sec:asymmetricmomentum}

The analysis of the Wright-Fisher model with different asymmetric pressures above confirmed the theoretical criticisms presented earlier: while asymmetries due to either accumulation of error or selection express themselves in the synchronic distribution of traits as expected, the investigation of the diffusion probabilities indicated that, under most conditions, selection pressures would not allow dispreferred traits to spread, while innovation pressures by themselves would only ever exhibit noisy trajectories far from the directed transitions we normally see in language change.
In other words, universal innovation and selection alone seem \emph{too strong} a predictor to account for particular, as opposed to universal, language features, as I argued for in the the discussion of the actuation problem in Section~\ref{sec:actuationproblem}.

The momentum-based selection model presented in Chapter~\ref{ch:momentummodel} on the other hand promised to account for the sporadic nature of language changes, as well as their spontaneous actuation. The multi-agent model presented in the earlier chapter considered \emph{selection} only, without taking into account the effect that \emph{innovation}, and in particular \emph{asymmetries} in innovation probabilities, might have on the macro-level dynamics of the model, such as the expected synchronic distribution of competing traits. To flesh out the predictions of these interactions, I will therefore present a modification of the Wright-Fisher model that incorporates a simplified version of the momentum-based selection mechanism.

\subsection{A Markov model state space for momentum}

So far, we have used the Wright-Fisher model to explore the quantitative dynamics of the well-known pressures of innovation and selection, both of which have direct parallels in biological evolution. The logical next step is to investigate how these results compare to a trend-amplifying bias such as the one implemented by momentum-based selection. On the face of it, the idea of momentum and the assumptions of a Markov model discussed in Section~\ref{sec:markovmodel} seem at odds: to reiterate, the Markov assumption states explicitly that the probability of transitioning into a particular state must only be influenced by a system's current state, not by any previous states or state trajectories.

The Markov model framework itself is oblivious to the structure of a model's state space and the `meaning' of individual states in terms of how they are interpreted by the modeller. In order to represent a population of $N$ individuals (or memory size of~$N$ tokens), \citeauthor{Reali2009} constructed a space of~$N+1$ states. Just looking at the level of connectivity between states, this state space might seem unstructured: all states had non-zero (if very small) transition probabilities to each other, forming one fully connected graph. But on top of this homogeneous structure there was a semantically meaningful organisation of states: each state corresponded to a certain memory state of an agent, with corresponding transition probabilities corresponding to the `proximity' to the other states in terms of the usage rates they represented, as shown in Figure~\ref{fig:markovstatespacesimple}.

Starting from this basic pattern of connectivity, we can construct a Markov model that lends a sense of time to the model state space.
In order to augment the with a momentum bias, we simply multiply the number of states: for every state of the \citeauthor{Reali2009} model which represents a certain prevalence~$x$ of variant~$0$, we create two additional states, representing the same value of~$x$, but with copies of the state that indicate positive and negative momentum terms~$m$.

\begin{figure}

\newcommand{\N}{5}
\newcommand{\Nm}{4}
\newcommand{\Nmm}{3}

\begin{subfigure}[b]{\textwidth}
\centering
\begin{tikzpicture}[scale=1.7]
\tikzset{every state/.style={minimum size=1em}}
\node at (-1,-1) {x =};
\node[text=white] at (-2,-1) {m = -1}; % for identical graph width with below

\foreach \x in {0,...,\N} {
  \node at (\x,-1) {$\x$};
  \node[state] (\x) at (\x,0) {};
  \path[->] (\x) edge  [loop left] node {} ();
}

\foreach \x in {0,...,\Nm} {
  \foreach \y in {\N,...,\number\numexpr\x+1}
    \pgfmathtruncatemacro{\diff}{100*(\y-\x)/\N}
    \path[->,draw=white!\diff!black]
      (\x) edge [bend left] node {} (\y)
      (\y) edge [bend left] node {} (\x);
}
\end{tikzpicture}
\caption{State space of the~\citet{Reali2009} Markov model with $N=\N$.}\label{fig:markovstatespacesimple}
\end{subfigure}

\begin{subfigure}[b]{\textwidth}
\centering
\begin{tikzpicture}[scale=1.7]
\tikzset{every state/.style={fill=white,minimum size=1em}}

\node at (-1,-2) {x =};
\node at (-2,-1) {m = -1};
\node at (-2,0) {m = 0};
\node at (-2,1) {m = 1};

% momentum states
\foreach \x in {0,...,\Nm} {
  \pgfmathtruncatemacro{\next}{\x + 1}
  \node[state] (\x-1) at (\x,-1) {};
  \node[state] (\next+1) at (\next,1) {};
  \pgfmathtruncatemacro{\diff}{100/\N}
  \path[<->,draw=white!\diff!black] % immediate reversal
    (\x-1) edge node {} (\next+1); % [bend left=15] to avoid crossing loops
}

% paths along momentum states
\foreach \x in {\Nm,...,1} {
  \pgfmathtruncatemacro{\prev}{\x-1}
  \pgfmathtruncatemacro{\next}{\x+1}

%  \path[->] (\x-1) edge node {} (\prev-1)
%            (\x+1) edge node {} (\next+1);

% along top to right
%  \ifnum \prev>0, draw from \number\numexpr\prev-1\relax only
    \foreach \y in {\prev,...,0} {
      \pgfmathtruncatemacro{\diff}{100*(\x-\y-1)/\N}
      \path[->,draw=white!\diff!black]
        (\x-1) edge [bend left] node {} (\y-1);
    }
%  \fi
% along bottom to left
  \foreach \y in {\N,...,\next} {
    \pgfmathtruncatemacro{\diff}{100*(\y-\x-1)/\N}
    \path[->,draw=white!\diff!black]
      (\x+1) edge [bend left] node {} (\y+1);
  }
}

% long distance reversals
\foreach \x in {0,...,\Nmm} {
  \pgfmathtruncatemacro{\nnext}{\x + 2}
  \foreach \y in {\nnext,...,\N} {
    \pgfmathtruncatemacro{\diff}{100*(\y-\x)/(\N+1)}
    \path[<->,draw=white!\diff!black] (\x-1) edge node {} (\y+1);
    % [bend left=15]
  }
}

% momentum-free states
\foreach \x in {0,...,\N} {
  \node at (\x,-2) {$\x$};
  \node[state] (\x0) at (\x,0) {};
}

% straight paths out of and into momentum-free states
\foreach \x in {0,...,\Nm} {
  \pgfmathtruncatemacro{\next}{\x+1}
  % initiate increase (longest jumps first)
  \foreach \y in {\N,...,\next} {
    \pgfmathtruncatemacro{\diff}{100*(\y-\x-1)/\N}
    \path[->,draw=white!\diff!black] (\x0) edge node {} (\y+1);
  }
  % initiate decrease
  \foreach \y in {0,...,\x} {
    \pgfmathtruncatemacro{\diff}{100*(\x-\y)/\N}
    \path[->,draw=white!\diff!black] (\next0) edge node {} (\y-1);
  }
  % stalling
  \pgfmathtruncatemacro{\diff}{200/\N}
  \path[->,draw=white!\diff!black]
    (\x-1) edge node {} (\x0)
    (\next+1) edge node {} (\next0);
}
% old: curved initation straight to goal
%\pgfmathtruncatemacro{\diff}{100*(\N-1)/\N}
%\path[->,draw=white!\diff!black]
%  (00) edge [bend left=53] node {} (\N+1)
%  (\N0) edge [bend left=53] node {} (0-1);


% completion (drawn in gray right above by default)
%\path[->]
%  (0-1) edge node {} (00)
%  (\N+1) edge node {} (\N0);

% loops
\path[->,out=205,in=155] \foreach \x in {0,...,\N}
  {(\x0) edge [loop] node {} ()};


\end{tikzpicture}

\caption{State space of the Markov model with $N=\N$ and a momentum bias $b>0$.}\label{fig:markovstatespacemomentum}
\end{subfigure}

\caption[Schematic visualisation of the Markov model state space]{Schematic visualisation of the Markov model state space. Colouring of the edges indicates relative probability of the transitions, with darker edges representing more likely transitions.} \label{fig:markovstatespace}
\end{figure}

A schematic visualisation of the shape of this state space is shown in Figure~\ref{fig:markovstatespacemomentum}. While the number of states of this momentum model is almost threefold in comparison to the baseline model, the pattern of transitions is actually not much more complex. In particular, the model is not fully connected: every state has exactly $N$ outward transitions, exactly one each to every level of $x=0\ldots N$. 
The semantics of these three parallel states determines which of them a given previous state will transition into: all transitions from states with a lower to a higher~$x$ go into the $m=1$ state, transitions from higher to lower values of $x$ go into the $m=-1$ state, and only transitions from identical values of~$x$ enter the state with~$m=0$, indicating stagnation and therefore the absence of a trend.

In order to affect the dynamics of the system, the probabilities of transitioning between different levels of~$x$ are affected by the value of the momentum term: for the middle row with $m=0$, the probabilities of producing a given~$x$ are equivalent to the \citeauthor{Reali2009} model, which means they correspond to the Wright-Fisher model with innovation only, as in Equation~\ref{eq:mutation}.

For the upper and lower rows in the diagram, corresponding to a positive~(top)
 or negative~(bottom) momentum term, the momentum affects the probabilities of producing a certain number of $x$~tokens by exerting a selection pressure on the variant that is currently `trending', i.e.~whose frequency~$x$ has increased at the last time step. To this end the transition probabilities out of a state with $m=1$ are calculated according to the Wright-Fisher model with innovation and selection as given in Equation~\ref{eq:mutationselection}, with the selection coefficient~$s$ in favour of the incoming variant~1 set to a fixed constant. Conversely, the transition probabilities for states with~$m=-1$ are controlled by the same equation, only that the same selection coefficient is selecting variant~1. Note how, as long as the innovation probabilities are equal, this system is again symmetric, and thus replicator-neutral. Even though there are clear paths of directed selection, e.g.~towards higher values of~$x$ along the top of the state space, these paths are mirrored exactly on the other side. A concrete example of a Markov chain transition matrix for such a momentum model is given in Table~\ref{tbl:momentumtransitionmatrix}.

% latex table generated in R 3.2.3 by xtable 1.8-2 package
% Wed Jul 13 17:10:00 2016
\begin{sidewaystable}[htbp]
\centering
\begin{tabular}{rrrrrrrrrrrrr}
  \hline
 & $x'=0-$ & $x'=0=$ & $x'=0+$ & $x'=1-$ & $x'=1=$ & $x'=1+$ & $x'=2-$ & $x'=2=$ & $x'=2+$ & $x'=3-$ & $x'=3=$ & $x'=3+$ \\ 
  \hline
$x=0-$ & 0.0000 & 0.9524 & 0.0000 & 0.0000 & 0.0000 & 0.0468 & 0.0000 & 0.0000 & 0.0008 & 0.0000 & 0.0000 & 0.0000 \\ 
  $x=0=$ & 0.0000 & 0.9524 & 0.0000 & 0.0000 & 0.0000 & 0.0468 & 0.0000 & 0.0000 & 0.0008 & 0.0000 & 0.0000 & 0.0000 \\ 
  $x=0+$ & 0.0000 & 0.9524 & 0.0000 & 0.0000 & 0.0000 & 0.0468 & 0.0000 & 0.0000 & 0.0008 & 0.0000 & 0.0000 & 0.0000 \\ 
  $x=1-$ & 0.3656 & 0.0000 & 0.0000 & 0.0000 & 0.4371 & 0.0000 & 0.0000 & 0.0000 & 0.1742 & 0.0000 & 0.0000 & 0.0231 \\ 
  $x=1=$ & 0.2892 & 0.0000 & 0.0000 & 0.0000 & 0.4444 & 0.0000 & 0.0000 & 0.0000 & 0.2276 & 0.0000 & 0.0000 & 0.0389 \\ 
  $x=1+$ & 0.2189 & 0.0000 & 0.0000 & 0.0000 & 0.4329 & 0.0000 & 0.0000 & 0.0000 & 0.2855 & 0.0000 & 0.0000 & 0.0627 \\ 
  $x=2-$ & 0.0627 & 0.0000 & 0.0000 & 0.2855 & 0.0000 & 0.0000 & 0.0000 & 0.4329 & 0.0000 & 0.0000 & 0.0000 & 0.2189 \\ 
  $x=2=$ & 0.0389 & 0.0000 & 0.0000 & 0.2276 & 0.0000 & 0.0000 & 0.0000 & 0.4444 & 0.0000 & 0.0000 & 0.0000 & 0.2892 \\ 
  $x=2+$ & 0.0231 & 0.0000 & 0.0000 & 0.1742 & 0.0000 & 0.0000 & 0.0000 & 0.4371 & 0.0000 & 0.0000 & 0.0000 & 0.3656 \\ 
  $x=3-$ & 0.0000 & 0.0000 & 0.0000 & 0.0008 & 0.0000 & 0.0000 & 0.0468 & 0.0000 & 0.0000 & 0.0000 & 0.9524 & 0.0000 \\ 
  $x=3=$ & 0.0000 & 0.0000 & 0.0000 & 0.0008 & 0.0000 & 0.0000 & 0.0468 & 0.0000 & 0.0000 & 0.0000 & 0.9524 & 0.0000 \\ 
  $x=3+$ & 0.0000 & 0.0000 & 0.0000 & 0.0008 & 0.0000 & 0.0000 & 0.0468 & 0.0000 & 0.0000 & 0.0000 & 0.9524 & 0.0000 \\ 
   \hline
\end{tabular}
\caption[Markov chain transition matrix with momentum]{Markov chain transition matrix with momentum, $N=80$, $\alpha=0.1, b=0.3$. The sign at the end of the name of the state indicates whether the state's momentum is positive~(upward-trending), negative~(downward-trending) or neutral~(stagnant).} 
\label{tbl:momentumtransitionmatrix}
\end{sidewaystable}


\subsection{The interaction of momentum and innovation}\index{momentum-based selection}

So how do the dynamics of this momentum selection model differ from the original innovation-only version of the Wright-Fisher model? A direct comparison of the two models' stationary distributions for various settings of the regularisation parameter~$\alpha$ and momentum bias strength~$b$ is shown in Figure~\ref{fig:momentumstationary}. While a population size of $N=80$ only allows for 81 different states in terms of the \emph{frequency} distribution of the competing variants, the corresponding momentum model possesses 241~states. To aid interpretability, the stationary distributions are therefore grouped by the absolute frequency~$x$ of the variants, with the stationary probability of states with the same frequency but different momentum values stacked on top of each other. The three colours indicate the momentum: the red portions show the probability of being in a state with positive momentum, where productions are biased towards higher levels of~$x$, while the blue sections express the same but for negative momentum. The white sections of the bars represent the probability of being in a state with a momentum of~$0$, resulting in sampling that is not biased towards either variant. %The order of colouring is inverted around the mid-point merely to highlight that, assuming equal innovation rates for both variants, a momentum-based selection bias is indeed completely symmetric.

In order to confirm that this specific model of selection works as expected, we can set the momentum bias to~$b=0$, in which case we recover exactly the same stationary distribution as the Wright-Fisher model with innovation only, as shown in Figure~\ref{fig:momentumstationary1}.
Despite the selection bias being ineffective in this model, the colour indication of the momentum term is still informative: it shows how much of the time the model \emph{remains} at a given proportion, resulting in a momentum term of~$0$ as indicated in white. As the impact of the momentum bias is increased, the probability of remaining in a state of mixed usage of variants decreases, as can be seen in Figure~\ref{fig:momentumstationary1}.

\begin{knitrout}
\definecolor{shadecolor}{rgb}{0.969, 0.969, 0.969}\color{fgcolor}\begin{figure}[htbp]

{\centering \subfloat[Without a momentum bias~($b=0$) the stationary distribution is identical to \citet{Reali2009}'s \emph{averaging} learner~(compare Figure~\ref{fig:averagerstationarydistribution}). The colour-coded differentiation between the three different states for each usage level makes the temporal dynamics of the model more explicit, but the extra states do not otherwise change the dynamics of the original model.\label{fig:momentumstationary1}]{\includegraphics[width=\maxwidth]{figure/momentumstationary-1} }
\subfloat[With a momentum bias of $b=3$, the model naturally avoids states corresponding to mixed usage of the competing variants, as it preferentially sweeps through this middle region in a directed fashion. The higher the momentum bias, the less time the model spends in regions of mixed usage of the two variants.\label{fig:momentumstationary2}]{\includegraphics[width=\maxwidth]{figure/momentumstationary-2} }

}

\caption[Stationary distributions of Markov chains with momentum.]{Stationary distributions of Markov chains with momentum in a population of $N=10$ for various regularisation parameters~$\alpha$ and momentum bias strengths~$b$.}\label{fig:momentumstationary}
\end{figure}


\end{knitrout}

So how does the model with a momentum bias differ from a simple model of regularisation through symmetric selection of the more frequent variant? To appreciate the temporal dynamics of the momentum-based Wright-Fisher model we have to again move beyond simply looking at its stationary distributions and instead consider the likelihood of actuated transitions to diffuse to the entire population, as was done in the analyses above. 
In the absence of momentum-based selection shown in Figure~\ref{fig:symmetricmomentumcompletionprobabilities}(i) we again recover the same dynamics as for simple symmetric innovation~(compare Figure~\ref{fig:symmetricmutation2}), where the diffusion probabilities do not stray far from the neutral evolution baseline.
This changes drastically with the introduction of a momentum bias shown in Figure~\ref{fig:symmetricmomentumcompletionprobabilities}(ii) and~(iii): the panels show the probability of diffusing based on the model states with the given frequency that correspond to a positive momentum term, i.e.~trajectories that have come to occupy the relative frequency of that variant as part of a recent rise in frequency. Trajectories that start off with positive momentum have significantly higher completion probabilities, even at low initial frequencies of the variant.

Interestingly, this effect is slightly reversed for states corresponding to near completion. As was shown above, changes in variant frequency due to selection pressures slow down as the variant starts to prevail in the population.
The increased likelihood of stagnation towards the end of transitions in combination with the simplistic ternary distinction of momentum based on the direction of change at the last time step in the present model means that, the initiation of a momentum-driven trend in the opposite direction becomes more likely as the changes slow down near completion. This artefact of the current toy model would therefore be greatly alleviated by a more gradual measure of trends that captures directedness on a greater time depth, such as the one implemented in Chapter~\ref{ch:momentummodel}.
%The possibility of spontaneous innovation of the competing variant

\begin{knitrout}
\definecolor{shadecolor}{rgb}{0.969, 0.969, 0.969}\color{fgcolor}\begin{figure}[htbp]

{\centering \includegraphics[width=\maxwidth]{figure/symmetricmomentumcompletionprobabilities-1} 

}

\caption[Probability of an actuated transition completing.]{Probability of an actuated transition completing as a function of the initial frequency of the variant, assuming that the variant has positive~(upwards) momentum. Innovation probabilities of both variants are equal, with~$\alpha/2=0.1$.}\label{fig:symmetricmomentumcompletionprobabilities}
\end{figure}


\end{knitrout}

A fact that is possibly not immediately appreciated about the present diffusion probability plot is that the increased average likelihood of completion actually holds for \emph{both} of the competing variants.
In the earlier analysis of the general Wright-Fisher model it was sufficient to only plot the completion probabilities of the incoming variant. The likelihood of the other variant succeeding was simply the complement of the probability for its competitor, meaning that an increased success rate of one variant was automatically associated with a decrease in the other.
 Under the present model on the other hand we are particularly interested in the completion probabilities of directed transitions which have positive momentum. When taking this temporal dimension of our model into account while calculating the respective diffusion probabilities for either variant, we are therefore not actually computing those probabilities based on the same initial state, but based on the two corresponding states with positive and negative momentum respectively. The important consequences of this point will become even more apparent when we consider the case of momentum-based selection applying on top of asymmetric innovation rates.

\subsection{Momentum and asymmetric innovation}\index{momentum-based selection}\index{pressures!interaction of|textbf}

In the simple Wright-Fisher model with asymmetric innovation rates discussed earlier, the preferentially innovated variant was shown to dominate the dynamics, with the model by remaining in a state of categorical usage of that variant most of the time.
However, this bias towards one variant was not strongly evident in the dynamics of individual transitions, where even strong asymmetries resulted in only slight increases in the diffusion probability of the preferred variant at the expense of competing variants, indicative of rare and noisy transitions between the two extreme states.

While Figure~\ref{fig:asymmetricmomentumcompletionprobabilities1} shows that having symmetric momentum-based selection on top of asymmetric innovation biases does not significantly alter the expected synchronic distribution of variants, the probabilities of diffusion out of states with positive momentum draw a picture of a very different dynamic.
Figure~\ref{fig:asymmetricmomentumcompletionprobabilities2} shows the probability of successful diffusion for both of the competing variants across different initial frequencies. While in the case of asymmetric innovation without a momentum bias~($b=0$) an increase in one variant's diffusion probability entails a decrease in that of the other, the presence of a momentum bias actually raises the likelihood of successful transitions above the neutral evolution baseline for \emph{both} of the variants, albeit at slightly different rates.
That there is still a difference between the diffusion probabilities of the two variants throughout can be explained by the asymmetry in innovation, since the less frequently generated variant is still slightly more likely to have its directed transition interrupted by the spontaneous generation of an instance of its competitor variant.

\begin{knitrout}
\definecolor{shadecolor}{rgb}{0.969, 0.969, 0.969}\color{fgcolor}\begin{figure}[htbp]

{\centering \subfloat[Stationary distributions. The presence of a momentum bias increases the relative likelihood of synchronically observing the less frequently innovated variant.\label{fig:asymmetricmomentumcompletionprobabilities1}]{\includegraphics[width=\maxwidth]{figure/asymmetricmomentumcompletionprobabilities-1} }
\subfloat[Completion probabilities for transitions where the incoming variant is the one that is \emph{more} likely to be introduced through innovation~(plusses) as well as for transitions where the incoming variant is the one that is \emph{less} likely to be introduced through innovation~(crosses).\label{fig:asymmetricmomentumcompletionprobabilities2}]{\includegraphics[width=\maxwidth]{figure/asymmetricmomentumcompletionprobabilities-2} }

}

\caption[Dynamics of momentum-based selection with asymmetric innovation.]{Dynamics of momentum-based selection with asymmetric innovation probabilities for different strengths of the momentum bias~$b$, with $\alpha_1=0.2$, $\alpha_1/\alpha_0=10$.}\label{fig:asymmetricmomentumcompletionprobabilities}
\end{figure}


\end{knitrout}

\subsection{Momentum, asymmetric innovation and the synchronic distribution of variants}

After having considered several different pressures and combinations of pressures in this chapter we found that, based on the Wright-Fisher model as a simplified model of cultural evolution, only the presence of biases for the spontaneous \emph{innovation} of new variants in combination with a symmetric selection bias such as momentum-based selection was able to capture both asymmetries in synchronic distributions while at the same time producing directed transitions towards the categorical usage of both preferred as well as less frequently attested variants.
%which do not amplify any asymmetric biases so much as to predict only the preferred variant to be attested cross-linguistically.
%Figure~\ref{fig:asymmetricmomentumcompletionprobabilities} shows that transitions are possible in both directions

Even though the present model of the interaction between asymmetric variation and momentum-based selection is highly simplified, it is interesting to get an idea of the degree to which the momentum-based selection bias projects asymmetries in innovation onto population-level distributions of variants. In other words, assuming that individual language changes are driven by replicator-neutral trend-amplification mechanisms, to what degree should we expect innovation biases that we find in individuals to be reflected in cross-linguistic distributions? To investigate this relationship, Figure~\ref{fig:innovationprobabilities2} plots the relative synchronic frequency of the two competing variants as a function of an increasing difference in their innovation probabilities. Here, we return to the direct specification of the innovation rates~$\mu$, since we are actually interested in how one and the same probability of spontaneously producing a variant interacts with population size, rather than trying to model a specific regime of (de-)regularisation.

Figure~\ref{fig:innovationprobabilities2}(i) shows the effect of increasing the parameter~$N$, indicating population size in the biological framing of the Wright-Fisher model, or otherwise the size of the learning sample in \citeauthor{Reali2009}'s model of iterated learning. In either case, the argument will correspond to a measure of how precisely variable usage of competing variants can be represented, either as distributed knowledge across the population or within the individual. While we find a linear mapping from innovation asymmetries to predicted synchronic frequency for all population sizes, the strength of this mapping \emph{decreases} with increasing population size so that, assuming the same absolute level of innovation rates, we should find asymmetries in innovation to be relatively \emph{less} expressed in larger populations. % or in more frequent variables

Figure~\ref{fig:innovationprobabilities2}(ii) shows how different baseline rates of innovation for both variants affect the degree to which any asymmetry should be reflected in synchronic data. The results indicate that, the \emph{less} likely spontaenous innovations in a specific trait occur, the \emph{more} should any asymmetries in those innovation probabilities be manifested in synchronic data.

While general trends can be derived from the present models, caution should be taken in attempting to derive empirical predictions about social factors such as community size from them directly. To be more precise, the predictions shown here are based on the assumption that innovation rates are constant across populations, which is not necessarily the case since innovation might itself be affected by external pressures. A momentum-based selection account thus offers another explanation of how environmental and sociocultural factors that differ between societies can affect the evolution of their linguistic systems, but through steering the \emph{innovation} of novel variants and traits, rather than by causing the selection of specific variants directly.

\begin{knitrout}
\definecolor{shadecolor}{rgb}{0.969, 0.969, 0.969}\color{fgcolor}\begin{figure}[htbp]

{\centering \subfloat[Stationary distributions with asymmetric variation, where~$\alpha_1=0.2$.\label{fig:innovationprobabilities1}]{\includegraphics[width=\maxwidth]{figure/innovationprobabilities-1} }
\subfloat[Expected odds of synchronically observing one variant over the other as a function of the odds of the spontaneous innovation of one variant over the other. The degree to which the same asymmetry in innovation is expressed in synchronic distributions varies, depending on \emph{(i)}~the Wright-Fisher model parameter $N$, corresponding to population or sample size~(data for~$\mu_1=0.015$), and \emph{(ii)}~absolute innovation probabilities, given $N=20$.\label{fig:innovationprobabilities2}]{\includegraphics[width=\maxwidth]{figure/innovationprobabilities-2} }

}

\caption[Expected synchronic distribution of variants given asymmetric innovation probabilities with momentum-based selection.]{Expected synchronic distribution of variants given asymmetric innovation probabilities with momentum-based selection~($b=1$).}\label{fig:innovationprobabilities}
\end{figure}


\end{knitrout}
